\documentclass[a4paper]{article}
\usepackage{amsmath, amssymb, amsfonts}
\usepackage[margin=1in]{geometry}
\usepackage{graphicx}
\usepackage{tikz}
\usepackage{esint}
\setlength{\parindent}{0em}
\setlength{\parskip}{1ex}
\newcommand{\vct}[1]{\overrightarrow{#1}}
\newcommand{\pd}[2]{\frac{\partial {#1}}{\partial {#2}}}
\newcommand{\dd}[2]{\frac{\mathrm{d} {#1}}{\mathrm{d} {#2}}}
\newcommand{\C}{\mathbb{C}}
\newcommand{\R}{\mathbb{R}}
\newcommand{\Q}{\mathbb{Q}}
\newcommand{\Z}{\mathbb{Z}}
\newcommand{\N}{\mathbb{N}}
\newcommand{\fn}[3]{{#1}\colon {#2} \rightarrow {#3}}
\newcommand{\avg}[1]{\langle {#1} \rangle}
\newcommand{\Sum}[2][0]{\sum_{{#2} = {#1}}^{\infty}}
\newcommand{\Lim}[1]{\lim_{{#1} \rightarrow \infty}}
\newcommand{\Binom}[2]{\begin{pmatrix} {#1} \cr {#2} \end{pmatrix}}


\begin{document}
\paragraph{1. naloga} Razvij funkcijo $f(z) = z^2e^{1/z}$ v Laurentovo vrsto s središčem v 0 na kolobarju $0<|z|$.
$$e^z = 1 + z + \frac{z^2}{2} + ...$$
$$z^2e^{1/z} = z^2 + z + \frac{1}{2} + \frac{1}{6z} + ...$$
Dana je funkcija $\displaystyle{\frac{1}{(z-1)(z-2)}}$. Razvij jo v Laurentovo vrsto na kolobarju
\begin{itemize}
    \item $0<|z-1|<1$
    \item $|z-1|>1$
\end{itemize}
Prvi del:
$$\frac{1}{(z-1)(z-2)}=\frac{1}{z-1}\cdot\frac{1}{z-2}=\frac{1}{z-1}\cdot\frac{1}{(z-1)-1}$$
$$=-\frac{1}{z-1}\frac{1}{1-(z-1)}=\frac{1}{z-1}\left(1+(z-1)+(z-1)^2+...\right)=\frac{1}{z-1}-1-(z-1)-(z-1)^2-...$$
Drugi del:
$$\frac{1}{(z-1)(z-2)}=\frac{1}{(z-1)^2}\left(\frac{1}{1-\frac{1}{z-1}}\right)=\frac{1}{(z-1)^2}\left(1+(z-1)^{-1}+(z-1)^{-2}+...\right)$$
$$=\frac{1}{(z-1)^2}+\frac{1}{(z-1)^3}+\frac{1}{(z-1)^4}+...$$
\paragraph{2. naloga} Razvij $\displaystyle{\frac{1}{z(z-1)(z-2)}}$ v Laurentovo vrsto s središčem v $z_0=0$ na kolobarjih
\begin{itemize}
    \item $0 < |z| < 1$
    \item $1 < |z| < 2$
    \item $|z| > 2$
\end{itemize}
$$\frac{1}{z(z-1)(z-2)} = \frac{A}{z} + \frac{B}{z-1} + \frac{C}{z-2}$$
Dobimo sistem enačb:
\begin{eqnarray*}
    A + B + C = 0 \\
    -3A -2B - C = 0 \\
    2A = 1
\end{eqnarray*}
Dobimo $\displaystyle{f(z) = \frac{1}{2z} - \frac{1}{z-1} + \frac{1}{2(z-2)}}$
$$= \frac{1}{z} + \frac{1}{1-z} + \frac{\frac{1}{4}}{1 - \frac{z}{2}}$$
Nato razvijemo vsako posebej.
\paragraph{3. naloga} Dana je $f(z) = \displaystyle{\frac{1}{e^z-1}}$.
\begin{itemize}
    \item Določi definicijsko območje $f$
    \item Zapiši prve štiri člene Laurentove vrste. Kje dobljena vrsta konvergira?
\end{itemize}
Funkcija $f$ ni definirana, kadar je $e^z-1=0$
$$e^z = e^{x+iy} = e^xe^{iy} = e^x(\cos y + i\sin y)$$
$|\cos y + i\sin y| = 1$, torej dobimo prvo zahtevo: $x=0$.
Zdaj želimo še, da je $\cos y + i\sin y = 1$. To dosežemo pri $y = 2\pi k$, kjer je $k\in\Z$. \\
Drugi del:
$$\frac{1}{e^z-1}=\frac{1}{1 + z + \frac{z^2}{2} + \frac{z^3}{6} + ... - 1} = \frac{1}{z}\cdot\frac{1}{1 + \frac{z}{2} + \frac{z^2}{6} + ...}$$
Drugi faktor lahko zamenjamo z vrsto za $\displaystyle{{1}\over{1+x}}$, kar pa lahko storimo le, če je $\displaystyle{\left|\frac{z}{2} + \frac{z^2}{6} + ...\right|<1}$.
Ker je ta vrsta holomorfna, gotovo obstaja tak $r>0$, da je za vse $|z|<r$ to res.
$$f(z) = \frac{1}{z}\left(1 - \frac{z}{2} + z^2\left[\frac{1}{4}-\frac{1}{6}\right] + z^3\left[\frac{1}{24} + \frac{1}{6} + \frac{1}{8}\right] + \dots\right)$$
$$= \frac{1}{z} - \frac{1}{2} + \frac{1}{12}z + 0\cdot z^2$$
\end{document}