\documentclass[a4paper]{article}
\usepackage{amsmath, amssymb, amsfonts}
\usepackage[margin=1in]{geometry}
\usepackage{graphicx}
\usepackage{tikz}
\usepackage{esint}
\setlength{\parindent}{0em}
\setlength{\parskip}{1ex}
\newcommand{\vct}[1]{\overrightarrow{#1}}
\newcommand{\dif}{\mathrm{d}}
\newcommand{\pd}[2]{\frac{\partial {#1}}{\partial {#2}}}
\newcommand{\dd}[2]{\frac{\mathrm{d} {#1}}{\mathrm{d} {#2}}}
\newcommand{\C}{\mathbb{C}}
\newcommand{\R}{\mathbb{R}}
\newcommand{\Q}{\mathbb{Q}}
\newcommand{\Z}{\mathbb{Z}}
\newcommand{\N}{\mathbb{N}}
\newcommand{\fn}[3]{{#1}\colon {#2} \rightarrow {#3}}
\newcommand{\avg}[1]{\langle {#1} \rangle}
\newcommand{\Sum}[2][0]{\sum_{{#2} = {#1}}^{\infty}}
\newcommand{\Lim}[1]{\lim_{{#1} \rightarrow \infty}}
\newcommand{\Int}{\int_{-\infty}^{\infty}}
\newcommand{\Binom}[2]{\begin{pmatrix} {#1} \cr {#2} \end{pmatrix}}
\newcommand{\duline}[1]{\underline{\underline{#1}}}

\begin{document}
\paragraph{Fourierjeva transformacija.} Če je $\fn{f}{\R}{\C}$, je njena Fourierjeva transformiranka definirana kot $$\widehat{f}(\xi) = \frac{1}{\sqrt{2\pi}}\Int f(x)e^{-ix\xi}\dif x$$
Ima sledeče lastnosti:
$$\widehat{f(x)e^{itx}}(\xi) = \widehat{f}(\xi - t)$$
$$\widehat{f(ax)} = \frac{1}{a}\widehat{f}\left(\frac{\xi}{a}\right)$$
$$\widehat{f(x-t)}(\xi) = \widehat{f}(\xi) e^{-it\xi}$$
Inverzna Fourierjeva transformacija:
$$f(x) = \frac{1}{\sqrt{2\pi}}\Int \widehat{f}(\xi) e^{-ix\xi}\dif\xi$$
Velja $\widehat{\widehat{f}}(x) = f(-x)$
\paragraph{Tabela znanih transofrmacij.} \text{}
\begin{table}[h!]
    \centering
    \begin{tabular}{c|c}
        \hline
        $f(x)$ & $\widehat{f}(\xi)$ \\[2mm]
        \hline
        $e^{-|x|}$ & $\sqrt{\frac{1}{\pi}} \frac{1}{1+\xi^2}$ \\[2mm]
        $e^{-a^2x^2/2}$ & $\frac{1}{a}e^{-\xi^2/2a^2}$ $(a>0)$ \\[2mm]
        $\chi_{[a, a]}(x)$ & $\sqrt{\frac{2}{\pi}}\frac{\sin(a\xi)}{\xi}$ \\[2mm]
        $\frac{1}{1+x^2}$ & $\sqrt{\frac{\pi}{2}}e^{-|\xi|}$ \\[2mm]
        $e^{-ax^2}$ & $\frac{1}{\sqrt{2a}} e^{-\xi^2/4a}$
    \end{tabular}
\end{table}
Ravno tako pa prav pride sledeče dejstvo:
$$\widehat{f'(x)} = i\xi\widehat{f}(\xi)$$
$$\widehat{f''(x)} = -\xi^2\widehat{f}(\xi)$$
\end{document}