\documentclass[a4paper]{article}
\usepackage{amsmath, amssymb, amsfonts}
\usepackage[margin=1in]{geometry}
\usepackage{graphicx}
\usepackage{tikz}
\usepackage{esint}
\setlength{\parindent}{0em}
\setlength{\parskip}{1ex}
\newcommand{\vct}[1]{\overrightarrow{#1}}
\newcommand{\pd}[2]{\frac{\partial {#1}}{\partial {#2}}}
\newcommand{\dd}[2]{\frac{\mathrm{d} {#1}}{\mathrm{d} {#2}}}
\newcommand{\C}{\mathbb{C}}
\newcommand{\R}{\mathbb{R}}
\newcommand{\Q}{\mathbb{Q}}
\newcommand{\Z}{\mathbb{Z}}
\newcommand{\N}{\mathbb{N}}
\newcommand{\fn}[3]{{#1}\colon {#2} \rightarrow {#3}}
\newcommand{\avg}[1]{\langle {#1} \rangle}
\newcommand{\Sum}[2][0]{\sum_{{#2} = {#1}}^{\infty}}
\newcommand{\Lim}[1]{\lim_{{#1} \rightarrow \infty}}
\newcommand{\Binom}[2]{\begin{pmatrix} {#1} \cr {#2} \end{pmatrix}}


\begin{document}
\section{$\Gamma$ funkcija}
\paragraph{Izrek.} Kompleksna $\Gamma$ funkcija je holomorfna.
\paragraph{Dokaz.} Definiramo zaporedje funkcij $$F_n(z)=\int_{1/n}^{n}e^{z-1}e^{-t}dt$$
$F_n(z)$ so holomorfne, ker je $f(z, t) \to t^{z-1}e^{-t}$ zvezna in je njen odvod zvezen. Očitno zaporedje $(F_n)n\in\N$ konvergira proti $\Gamma$ funkciji,
torej moramo pokazati le še, da je ta konvergenca enakomerna. To dokažemo v dveh korakih: \\
\begin{enumerate}
    \item Pokazati moramo, da je za vsak $z$ iz definicijskega območja $$\Lim{n} \int_{0}^{1/n} t^{z-1}e^{-t}dt = 0$$
    \item in da je $$\Lim{n} \int_{n}^{\infty} t^{z-1}e^{-t}dt = 0$$
\end{enumerate}
Spomnimo se: $\displaystyle{t^{z-1} = e^{(z-1)\ln t}} = e^{(x-1)\ln t}\cdot e^{iy\ln t}$, če $z$ zapišemo kot $x + iy$. Sledi, da je $\displaystyle{|t^{z-1}| = t^{x-1}}$. \\[3mm]
$$\left|\int_{0}^{1/n}t^{z-1}e^{-t} dt\right| \leq \int_{0}^{1/n} t^{\mathfrak{Re}z - 1} e^{-t}dt \leq \int_{0}^{1/n} t^{M-1}e^{-t}dt \leq \int_{0}^{1/n}t^{M-1}dt = \frac{t^{M}}{M}\Bigg|_{0}^{1/n} = \frac{1}{M}\left(\frac{1}{n}\right)^M$$
Ocenili smo $\mathfrak{Re}(z) \leq M$.
$$\left|\int_{n}^{\infty}t^{z-1}e^{-t}dt\right| \leq \int_{n}^{\infty} t^{M-1}e^{-t}dt$$
Tak integral pa veno konvergira.
\paragraph{Opomba.} Kakor v množici realnih števil tudi tu velja $\displaystyle{\Gamma(z+1)=z\Gamma(z)}$.
Če je $\mathfrak{Re}(z) \in (-1, 0)$, lahko še vedno definiramo $\Gamma$ funkcijo kot $\displaystyle{\Gamma(z) = \frac{1}{z}\Gamma(z+1)}$, saj je $\Gamma(z+1)$ gotovo definirana. Podobno lahko funkcijo definiramo povsod, kjer $\mathfrak{Re}(z)$ ni negativno celo število ali 0.
\paragraph{Zanimivost.} Še ena slavna funkcija je Riemannova Zeta funkcija:
$$\zeta(s) = \Sum[1]{n}\frac{1}{n^s}$$
Domneva se, da ima ničle le vzdolž osi $\mathfrak{Re}(z) = 1/2$, vendar tega še nikomur ni uspelo dokazati.
\section{Harmonične funkcije}
\paragraph{Definicija.} Bodi $\mathcal{D}$ odprta v $\R^n$ in bodi $\fn{u}{\mathcal{D}}{\R}$ dvakrat zvezno odvedljiva.
Pravimo, da je $u$ harmonična na $\mathcal{D}$, če je
$$\Delta u = 0$$
Pri tem smo z $\Delta$ označili Laplaceov operator $\displaystyle{\Delta u = \pd{^2u}{x_1^2}} + \pd{^2u}{x_2^2} + ... + \pd{^2u}{x_n^2}$.
\paragraph{Zgled.} Obravnavajmo najprej primer $n=1$.
$$\dd{u^2}{x^2} = 0 \Rightarrow u = Ax + B$$
Kjer sta $A, B$ konstanti. \\[3mm]
Poseben primer so harmonične funkcije, ki so radialno simetrične, torej odvisne le od $r$, kjer je $r$ oddaljenost telesa glede na koordinatno izhodišče. Označimo $f(r) = u(x_1 + x_2 + ... + r_n)$
$$r = |x| = \sqrt{x_1^2 + x_2^2 + ... + x_n^2}$$
$$\pd{u}{x_i} = \pd{u}{r} \cdot \pd{r}{x_i} = f'(r) \frac{x_i}{r}$$
$$\pd{^2u}{x_i^2} = \pd{}{x_i}\left(\frac{f'(r)}{r}\cdot x_i\right) = \frac{f''(r)\cdot r - f'(r)}{r^3} x_i^2 + \frac{f'(r)}{r}$$
$$\Delta u = f''(r) + (n+1)\frac{f'(r)}{r} \equiv 0$$
Rešujemo diferencialno enačbo ene spremenljivke. Uvedemo $g(r) = f'(r)$:
$$g'(r) = (1-n)\frac{g(r)}{r}$$
Ločimo spremenljivke in integriramo na obeh straneh.
$$g(r) = Ar^{1-n}$$
Nato lahko izrazimo še $f$:
$$f(r) = \begin{cases}
    A\cdot\ln(r) + B & n = 2 \\
    A \frac{1}{r^{n-2}} + B & n \neq 2 \\
\end{cases}$$
$$u(\mathbf{x}) = \begin{cases}
    A \ln(|x|) + B & n = 2 \\
    A \frac{1}{|x|^{n-2}} + B & n \neq 2 \\
\end{cases}$$
Običajno si za $B$ izberemo vrednost $B=0$, za $A$ pa
$$A = \begin{cases}
    \frac{1}{2\pi} & n = 2 \\
    \frac{1}{(2-n)w_n} & n \neq 2 \\
\end{cases}$$
$$w_n = \frac{2\pi^{n/2}}{\Gamma(n/2)}$$
$w_n$ je površina n-dimenzionalne sfere s polmerom 1. Za $n=3$ je to ravno $4\pi$.
Funkcijam $u$ s to izbior konstant pravimo Newtonovi potenciali.
\paragraph{Harmonične funkcjie v $\R^2$} Bodi $\fn{\mathcal{U}}{\mathcal{D}}{\R}$ taka, da je $\Delta \mathcal{U} = 0$ in $\mathcal{D} \subseteq \R$.
$\mathcal{D}$ lahko obravnavamo kot podmnožico $\C$. Naj bo $f$ holomorfna na $D$:
$$f(z) = f(x + iy) = u(x, y) + iv(x, y)$$
Ker je $f$ holomorfna, je neskončnokrat odvedljiva. Tedaj je $$f'(x+iy) = u_x + iv_x = v_y -iu_y$$
$$\Delta u = u_{xx} + u_{yy}$$
Upoštevamo Cauchy-Riemannovi enakosti in dobimo
$$\Delta u = v_{xy} - v_{yx} = 0$$
Sledi: Če je $\fn{f}{\mathcal{D}}{\C}$ holomorfna, je $u = \mathfrak{Re}(f)$ harmonična na $\mathcal{D}$.
\paragraph{Izrek.} Bodi $\mathcal{D}$ enostavno povezano območje v $\R^2$ in funkcija $u$ na njem harmonična. Tedaj obstaja holomorfna funkcija $f$, da je $u = \mathfrak{Re}(z)$.
\paragraph{Dokaz.} Definirati moramo tako $v(x, y)$, da bo $f(x + iy) = u(x, y) + iv(x, y)$ holomorfna. Iz razlogov, ki bodo postali očitni pozneje, izberemo
$$v(x, y) = \int_{\gamma} (-u_y dx + u_x dy)$$
Pokazati moramo, da je ta definicija neočitna od izbire poti. Dovolj je pokazati, da je funkcija $u$ potencialno polje (to lahko storim npr. z Greenovo formulo).
\end{document}