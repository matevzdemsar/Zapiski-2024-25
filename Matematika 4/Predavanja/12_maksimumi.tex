\documentclass[a4paper]{article}
\usepackage{amsmath, amssymb, amsfonts}
\usepackage[margin=1in]{geometry}
\usepackage{graphicx}
\usepackage{tikz}
\setlength{\parindent}{0em}
\setlength{\parskip}{1ex}
\newcommand{\vct}[1]{\overrightarrow{#1}}
\newcommand{\pd}[2]{\frac{\partial {#1}}{\partial {#2}}}
\newcommand{\dd}[2]{\frac{\mathrm{d} {#1}}{\mathrm{d} {#2}}}
\newcommand{\C}{\mathbb{C}}
\newcommand{\R}{\mathbb{R}}
\newcommand{\Q}{\mathbb{Q}}
\newcommand{\Z}{\mathbb{Z}}
\newcommand{\N}{\mathbb{N}}
\newcommand{\fn}[3]{{#1}\colon {#2} \rightarrow {#3}}
\newcommand{\avg}[1]{\langle {#1} \rangle}
\newcommand{\Sum}[2][0]{\sum_{{#2} = {#1}}^{\infty}}
\newcommand{\Lim}[1]{\lim_{{#1} \rightarrow \infty}}
\newcommand{\Binom}[2]{\begin{pmatrix} {#1} \cr {#2} \end{pmatrix}}


\begin{document}
\paragraph{Izrek.} (Izrek o maksimumih in minimumih) \\
Bodi $\fn{f}{\mathcal{D}}{\C}$ holomorfna in nekonstantna. Potem $|f(z)|$ ne zavzame maksimuma na $D$. Minimum $f(z)$ lahko doseže le v ničlah $f$ na $D$.
\paragraph{Dokaz.} Recimo, da imamo tak $z_0\in\mathcal{D}$, da je $|f(z_0)|$ maksimalna možna na $D$.
Ker je $\mathcal{D}$ območje, je odprto, in tedaj je tudi $f(\mathcal{D})$ odprta. To pomeni, da obstaja tak $r>0$, da je $D(f(z_0), r) \subseteq f(\mathcal{D})$. Izmed točk znotraj $D(f(z_0), r)$ bo zagotovo kakšna
točka $w$, za katero bo veljalo $|w|>|f(z_0)|$, recimo, da je $w=f(z_1)$.
To pomeni, da je $|f(z_1)|>f(z_0)$ in prišli smo do protislovja. \\[4mm]
Kar pa se tiče minimumov: Recimo, da $f(z)$ zavzame minimum v točki $z_0 \in \mathcal{D}$.
Ker $f$ nima ničel, je $\displaystyle{g(z) = \frac{1}{f(z)}}$ holomorfna.
Tedaj bi imela $g$ maksimum v $z_0$, ravnokar pa smo dokazali, da to ni mogoče.
\paragraph{Posledica.} Bodi $\fn{f}{\mathcal{D}}{\C}$ holomorfna in nekonstantna in $K \subseteq \mathcal{D}$ kompaktna. Tedaj lahko $f(z)$ doseže maksimum le na $\partial K$,
minimum pa na robu in v ničlah.
\section{Biholomorfne preslikave.}
\paragraph{Definicija.} Biholomorfne preslikave so preslikave, ki so hkrati bijektivne in holomorfne.
V nadaljnjem bomo obravnavali bijektivne preslikave kroga. Bolj specifično nas zanimajo vse možne preslikave $\fn{f}{D(0, 1)}{D(0, 1)}$.
\paragraph{Lema.} (Schwartzeva lema) Bodi $\fn{f}{D(0, 1)}{D(0, 1)}$ holomorfna in naj velja $f(0)=0$.
Tedaj:
\begin{itemize}
    \item $|f(z)| \leq |z|$ za vsak $z\in D(0, 1)$
    \item $|f'(0)| \leq 1$
    \item Če obstaja $z_0 \in D(0, 1)$, da je $|f(z_0)|=|z_0|$ ali $f'(0)=1$, potem je $f(z) = \alpha z$, kjer je $\alpha$ konstanta in $|\alpha|=1$
\end{itemize}
\paragraph{Dokaz.} Najprej dokažimo prvo točko: $$f(z) = a_0 + a_1z + a_2z^2 + ... ~a_0 = 0$$
$$g(z) = \frac{f(z)}{z} = a_1 + a_2z + ...$$
V $z=0$ ima $g$ singularnost, toda vidimo, da je odpravljiva. Poleg tega lahko definiramo $g(0) = a_1$, torej je $g$ holomorfna.
Uporabimo princip maksimuma:
$$|g(z)| \leq \max|g(\zeta)|,~|\zeta|=r$$
Pri tem bodi $r$ polmer zaprtega kroga $D(0, r) \subset D(0, 1)$.
Tedaj je $$\frac{|f(z)|}{|z|} \leq \frac{|f(z_0)|}{r} \leq \frac{1}{r}$$
Ker je $r \leq 1$, velja $|f(z)| \leq |z|$. \\
Druga točka: $f'(0) = |a_1| = g(0)$, kar pa je po prvi točki $\leq 1$. \\
Tretja točka: Recimo, da obstaja tak $z_0$, da je $|f(z_0)| = |z_0|$. Tedaj je $|g(z_0)|=1$, torej v $z_0$ funkcija $g$ zavzame maksimum. Ker se to ne sklada z izrekom o maksimumih, mora biti $g$.
$$\forall z \in \mathcal{D}:~g(z) = \alpha\Rightarrow f(z) = \alpha z$$
Če velja $|f'(0) = 1|$, to pomeni $|g(0)|=1$ in ima $g$ ponovno maksimum v $z$. \\[4mm]
Oglejmo si naaslednje preslikave: \\
Za neki $\alpha \in D(0, 1)$ definiramo 
$$f_\alpha(z) = \frac{z-\alpha}{1-\overline{\alpha}z}$$
Imamo pol $z=1/\overline{\alpha}$, $|z| = \frac{1}{|\alpha|} > 1$. Poleg tega je $f$ holomorfna na neki okolici $\overline{D}(0, 1)$
\paragraph{Trditev.} $f_\alpha(\partial D(0, 1)) \subseteq \partial D(0, 1)$ in $f_\alpha(D(0, 1)) \subseteq D(0, 1)$
\paragraph{Dokaz.} V resnici moramo dokazati $|z|=1 \Leftarrow |f(z)| = 1$
$$\left|f_\alpha(z)\right| = \left|\frac{z-\alpha}{1-\overline{\alpha}z}\right| = \left|\frac{z-\alpha}{1-\alpha\overline{z}}\right| = \left|\frac{z-\alpha}{1-\alpha/z}\right| = \left|\frac{z(z-\alpha)}{z-\alpha}\right| = |z| = 1$$
Zdaj dokažimo še $|z|<1\Leftarrow|f(z)|<1$. Recimo, da obstaja $z_0\in D(0, 1)$, da bo $|f_\alpha(z_0)| \geq 1$
Potem $|f_\alpha(z)|$ zavzame maksimum v $D(0, 1)$. Po principu maksimuma bi morala biti $f_\alpha(z)$ konstantna, kar pa očitno ni res.
\paragraph{Trditev.} $f_\alpha$ je bijektivna in velja $f_\alpha(\partial D(0, 1)) = \partial D(0, 1)$ ter $f_\alpha(D(0, 1)) = D(0, 1)$.
\paragraph{Dokaz.} Dovolj je pokazati, da ima $\alpha$ obojestranski inverz. Ugibamo, da je ta inverz $f_{-\alpha}$.
$$\left(f_{-\alpha} \circ f_\alpha\right)(z) = \frac{\frac{z-\alpha}{1-\overline{\alpha}z} + \alpha}{1 + \overline{\alpha}\frac{z-\alpha}{1-\overline{\alpha}z}} = \frac{z - \alpha\overline{\alpha}z}{1 - \alpha\overline{\alpha}} = z$$
Z enakim postopkom pokažemo, da je $\displaystyle{\left(f_\alpha \circ f_{-\alpha}\right)(z) = z}$.
$$\partial D(0, 1) = \mathrm{id}\left(\partial D(0, 1)\right) = f_\alpha\left(f_{-\alpha}\left(\partial D(0, 1)\right)\right)$$
$$\subseteq f_\alpha\left(\partial D(0, 1)\right) \subseteq \partial D(0, 1)$$
Sledi, da morajo biti povsod enačaji.
Enak postopek naredimo za odprti krog $D(0, 1)$.
\paragraph{Izrek.} Vsaka biholomorfna preslikava $\fn{f}{D(0, 1)}{D(0, 1)}$ je oblike
$$f(z) = \beta f_\alpha(z);~\beta \in \partial D(0, 1)$$
\paragraph{Ločimo dva primera:} $f(0) = 0$. Po Schwarzevi lemi mora veljati
$$|f(z)|\leq|z|~\forall z\in D(0, 1)$$
$$|f'(0)|\leq 1$$
$f$ ima inverz, kar pomeni:
$$\fn{f^{-1}}{D(0, 1)}{D(0, 1)}$$
$$f^{-1}(0) = 0$$
$$|z| = |f^{-1}\left(f(z)\right)| \leq |f(z)| \leq z$$
Sledi, da morajo biti povsod enačaji: $|f(z)| = |z|$ za vsak $z$. Po Schwarzevi lemi je torej $f(z) = \beta z = \beta f_\alpha(z)$, pri čemer vzamemo $\alpha = 0$.
Če je $f(0) = \gamma$, konstruiramo funkcijo $g := f_\gamma \circ f$.
Ker sta $f$ in $f_\gamma$ biholomorfni, je tudi $g$ biholomorfna.
$$g(0) = f_\gamma(f(0)) = f_\gamma (\gamma) = 0$$
Imamo torej biholomorfno funkcijo, ki 0 slika v 0.
$$g(z) = \beta z;~\beta = 1$$
$$f(z) = f_{-\gamma}(\beta z) = \frac{\beta z + \gamma}{1 + \overline{\gamma}\beta z} = \beta \frac{z + \gamma/\beta}{1 + \overline{\gamma}{\beta}z} = \beta\frac{z + \gamma\overline{\beta}}{1 + \overline{\gamma\overline{\beta}}z} = \beta f_{-\gamma\overline{\beta}}(z)$$
\section{Ulomljene linearne transformacije}
\paragraph{Definicija.} Ulomljena linearna transformacija je preslikava $\fn{f}{\C}{\C}$, dana s predpisom
$$f(z) = \frac{az+b}{cz+d},~a, b, c, d \in \C$$
Opomba: Če je $c \neq 0$, ima f(z) pol pri $z=-d/c$.
\end{document}