\documentclass[a4paper]{article}
\usepackage{amsmath, amssymb, amsfonts}
\usepackage[margin=1in]{geometry}
\usepackage{graphicx}
\usepackage{tikz}
\usepackage{esint}
\setlength{\parindent}{0em}
\setlength{\parskip}{1ex}
\newcommand{\vct}[1]{\overrightarrow{#1}}
\newcommand{\dif}{\,\mathrm{d}}
\newcommand{\pd}[2]{\frac{\partial {#1}}{\partial {#2}}}
\newcommand{\dd}[2]{\frac{\mathrm{d} {#1}}{\mathrm{d} {#2}}}
\newcommand{\C}{\mathbb{C}}
\newcommand{\R}{\mathbb{R}}
\newcommand{\Q}{\mathbb{Q}}
\newcommand{\Z}{\mathbb{Z}}
\newcommand{\N}{\mathbb{N}}
\newcommand{\fn}[3]{{#1}\colon {#2} \rightarrow {#3}}
\newcommand{\avg}[1]{\langle {#1} \rangle}
\newcommand{\Sum}[2][0]{\sum_{{#2} = {#1}}^{\infty}}
\newcommand{\Lim}[1]{\lim_{{#1} \rightarrow \infty}}
\newcommand{\Binom}[2]{\begin{pmatrix} {#1} \cr {#2} \end{pmatrix}}
\newcommand{\duline}[1]{\underline{\underline{#1}}}

\begin{document}
\paragraph{Uteženi prostori.} Definiramo skalarni produkt kot
$$\avg{u, v}_w = \int u(x)v(x)w(x)\dif x$$
Tedaj $w$ imenujemmo utež. Biti mora pozitivna. Ker smo spremenili definicijo skalarnega produkta, moramo hkrati ponovno definirati tudi normo (označimo $||\cdot||_w$) in prostor $L^2(\R)$ (označimo $L_w^2(\R)$).
\paragraph{Izrek.} (Sturm Liouviellov izrek) Dan je operator $$Ly = (Py')' + Ry,~~y\in C(a, b)$$
kjer bodi $P$ zvezno odvedljiva, $R$ pa zvezna. Bodi $w$  utež, ki bodi zvezna in pozitivna na $(a, b)$.
Rešujemo uteženi lastni problem $$Ly = -\lambda w y$$
Kjer $y$ zadošča ločenima robnima pogojema
$$\alpha_1 y(a) + \alpha_2 y'(a) = 0$$
$$\beta_1 y(b) + \beta_2 y'(b) = 0$$
\begin{enumerate}
    \item (Posplošene) lastne vrednosti so realne
    \item (Posplošeni) lastni funkciji, ki pripadata različnima (posplošenima) lastnima vrednostima, sta uteženo ortogonalna.
    \item Lastni podprostor, ki pripada neki (posplošeni) lastni vrednosti, je enorazsežen.
    \item Obstaja zaporedje posplošenih lastnih funkcij $(y_n)_{n \in \N}$ za lastne vrednosti $(\lambda_n)_{n\in\N}$, ki tvorijo ortonormirano bazo $L_w^2(a, b)$
    Torej za vsak $y \in L^2_w(a, b)$ velja $$y = \Sum[1]{n}\avg{y, y_n}_w y_n$$ (Imamo konvergenco po točkah)
    Pri tem velja še $\lim_{n\to\infty}\lambda_n = \infty$
\end{enumerate}
\paragraph{Dokaz.} $L$ je formalno sebi adjungiran:
$$\avg{Lu, v} = \avg{u, Lv}$$
\begin{enumerate}
    \item Bodi $\lambda$ lastna vrednost in $u$ pripadajoča lastna funkcija:
    $$Lu = -\lambda wu$$
    $$\avg{Lu, u} = -\lambda \avg{wu, u} = -\lambda \int_{a}^{b}wu\overline{u}\dif x = -\lambda ||u||^2_w$$
    Hkrati je $\avg{Lu, u} = \avg{u, Lu} = -\overline{\lambda} \avg{u, wu} = ... = -\overline{\lambda} ||u||^2_w$. Sledi $\lambda||u||_w^2 = \overline{\lambda}||u||_w^2$. Ker $u\neq0$, mora biti $\lambda \in \R$.
    \item Naj bosta $u$ in $v$ lastni funkciji z različnima lastnima vrednostima. $$\avg{Lu, v} = -\lambda_1\avg{wu, v} = -\lambda_1\avg{u,v}_w$$
    Hkrati je $\avg{Lu, v} = \avg{u, Lv} = ... = -\lambda_2\avg{u, v}_w$. Ker sta si $\lambda_1$ in $\lambda_2$ po predpostavki različni, mora biti $\avg{u, v}_w = 0$.
    \item Naj bosta $u$ in $v$ dve lastni funkciji za isto lastno vrednost $\lambda$. Dokazujemo, da sta linearno odvisni.
    Upoštevamo robna pogoja:
    $$\alpha_1u(a) + \alpha_1u'(a) = 0$$
    $$\beta_1 u(b) + \beta_2 u'(b) = 0$$
    \\
    $$\alpha_1v(a) + \alpha_1v'(a) = 0$$
    $$\beta_1 v(b) + \beta_2 v'(b) = 0$$
    Lahko si zamislimo, da gre za skalarni produkt dveh vektorjev:
    $$\begin{bmatrix}
        \alpha_1 \\ \alpha_2
    \end{bmatrix}\begin{bmatrix}
        u(a) \\ u'(a)
    \end{bmatrix} = 0$$
    $$\begin{bmatrix}
        \beta_1 \\ \beta_2
    \end{bmatrix}\begin{bmatrix}
        u(b) \\ u'(b)
    \end{bmatrix}$$$$\begin{bmatrix}
        \alpha_1 \\ \alpha_2
    \end{bmatrix}\begin{bmatrix}
        v(a) \\ v'(a)
    \end{bmatrix} = 0$$
    $$\begin{bmatrix}
        \beta_1 \\ \beta_2
    \end{bmatrix}\begin{bmatrix}
        v(b) \\ v'(b)
    \end{bmatrix}$$
    $(\alpha_1, \alpha_2)$ je neničeln vektor, ki je hkrati
    pravokoten na $(u(a), u'(a))$ in $(v(a), v'(a))$. To je v $\R^2$ mogoče le, če sta vektorja linearno odvisna.
    Sledi, da obstajata taka $C_1, C_2$, da je $$\begin{bmatrix}
        C_1\begin{bmatrix}
            u(a) \\ u'(a)
        \end{bmatrix}
        + C_2 \begin{bmatrix}
            v(a) \\ v'(a)
        \end{bmatrix} = 0
    \end{bmatrix}$$
    Če je $y = C_1 u + C_2 v$ in dokažemo $y = 0$, vemo, da sta $u, v$ linearno odvisni.
    $$Ly = -\lambda w y$$
    $$y(a) = C_1u(a) + C_2 v(a) = 0$$
    $$y'(a) = 0$$
    Imamo Cauchyjevo nalogo za $y'$. Vemo, da ima taka naloga natanko eno rešitev, in vidimo, da je $y=0$ rešitev. Torej sta $u, v$ linearno odvisni.
    \item Ne bomo dokazovali.
\end{enumerate}
\end{document}