\documentclass[a4paper]{article}
\usepackage{amsmath, amssymb, amsfonts}
\usepackage[margin=1in]{geometry}
\usepackage{graphicx}
\usepackage{tikz}
\usepackage{esint}
\setlength{\parindent}{0em}
\setlength{\parskip}{1ex}
\newcommand{\vct}[1]{\overrightarrow{#1}}
\newcommand{\dif}{\mathrm{d}}
\newcommand{\pd}[2]{\frac{\partial {#1}}{\partial {#2}}}
\newcommand{\dd}[2]{\frac{\mathrm{d} {#1}}{\mathrm{d} {#2}}}
\newcommand{\C}{\mathbb{C}}
\newcommand{\R}{\mathbb{R}}
\newcommand{\Q}{\mathbb{Q}}
\newcommand{\Z}{\mathbb{Z}}
\newcommand{\N}{\mathbb{N}}
\newcommand{\fn}[3]{{#1}\colon {#2} \rightarrow {#3}}
\newcommand{\avg}[1]{\langle {#1} \rangle}
\newcommand{\Sum}[2][0]{\sum_{{#2} = {#1}}^{\infty}}
\newcommand{\Lim}[1]{\lim_{{#1} \rightarrow \infty}}
\newcommand{\Binom}[2]{\begin{pmatrix} {#1} \cr {#2} \end{pmatrix}}

\begin{document}
Prejšnjič smo prišli do enačbe
$$(k+\mu)(k+\mu-1)c_k + \sum_{j=0}^{k}(j + \mu)c_jp_{k-j} + \sum_{j=0}^{k}c_jq_{k-j} = 0$$
Člene, ki vsebijejo $c_k$, zberemo skupaj.
$$c_k\left((\mu + k)(\mu + k - 1) + (\mu + k)p_0 + q_0\right) + \sum_{j=0}^{k-1}\left((\mu + j)c_jp_{k-j} + c_jq_{k-j}\right) = 0$$
Tako smo izrazili $c_k$ s pomočjo ostalih členov zaporedja. Potrebujemo še $\mu$. Pri $k=0$ je
$$c_0\left(\mu(\mu - 1) + \mu p_0 + q_0\right) = 0$$
Dobili smo kvadratno enačbo, ki ji pravimo tudi odločitvena zveza. Njeni rešitvi sta neki $\mu_1$ in $\mu_2$,
običajno vzamemo $\mathfrak{Re}(\mu_1) >= \mathfrak{Re}(\mu_2)$. \\
Označimo $f(\mu) = \mu(\mu - 1) + \mu p_0 + q_0 = (\mu - \mu_1)(\mu - \mu_2)$. Opazimo, da je člen pri $c_k$ ravno enak $f(\mu + k)$:
$$c_k \cdot f(\mu + k) = -\sum_{j=0}^{k-1} \left((\mu + j) p_{k-j} + q_{k-j}\right)c_j$$
$$c_k(\mu + k - \mu_1)(\mu + k - \mu_2) = -\sum_{j=0}^{k-1} \left((\mu + j) p_{k-j} + q_{k-j}\right)c_j$$
Zdaj poglejmo primer $\mu = \mu_1$:
$$c_k\cdot k \cdot \left(k + \mu_1 - \mu_2\right) = -\sum_{j}^{k-1}\left((\mu_1 + j)p_{k-j} + q_{k-j}\right)c_j$$
Ker je $\mathfrak{Re}(\mu_1) > \mathfrak{Re}(\mu_2)$, mora biti vsota $k + \mu_1 - \mu_2$ različna od 0 (ker je $k>0$, pravzaprav večja od 0).
Torej lahko pišemo
$$c_k = -\frac{1}{k + \mu_1 - \mu_2} \sum_{j}^{k-1}\left((\mu_1 + j)p_{k-j} + q_{k-j}\right)c_j$$
To je ena rešitev. Druga rešitev je, da vstavimo $\mu = \mu_2$.
$$c_k\cdot k \cdot \left(k + \mu_2 - \mu_1\right) = -\sum_{j}^{k-1}\left((\mu_2 + j)p_{k-j} + q_{k-j}\right)c_j$$
Zdaj nimamo več zagotovila, da je $k + \mu_2 - \mu_1 \neq 0$. Če je $\mu_1 - \mu_2 \in \N$, potem $\mu_1 - \mu_2$ -tega 
člena ne bomo mogli izračunati. Sicer pa naredimo isto kot prej - delimo in dobimo linearno neodvisno rešitev.
\paragraph{Primer.} $4z^2y'' + 2zy' + zy = 0$
$$y'' + \frac{1}{2z} y' + \frac{1}{4z} y = 0$$
$$zp(z) = \frac{1}{2}~~~p_0 = \frac{1}{2},~p_1 = p_2 = ... = 0$$
$$z^2q(z) = \frac{z}{4}~~~q_1 = \frac{1}{4},~p_0 = p_2 = ... = 0$$
Odločitvena zveza: $$\mu\left(\mu - 1 + p_0\right) + q_0 = 0$$
$$\mu \left(\mu - \frac{1}{2}\right) = 0$$
Dobili smo $\mu_1 = 1/2$ in $\mu_2 = 0$. Njuna razlika ni naravno število, torej ne bomo imeli težav. \\
Vstavimo $\mu = \frac{1}{2}$:
$$y = \Sum{k} c_kz^{k+\frac{1}{2}}$$
$$y' = \Sum{k} c_k\left(k + \frac{1}{2}\right)z^{k-\frac{1}{2}}$$
$$y'' = \Sum{k} c_k\left(k + \frac{1}{2}\right)\left(k - \frac{1}{2}\right)z^{k-\frac{3}{2}}$$
To vstavimo v začetno enačbo, poiščemo koeficient pri $z^{k + \frac{1}{2}}$.
$$c_k(2k+1)(2k-1) + c_k(2k+1) + c_{k-1} = 0$$
$$c_k = -\frac{c_{k-1}}{2k(2k+1)}$$
Ko izračunamo par členov (ali pa opazimo, da bomo v imenovalcu dobili $(2k-1)!$), vidimo, da ne potrebujemo rekurzivne zveze:
$$c_k = \frac{(-1)^kc_0}{(2k+1)!}$$
To je podobno vrsti za sinus, vendar ne čisto, saj $c_k$ stoji pred potenco $z^{k+1/2}$.
Zdaj vstavimo $\mu = 0$ in pogledamo koeficient pri $z^k$.
$$4c_kk(k-1)+ 2c_kk + d_{k-1} = 0$$
$$c_k = -\frac{c_{k-1}}{2k(2k-1)}$$
Spet opazimo, da je $\displaystyle{c_k = \frac{(-1)^kc_0}{(2k)!}}$. Tokrat imamo opravka s potencami $z^k = \displaystyle{(z^{1/2})^2k}$, torej je $$y_2 = c_0\cos\sqrt{z}$$
Zdaj obravnavajmo še problematični primer: $\mu_1 - \mu_2 \in \N$, recimo $\mu_1 - \mu_2 = m$.
$$0 = -\sum_{j=0}^{m-1}(...)c_j$$
Tedaj postavimo $c_0 = c_1 = ... = c_{m-1} = 0$ in izberemo poljuben $c_m$. Ostale koefiiente lahko računamo z rekurzivno zvezo.
S tem smo dobili rešitev, ki pa ni nujno linearno neodvisna od prve. \\
Če sta naši rešitvi linearno neodvisni, lahko (brez dokaza) drugo rešitev dobimo z nastavkom
$$y_2 = y_1\ln z + z^{\mu_2}\cdot f(z)$$
Pri čemer mora biti $f(z)$ holomorna v okolici $0$ - tedaj jo lahko razvijemo, vstavimo v originalno enačbo in računamo koeficiente.
\paragraph{Besselova diferencialna enačba.}
$$z^2y'' + zy' + (z^2 - \nu^2) y = 0$$
$$y'' + \frac{1}{z}y' + \left(1 - \frac{\nu^2}{z^2}\right)y = 0$$
$0$ je pravilna singularna točka te enačbe.
$$zp(z) = 1~~~p_0 = 1$$
$$z^2q(z) = z^2 - \nu^2~~~p_0 = -\nu^2,~p_2 = 1$$
$$\mu(\mu-1+p_0)+q_0=0$$
$$\mu_1 = \nu,~~\mu_2 = -\nu$$
Spet vzamemo $\mu = \mu_1 = \nu$, vstavimo v originalno enačbo in izrazimo koeficiente pri $z^{k+\nu}$.
$$c_k(k+\nu)(k+\nu-1)+c_k(k+\nu) + c_{k-2} - \nu^2c_k = 0$$
$$c_k\cdot k(k+2\nu) =-c_{k-2}$$
$$c_k = -\frac{c_{k-2}}{k(k+2\nu)}$$
Če poznamo $c_0$, lahko izračunamo $c_2, c_4, c_6 ...$ Za lihe $c_k$ lahko izberemo $c_1 = 0$ in dobimo \\ $c_1 = c_3 = c_5 = ... = 0$.
Izkaže se, da to celo moramo narediti, sicer bi (nekje) dobili protislovje. \\
Tudi za $c_0$ lahko izberemo $0$ in dobimo validno rešitev, ampak to je dolgočasno. Raje izberimo
$$c_0 = \frac{1}{2^\nu\Gamma(\nu + 1)}$$
Zdaj lahko izračunamo še ostale sode člene:
$$c_2 = -\frac{1}{4(1 + \nu)2^\nu\Gamma(\nu+1)} = -\frac{1}{2^{\nu + 2}\Gamma(\nu + 2)}$$
$$c_4 = \frac{1}{2^{\nu + 4} \Gamma(\nu + 3) \, (1 \cdot 2)}$$
$$c_6 = -\frac{1}{2^{\nu + 6} \Gamma(\nu + 4) \, (1 \cdot 2 \cdot 3)}$$
Splošen člen je: $\displaystyle{c_{2n} = \frac{(-1)^n}{2^{\nu + 2n}\Gamma(\nu + n + 1)\,n!}}$
\end{document}