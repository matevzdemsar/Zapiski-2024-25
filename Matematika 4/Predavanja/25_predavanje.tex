\documentclass[a4paper]{article}
\usepackage{amsmath, amssymb, amsfonts}
\usepackage[margin=1in]{geometry}
\usepackage{graphicx}
\usepackage{tikz}
\usepackage{esint}
\setlength{\parindent}{0em}
\setlength{\parskip}{1ex}
\newcommand{\vct}[1]{\overrightarrow{#1}}
\newcommand{\dif}{\mathrm{d}}
\newcommand{\pd}[2]{\frac{\partial {#1}}{\partial {#2}}}
\newcommand{\dd}[2]{\frac{\mathrm{d} {#1}}{\mathrm{d} {#2}}}
\newcommand{\C}{\mathbb{C}}
\newcommand{\R}{\mathbb{R}}
\newcommand{\Q}{\mathbb{Q}}
\newcommand{\Z}{\mathbb{Z}}
\newcommand{\N}{\mathbb{N}}
\newcommand{\fn}[3]{{#1}\colon {#2} \rightarrow {#3}}
\newcommand{\avg}[1]{\langle {#1} \rangle}
\newcommand{\Sum}[2][0]{\sum_{{#2} = {#1}}^{\infty}}
\newcommand{\Lim}[1]{\lim_{{#1} \rightarrow \infty}}
\newcommand{\Binom}[2]{\begin{pmatrix} {#1} \cr {#2} \end{pmatrix}}
\newcommand{\duline}[1]{\underline{\underline{#1}}}

\begin{document}
Prejšnjič:
$$P_n(z) = \sum_{k=0}^{\lfloor\frac{n}{2}\rfloor}(-1)^k\frac{(2n-2k)!}{2^n k!(n-k)!(n-2k)!}z^{n-2k}$$
Rodriguesova formula: $\displaystyle{P_n(z) = \frac{1}{2^n n!}\dd{^n}{z^n}\left((z^2 - 1)^n\right)}$
\paragraph{Dokaz.} Oglejmo si $\displaystyle{\dd{^n}{z^n}\left(z^{2n - 2k}\right)}$
$$= (2n - 2k)(2n - 2k - 1)...(n-2k + 1)z^{n-2k} = \frac{(2n - 2k)!}{(n-2k)!}z^{n-2k}$$
Vidimo, da lahko to vstavimo v formulo za $P_n(z)$:
$$P_n(z) = \frac{1}{2^n}\sum_{k=0}^{\lfloor\frac{n}{2}\rfloor}\frac{(-1)^k}{k! (n-k)!}\dd{^n}{z^n}\left(z^{2n - 2k}\right)$$
$$= \frac{1}{2^n}\dd{^n}{z^n}\sum_{k=0}^{\lfloor\frac{n}{2}\rfloor} \frac{1}{n!}\,\frac{n!}{k!(n-k)!}(z^2)^{n-k}$$
$1/n!$ nesemo iz oklepaja. Na preostanku vsote uporabimo binomski izrek, pri čemer lahko upoštevamo, da je za $k > n/2$ $n$-ti odvod člena $z^{n-k}$ enak 0.
$$P_n = \frac{1}{2^n n!} \dd{^n}{z^2} \sum_{k=0}^{n} \Binom{n}{k} (-1)^k (z^2)^{n-k} = \frac{1}{2^n n!}\dd{^n}{z^n}\left((z^2 - 1)^n\right)$$
$\backslash\mathrm{end\{proof\}}$ \\[3mm]
Zdaj lahko računamo Legendrove polinome, npr. $P_0(z) = 1$, $P_1(z) = z$, $P_2(z) = \displaystyle{\frac{3z^2 - 1}{2}}$, \\ $P_3(z) = ... = \displaystyle{\frac{5z^3 - 3z}{2}}$
\paragraph{Rodovna funkcija} Podobno kot za Besselove funkcije imamo tudi za Legendrove polinome funkcijo, katere koeficienti so ravno Legendrovi polinomi.
\paragraph{Trditev.} Okolli točke 0 velja razvoj: $$\frac{1}{\sqrt{1 - 2zt + t^2}} = \Sum{n}P_n(z)t^n$$
\paragraph{Ideja dokaza.} Uporabimo posplošeno binomsko vrsto $$(1 + t)^\alpha = \Sum{n}\Binom{\alpha}{n}t^n$$
$$(1 - 2zt + t^2)^{-1/2} = \left((t-z)^2 + 1 - z^2\right)^{-1/2} = \frac{1}{\sqrt{1-z^2}}\left(1 + \frac{(t-z)^2}{1-z^2}\right)^{-1/2}$$
To razvijemo v vrsto in pogledamo člene z istim $z^n$. S tovrstnim računanjem se zaradi časovne zahtevnosti postopka ne bomo ukvarjali. \\
\paragraph{Posledica.} $$(n+1)P_{n+1}(z) = \frac{2n + 1}{2}zP_n(z) - nP_{n-1}(z)$$
\paragraph{Dokaz.} $\displaystyle{\frac{1}{\sqrt{1 - 2tz + t^2}}}$ odvajamo po $t$.
\begin{equation}
    \frac{1}{\sqrt{1-2zt + t^2}} = \Sum{n}P_n(z)t^n
\end{equation}
\begin{equation}
    \frac{z-t}{(1-2zt + t^2)^{3/2}} = \Sum{n} nP_n(z)t^{n-1}
\end{equation}
Enačbo (1) pomnožimo z $z-t$, enačbo (2) pa z $1 - 2zt + t^2$. Nato ju odštejemo in dobimo
$$\Sum{n} nP_n(z)t^{n-1}(1 - 2zt + t^2) = \Sum{n}P_n(z)t^n(z-t)$$
Primerjamo koeficiente pri $t^n$ in dobimo iskano rekurzivno zvezo.
\paragraph{Posledica.} $P_n(1) = 1$, $P_n(-1) = (-1)^n$
\paragraph{Dokaz.} Imamo rekurzivno zvezo, torej lahko uporabimo indukcijo. $P_0$ in $P_1$ lahko izračunamo direktno.
\paragraph{Izrek.} $$\int_{-1}^{1} P_n(z)P_m(z)' \, \dif z = \delta_{mn} \frac{2}{2n+1}$$
\paragraph{Ideja dokaza.} Polinome zapišemo z Rodriguezovo formulo, vstavimo v integral in uporabimo per partes. \\[3mm]
S tem smo definirali skalarni produkt na $C([-1, 1])$ kot $$\avg{f, g} = \int_{-1}^{1} f(x)\overline{g(x)}\,\dif x$$
S skalarnim produktom je definirana tudi norma v $C([-1, 1])$:
$$||f||_2 = \sqrt{\avg{f, f}} = \int_{-1}^{1}|f(x)|^2 \, \dif x$$
In razdalja:
$$d(f, g) = ||f-g||_2$$
Napolnitev metričnega prostora $C[-1, 1]$ označimo z $L^2[-1, 1]$. \\[3mm]
Weierstrassov izrek: Za neko $f \in C[-1, 1]$ obstajajo polinomi $q_n$, da velja
$$\sup_{x \in [-1, 1]} |f(x) - q_n(x)| \to 0$$
Sledi $||f - q_n||_2 \to 0$, kajti $\displaystyle{\int_{-1}^{1} |f(x) - q_n(x)|^2 \,\dif x} \leq \sup_{x\in[-1, 1]} |f(x) - q_n(x)|^2\int_{-1}^{1}\dif x$
Ker so funkcije v $L^2$ limite takih zaporedij, mora to veljati tudi za $L^2$. Drugače povedano, Za vsako funkcijo $f \in L^2[-1, 1]$ lahko najdemo zaporedje polinomov, ki konvergira proti njej.
To hkrati pomeni, da so polinomi gosti v $L^2$ z normo $||\cdot||_2$
$$f(x) = \Sum{n} \alpha_n q_n(x)$$
Vsak $q_n(x)$ pa lahko zapišemo kot linearno kombinacijo Legendrovih polinomov. Se pravi lahko vsako funkcijo v $L^2$ zapišemo kot linearno kombinacijo Legendrovih polinomov.
$$f(x) = \Sum{n}\beta_nP_n(x)$$
Ker so $P_n$ med sabo pravokotni, je
$$\beta_n = \frac{\avg{f, P_n}}{||P_n||_2^2} = (2n + 1)\avg{f, P_n} = \frac{2n+1}{2}\int_{-1}^{1}f(x)P_n(x)\,\dif x$$
\paragraph{Pridruženi Legendrovi polinomi.} Za $m = 0, 1, ..., n$ definiramo
$$P_n^m(z) = (-1)^m(1-z^2)\frac{m}{2}\dd{^m}{z^m}P_n(z)$$
\paragraph{Trditev.} Pridruženi Legendrovi polinomi rešijo enačbo
$$((1-z^2)y')' + \left[n(n+1) - \frac{m^2}{1-z^2}\right]y = 0$$
\paragraph{Dokaz.} Vstavimo v enačbo.
\paragraph{Pomen.} Definiramo preslikavo $\fn{L}{C^2[-1, 1]}{C[-1, 1]}$, ki slika $y$ v $\displaystyle{\left((1-z^2)y'\right)' - \frac{m^2}{1-z^2}y}$ in se izkaže za linearno.
\paragraph{Trditev.} $P_n^m$ so lastni vektorji tako definirane preslikave $L$. Če je $L$ hermitska, so lastne vrednosti realne in iz lastnih vektorjev lahko napravimo
ortogonalno bazo.
\paragraph{Izrek.} $m \geq 0$ fiksiramo. Potem so $\left(P_n^m\right)_{n=0}^\infty$ ortogonalna baza $L^2[-1, 1]$
$$\int_{-1}^{1} P_n^m(x)P_k^m(x)\,\dif x = \delta_{nk}\frac{2}{2n+1}\frac{(n+m)!}{(n-m)!}$$
\paragraph{Hermitovi polinomi.} Imamo še druge ortogonalne polinome, najbolj znani med njimi so Hermitovi polinomi. Gre za rešitve diferencialne enačbe
$$y'' - 2zy' + 2\nu y = 0$$
Takšna diferencialna enačba sploh nima singularnosti, kot v primeru Legendrove enačbe pa recimo, da je $\nu \in \N$. Da dobimo Hermitov polinom $H_n(z)$, pri $z^n$ izberemo koeficient $2^n$. Ima sledeče lastnosti:
$$H_0(z) = 1,~~H_1(z) = 2z$$
Rodriguesova formula: $\displaystyle{H_n(z) = (-1)^n e^{z^2}\dd{^n}{z^n}\,e^{-z^2}}$ \\
Rodovna funkcija: $\displaystyle{e^{2zt - t^2} = \Sum{n} \frac{H_n(z)}{n!}t^n}$
Rekurzivne formule:
\begin{itemize}
    \item $H_n(z) = 2zH_{n-1}(z) - H'_{n-1}(z)$
    \item $H'_n(z) = 2nH_{n-1}(z)$
    \item $H_n(z) = 2zH_{n-1}(z) - 2(n-1)H_{n-2}(z)$
\end{itemize}
\end{document}