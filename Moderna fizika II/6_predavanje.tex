\documentclass[a4paper]{article}
\usepackage{amsmath, amssymb, amsfonts}
\usepackage[margin=1in]{geometry}
\usepackage{graphicx}
\usepackage{tikz}
\usepackage{esint}
\setlength{\parindent}{0em}
\setlength{\parskip}{1ex}
\newcommand{\vct}[1]{\overrightarrow{#1}}
\newcommand{\pd}[2]{\frac{\partial {#1}}{\partial {#2}}}
\newcommand{\dd}[2]{\frac{\mathrm{d} {#1}}{\mathrm{d} {#2}}}
\newcommand{\C}{\mathbb{C}}
\newcommand{\R}{\mathbb{R}}
\newcommand{\Q}{\mathbb{Q}}
\newcommand{\Z}{\mathbb{Z}}
\newcommand{\N}{\mathbb{N}}
\newcommand{\fn}[3]{{#1}\colon {#2} \rightarrow {#3}}
\newcommand{\avg}[1]{\langle {#1} \rangle}
\newcommand{\Sum}[2][0]{\sum_{{#2} = {#1}}^{\infty}}
\newcommand{\Lim}[1]{\lim_{{#1} \rightarrow \infty}}
\newcommand{\Binom}[2]{\begin{pmatrix} {#1} \cr {#2} \end{pmatrix}}


\begin{document}
\paragraph{Zgradba jeder.} Od prejšnjič poznamo semi empirično masno formulo, ki pa ni nujno popolnoma natančna.
Skiciramo lahko izmerjeno in izračunano (po SEMF) vezavno energijo za različne atome:
\begin{figure}[h!]
    \centering
    \includegraphics{SEMF.jpg}
\end{figure}
Opazimo, da so največja odstopanja pri vrednostih 2, 8, 20, 28, 50, 82 in 126. Tem številom pravimo magična števila - gre za jedra, ki so posebej dobro vezana.
Jedrom, ki imajo magično število protonov ali nevtronov, pravimo magična jedra - primer takih jeder so jedra žlahtnih plinov. Atomom, ki imajo magično število protonov in magično število elektronov, pravimo
dvojno magična jedra (na primer $^{208}Pb$). Jedra, katerih vrednost $Z$ ali $N$ ravno za 1 večje od magičnega števila (tj. 3, 9, 21 itd.) pa so veliko manj stabilna - primer so alkalijske kovine,
ki so tako ali tako reaktivne, saj imajo na zunanji lupini samo en elektron. \\[2mm]
Vzrok za takšno obnašanje iščemo v vezavi nukleona v jedru. Na nukleon mora delovati nekakšen potencial, in ker je sila med nukleoni kratkega dosega, je ta potencial zelo podoben porazdelitvi mase v jedru.
\begin{figure}[h!]
    \centering
    \includegraphics{potencial.png}
\end{figure}
(Če je razdalja med nukleonoma manjša od vsote njunih polmerov, je potencial neskončen, saj ne moreta eden skozi drugega.)
Nukleoni bodo v potencialu zasedli najnižje dostopno stanje. Ker so nukleoni fermioni, velja izključitveno načelo in sme biti v določenem stanju le en sam
nukleon. Ker imajo tudi nukleoni spin, se bosta na vsaki stopnji potenciala pojavila največ dva protona in dva nevtrona.
$$V_{ef} = - \frac{V_0}{e^{(M_p - M_j)/2} + 1}$$
Rešujemo Schrödingerjevo enačbo:
$$-\frac{\hbar^2}{2m}\nabla^2R(\vct{r}) + V_{ef}R(\vct{r}) = WR(\vct{r})$$
Enačba ni analitično rešljiva, lahko pa naredimo še kar dober približek in potencial aproksimiramo s parabolo:
$$V_{ef} \approx \frac{1}{2}mw^2r^2 = \frac{1}{2}mw^2(x^2 + y^2 + z^2)$$
Nato uporabimo nastavek $\displaystyle{R(x, y, z) = X(x)Y(y)Z(z)}$ - kajti $x$, $y$ in $z$ so neodvisne.
S tem dobimo vsoto treh enačb:
$$\left(-\frac{\hbar^2}{2m}X''YZ + \frac{1}{2}mw^2x^2XYZ\right) + \left(-\frac{\hbar^2}{2m}XY''Z + \frac{1}{2}mw^2y^2XYZ\right) + \left(-\frac{\hbar^2}{2m}XYZ'' + \frac{1}{2}mw^2z^2XYZ\right) = WXYZ$$
Na levi in desni strani delimo z $XYZ$:
$$\left(-\frac{\hbar^2}{2m}\frac{X''}{X} + \frac{1}{2}mw^2x^2\right) + \left(-\frac{\hbar^2}{2m}\frac{Y''}{Y} + \frac{1}{2}mw^2y^2\right) + \left(-\frac{\hbar^2}{2m}\frac{Z''}{Z} + \frac{1}{2}mw^2z^2\right) = W$$
Dobili smo tri enačbe (eno za $X$, eno za $Y$ in eno za $Z$), katerih vsote morajo biti konstantne. Ker so spremenljivke $x$, $y$ in $z$ neodvisne, morajo biti tudi izrazi v oklepajih konstantni.
Recimo $W_x + W_y + W_z = W$. Tedaj dobimo diferencialne enačbe za $X$, $Y$ in $Z$. Gre pa ravno za enačbe linearnih harmonskih oscilatorjev, torej je
$\displaystyle{W_x = \hbar w(\frac{1}{2} + n_x)}$, $\displaystyle{W_y = \hbar w(\frac{1}{2} + n_y)}$, $\displaystyle{W_z = \hbar w(\frac{1}{2} + n_z)}$.
Tedaj je $\displaystyle{W = \hbar w \left(\frac{3}{2} + n_x + n_y + n_z\right)}$
\end{document}