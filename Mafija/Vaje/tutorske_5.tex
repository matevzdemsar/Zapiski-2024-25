\documentclass[a4paper]{article}
\usepackage{amsmath, amssymb, amsfonts}
\usepackage[margin=1in]{geometry}
\usepackage{graphicx}
\usepackage{tikz}
\usepackage{esint}
\setlength{\parindent}{0em}
\setlength{\parskip}{1ex}
\newcommand{\vct}[1]{\overrightarrow{#1}}
\newcommand{\dif}{\,\mathrm{d}}
\newcommand{\pd}[2]{\frac{\partial {#1}}{\partial {#2}}}
\newcommand{\dd}[2]{\frac{\mathrm{d} {#1}}{\mathrm{d} {#2}}}
\newcommand{\C}{\mathbb{C}}
\newcommand{\R}{\mathbb{R}}
\newcommand{\Q}{\mathbb{Q}}
\newcommand{\Z}{\mathbb{Z}}
\newcommand{\N}{\mathbb{N}}
\newcommand{\fn}[3]{{#1}\colon {#2} \rightarrow {#3}}
\newcommand{\avg}[1]{\langle {#1} \rangle}
\newcommand{\Sum}[2][0]{\sum_{{#2} = {#1}}^{\infty}}
\newcommand{\Lim}[1]{\lim_{{#1} \rightarrow \infty}}
\newcommand{\Binom}[2]{\begin{pmatrix} {#1} \cr {#2} \end{pmatrix}}
\newcommand{\duline}[1]{\underline{\underline{#1}}}

\begin{document}
\paragraph{Diferencialne enačbe.} Imamo tri plasti, ki drsijo ena mimo druge. Med njimi deluje viskoznost s koeficientom $\gamma$.
$$m\dot{v}_1 = \gamma(v_2 - v_1)$$
$$m\dot{v}_2 = \gamma(v_3 - v_2) + \gamma (v_1 - v_2) = \gamma (v_3 - 2v_2 + v_1)$$
$$m\dot{v}_3 = \gamma(v_2 - v_3)$$
Sistem zapišemo v matrični obliki:
$$\dot{\vct{v}} = \tilde{A}\vct{v}$$
$$\tilde{A} = \frac{\gamma}{m} \begin{bmatrix}
    -1 & 1 & 0 \\
    1 & -2 & 1 \\
    0 & 1 & -1
\end{bmatrix} = \frac{\gamma}{m}A$$
Diagonalizirajmo $A$:
$$\det(A - \lambda I) = (-1 - \lambda)\left((- 2 - \lambda)(1 - \lambda) - 1\right) - (-1 - \lambda) = ... = -\lambda(\lambda + 1)(\lambda + 2)$$
Iammo lastne vrednosti $\lambda_1 = 0$, $\lambda_2 = -1$, $\lambda_3 = -3$. Izračunamo še pripadajoče lastne vektorja:
$$\lambda_1 = 0:~~\vct{u_1}(t) = u_0 \begin{bmatrix}
    1 \\ 1 \\ 1
\end{bmatrix}$$
$$\lambda_2 = -1:~~\vct{u_2}(t) = u_0 e^{-\gamma t/m}\begin{bmatrix}
    1 \\ 0 \\ -1
\end{bmatrix}$$
$$\lambda_3 = -3:~~\vct{u_3}(t) = u_0 e^{-3 \gamma t/m}\begin{bmatrix}
    1 \\ -2 \\ 1
\end{bmatrix}$$
Sledi: $\vct{v}(t) = a \vct{u_1}(t) + b\vct{u_2}(t) + c{u_3}(t)$
\paragraph{Kolokvij 2019.} Imamo kroglo z gostoto toplotnih izvorov $q$. Velja $q = kT^2$. Zanima nas toplotni tok, ki izhaja iz krogle.
$$\dd{W}{t} = qV - jS$$
$$mc\dd{T}{t} = qV - \lambda \frac{D}{s} T$$
$$\dot{T} = \frac{k}{\rho c} T^2 - \frac{\lambda}{k\rho c}\frac{3}{R} T$$
Imamo enačbo oblike $$\dot T = A(T^2 - BT)$$
$$A\dot T = A^2\left((T-\frac{B}{2})^2 - \frac{B^2}{4}\right) = \left(A\left(T - \frac{B}{2}\right)\right)^2 - \frac{A^2B^2}{4}$$
Vzamemo novo spremenljivko $u$:
$$\dot u  = u^2 - \left(\frac{AB}{2}\right)^2 = u^2 - u_0^2$$
$$\dif t = \frac{\dif u}{u^2 - u_0^2}$$
$$t = \begin{cases}
    -\frac{1}{u_0}\mathrm{artanh}\frac{u}{u_0} + t_0 & |u|<u_0 \\
    -\frac{1}{u_0}\mathrm{arcoth}\frac{u}{u_0} + t_0 & |u|>u_0 \\
\end{cases}$$
Poiščemo lahko inverz:
$$u_1 = -u_0\tanh\left[(t-t_0)u_0\right]$$
$$u_1 = -u_0\coth\left[(t-t_0)u_0\right]$$
\paragraph{Majhna nihanja.} Pri majhnih nihanjih imamo posplošene koordinate $\underline{q}$, v katerih mora veljati:
$$\pd{V}{q_i} = 0$$
Konstruiramo matriki $\duline{V}$ in $\duline{T}$, kjer je
$$V_{jk} = \partial_j\partial_k V$$
$$T_{jk} = \partial_j\partial_k T$$
Nato poiščemo lastne frekvence $\omega$ in pripadajoče lastne vektorje, da velja $$\det(\duline{V} - \omega^2\duline{T}) = 0$$
Tedaj je $$\underline{q}(t) = \mathrm{Span}\{\underline{q}_{last.} e^{i\omega t}\}$$
\end{document}