\documentclass[a4paper]{article}
\usepackage{amsmath, amssymb, amsfonts}
\usepackage[margin=1in]{geometry}
\usepackage{graphicx}
\usepackage{tikz}
\usepackage{esint}
\setlength{\parindent}{0em}
\setlength{\parskip}{1ex}
\newcommand{\vct}[1]{\overrightarrow{#1}}
\newcommand{\dif}{\mathrm{d}}
\newcommand{\pd}[2]{\frac{\partial {#1}}{\partial {#2}}}
\newcommand{\dd}[2]{\frac{\mathrm{d} {#1}}{\mathrm{d} {#2}}}
\newcommand{\C}{\mathbb{C}}
\newcommand{\R}{\mathbb{R}}
\newcommand{\Q}{\mathbb{Q}}
\newcommand{\Z}{\mathbb{Z}}
\newcommand{\N}{\mathbb{N}}
\newcommand{\fn}[3]{{#1}\colon {#2} \rightarrow {#3}}
\newcommand{\avg}[1]{\langle {#1} \rangle}
\newcommand{\Sum}[2][0]{\sum_{{#2} = {#1}}^{\infty}}
\newcommand{\Lim}[1]{\lim_{{#1} \rightarrow \infty}}
\newcommand{\Binom}[2]{\begin{pmatrix} {#1} \cr {#2} \end{pmatrix}}
\newcommand{\duline}[1]{\underline{\underline{#1}}}
\begin{document}
\paragraph{1. naloga} Imamo safirno ploščico, njena toplotna prevodnost ima lastni vrednosti $\lambda_\parallel$ in dvojno lastno vrednost $\lambda_\perp$.
$$\lambda = \begin{bmatrix}
    \lambda_\parallel & 0 & 0 \\
    0 & \lambda_\perp & 0 \\
    0 & 0 & \lambda_\perp
\end{bmatrix}$$
Ploščica ima debelino $a = 5\,mm$, širino $100\,mm$, velja $\lambda_\parallel = 36\,W/mK$ in $\lambda_\perp = 32\,W/mK$. Velja $\Delta T_x = 10\,K$. Os, ki pripada lastni vrednosti $\lambda_\parallel$, je za kot $45^\circ$ nagnjena glede na $\hat{e}_x$. Zanima nas toplotni tok skozi ploščico.
\paragraph{Reševanje.} Po definiciji je $$\vct{j} = -\duline{\lambda} \nabla T$$
$\duline{\lambda}$ moramo zapisati v laboratorijski (xyz) bazi. Takšno rotacijo opisuje
$$A' = RAR^T$$
$$R = \begin{bmatrix}
    \cos\varphi & \sin\varphi & 0 \\
    -\sin\varphi & \cos\varphi & 0 \\
    0 & 0 & 1 
\end{bmatrix}$$
Zavrteti moramo za $-45^\circ$.
$$\duline{\lambda}' = R\duline{\lambda}R^T = \begin{bmatrix}
    \cos\varphi & \sin\varphi & 0 \\
    -\sin\varphi & \cos\varphi & 0 \\
    0 & 0 & 1
\end{bmatrix}\begin{bmatrix}
    \lambda_\parallel && \\
    & \lambda_\perp & \\
    && \lambda_\perp \\
\end{bmatrix}\begin{bmatrix}
    \cos\varphi & -\sin\varphi & 0 \\
    \sin\varphi & \cos\varphi & 0 \\
    0 & 0 & 1 
\end{bmatrix}$$
$$=\begin{bmatrix}
    \cos\varphi & \sin\varphi & 0 \\
    -\sin\varphi & \cos\varphi & 0 \\
    0 & 0 & 1 
\end{bmatrix}\begin{bmatrix}
    \lambda_\parallel\cos\varphi & \lambda_\parallel\sin\varphi & 0 \\
    -\lambda_\perp\sin\varphi & \lambda_\perp\cos\varphi & 0 \\
    0 & 0 & 1
\end{bmatrix}$$
$$ = \begin{bmatrix}
    \lambda_\parallel\cos^2\varphi + \lambda_\perp\sin^2\varphi & \lambda_\parallel\sin\varphi\cos\varphi - \lambda_\perp\sin\varphi\cos\varphi & 0 \\
    \lambda_\parallel\sin\varphi\cos\varphi - \lambda_\perp\sin\varphi\cos\varphi & \lambda_\parallel\sin^2\varphi + \lambda_\perp\cos^2\varphi & 0 \\
    0 & 0 & 1
\end{bmatrix}$$
Zdaj izračunamo $\vct{j}$. Pričakujemo, da ta teče v smeri $y$, saj imamo tam temperaturno razliko:
$$\vct{j} = j\hat{e}_y,~~~
\nabla T = \begin{bmatrix}
    \partial_xT \\
    \partial_yT \\
    0
\end{bmatrix},~~~\partial_yT = \frac{\Delta T}{d}$$
$$\begin{bmatrix}
    0 \\ j \\ 0
\end{bmatrix} = \begin{bmatrix}
    \lambda_\parallel\cos^2\varphi + \lambda_\perp\sin^2\varphi & \lambda_\parallel\sin\varphi\cos\varphi - \lambda_\perp\sin\varphi\cos\varphi & 0 \\
    \lambda_\parallel\sin\varphi\cos\varphi - \lambda_\perp\sin\varphi\cos\varphi & \lambda_\parallel\sin^2\varphi + \lambda_\perp\cos^2\varphi & 0 \\
    0 & 0 & 1
\end{bmatrix}\begin{bmatrix}
    \partial_xT \\
    \partial_yT \\
    0
\end{bmatrix}$$
$$j_y = -\left(\lambda_\parallel \cdot \frac{1}{2} - \lambda_\perp\cdot\frac{1}{2}\right)\partial_xT - \left(\lambda_\parallel\cdot\frac{1}{2} + \lambda_\perp\cdot\frac{1}{2}\right)\partial_yT$$
$$j_x = 0 = -\left(\lambda_\parallel \cdot \frac{1}{2} + \lambda_\perp\cdot\frac{1}{2}\right)\partial_xT - \left(\lambda_\parallel\cdot\frac{1}{2} - \lambda_\perp\cdot\frac{1}{2}\right)\partial_yT$$
Izračunamo lahko $\displaystyle{\partial_xT = \frac{\frac{1}{2}(\lambda_\parallel - \lambda_\perp)}{\frac{1}{2}(\lambda_\parallel + \lambda_\perp)}\partial_yT = ... = \frac{1}{17}\partial_yT}$ (iz začetnih podatkov).
Nazadnje vstavimo številke, da izračunamo $j_y$.
\paragraph{Naloga.} Imamo kondenzator, v katerem velja:
$$\vct{j} = \duline{\sigma}\vct{E}$$
Tu $\duline{\sigma}$ označuje tenzor električne prevodnosti. Njegov inverz je tenzor električne upornosti $\duline{\zeta}$. Označimo
$$\vct{E} = \duline{\zeta}\vct{j}$$
Kot prej ima v lastnem sistemu $\duline{\sigma}$ obliko
$$\duline{\sigma} = \begin{bmatrix}
    \sigma_\parallel && \\
    & \sigma_\perp & \\
    & & \sigma_\perp \\ 
\end{bmatrix}$$ Kot prej je lastni vektor $\sigma$ nagnjen za nek kot glede na os $x$, vzdolž katere deluje polje.
\end{document}