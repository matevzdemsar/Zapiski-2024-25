\documentclass[a4paper]{article}
\usepackage{amsmath, amssymb, amsfonts}
\usepackage[margin=1in]{geometry}
\usepackage{graphicx}
\usepackage{tikz}
\usepackage{esint}
\setlength{\parindent}{0em}
\setlength{\parskip}{1ex}
\newcommand{\vct}[1]{\overrightarrow{#1}}
\newcommand{\dif}{\mathrm{d}}
\newcommand{\pd}[2]{\frac{\partial {#1}}{\partial {#2}}}
\newcommand{\dd}[2]{\frac{\mathrm{d} {#1}}{\mathrm{d} {#2}}}
\newcommand{\C}{\mathbb{C}}
\newcommand{\R}{\mathbb{R}}
\newcommand{\Q}{\mathbb{Q}}
\newcommand{\Z}{\mathbb{Z}}
\newcommand{\N}{\mathbb{N}}
\newcommand{\fn}[3]{{#1}\colon {#2} \rightarrow {#3}}
\newcommand{\avg}[1]{\langle {#1} \rangle}
\newcommand{\Sum}[2][0]{\sum_{{#2} = {#1}}^{\infty}}
\newcommand{\Lim}[1]{\lim_{{#1} \rightarrow \infty}}
\newcommand{\Binom}[2]{\begin{pmatrix} {#1} \cr {#2} \end{pmatrix}}

\begin{document}
\paragraph{Potencial kvadrupola.} Potencial telesa zapišemo kot
$$U(\vct{r}) = \int G(\vct{r} - \vct{r}')\rho(\vct{r}')\dif\vct{r}'$$
Tu je $\displaystyle{G(\vct{r}) = \frac{1}{4\pi r}}$, $\rho$ pa je odvisen od narave potenciala:
\begin{itemize}
    \item Potencial okoli monopola: $\rho(\vct{r}) = e\delta(\vct{r})$
    \item Potencial okoli dipola: $\rho(\vct{r}) = -(\vct{p}\nabla)\delta(\vct{r})$
\end{itemize}
Za dipolni moment naredimo sledeč izračun:
$$U(\vct{r}) = p_i \int G(\vct{r} - \vct{r}')\nabla_i\delta(\vct{r}') \dif\vct{r}'$$
Uporabimo Greenovo formulo:
$$= p_i\int \delta(\vct{r}')\nabla_iG(\vct{r} - \vct{r}')\dif\vct{r}' = p_i\nabla_iG(\vct{r})$$
$$= \frac{1}{4\pi} p_i \frac{r_i}{r^3} = \frac{\vct{p}\cdot\vct{r}}{4\pi r^3}$$
Kako pa izgleda kvadrupolni moment?
$$\rho(\vct{r}) = Q_{ij} \nabla_i\nabla_j\delta(\vct{r})$$
(gre za nekakšno matriko)
$$U(\vct{r}) = \frac{Q_{ij}}{4\pi} \frac{r^2\delta_{ij} - 3r_ir_j}{r^3} = \frac{r^2 \text{tr}Q - 3\vct{r}\cdot(Q\vct{r})}{4\pi r^3}$$
Hkrati je $$U(\vct{r}) = Q\frac{\delta_{ij}}{4\pi} \left(\frac{r^2\delta_{ij} - 3r_ir_j}{r^3}\right) = \frac{Q}{4\pi} = \frac{Q}{4\pi} \frac{\delta_{ii} r^2 - 3r_{i}r_{i}}{r^3} = \frac{Q}{4\pi}\frac{3r^2 - 3r^2}{r^3} = 0$$
Sledi: $\displaystyle{\text{tr}Q = 0}$. To pomeni, da ima $Q$ natanko 5 neodvisnih komponent, gre namreč za linearno kombinacijo matrik:
$$\left\{
    \begin{bmatrix}
        1&& \\
        &-1& \\
        &&0 \\
    \end{bmatrix},
    \begin{bmatrix}
        1&& \\
        &1& \\
        &&-2 \\
    \end{bmatrix},
    \begin{bmatrix}
        &1& \\
        1&& \\
        &&& \\
    \end{bmatrix},
    \begin{bmatrix}
        &&1 \\
        &&& \\
        1&& \\
    \end{bmatrix},
    \begin{bmatrix}
        &&& \\
        &&1 \\
        &1& \\
    \end{bmatrix}
\right\}$$
\paragraph{Izračun multipolnih koeficientov.} Lahko si mislimo, da je vsak potencial kombinacija več polov (mono-, di-, kvadru-).
$$U(\vct{r}) = \frac{1}{4\pi}\iiint \frac{1}{\left|\vct{r} - \vct{r}'\right|}\rho(r)'\dif\vct{r}'$$
$$= \frac{1}{2\pi r} \iiint \frac{1}{\sqrt{1 - \frac{2\vct{r}\cdot\vct{r}'}{r^2} + \frac{r'^2}{r^2}}}\rho(\vct{r}')\dif\vct{r}'$$
Tu bomo predpostavili, da je $r \gg r'$, kar nam omogoča koren razviti po Taylorju:
$$\frac{1}{\sqrt{1 - \frac{2\vct{r}\cdot\vct{r}'}{r^2} + r^2}} \approx 1 + \frac{1}{2}\left(\frac{2\vct{r}\cdot\vct{r}'}{r^2} - \frac{r'^2}{r^2}\right) + \frac{3}{2}\left(...\right)^2 = 1 + \frac{\vct{r}\cdot\vct{r}'}{r^2} + \frac{3(\vct{r}\cdot\vct{r}') - r'^2r^2}{2\pi r^4} + ...$$
Prvi člen nas spominja na potencial okoli monopola, drugi na potencial okoli dipola, tretji pa na potencial okoli kvadrupola.
$$U(\vct{r}) = \frac{1}{4\pi r} \int \rho(\vct{r}')\dif V' + \frac{\vct{r}'}{4\pi r^3} \cdot \int \vct{r}'\rho(\vct{r}')\dif V' + \frac{r_ir_j}{4\pi r^5}\int \frac{3r_i'r_j' - \delta_{ij}r^2}{2}\rho(\vct{r}')\dif V'$$
Prvi integral je ravno $e$, drugi je $\vct{p}$, tretji pa (približno) $Q$. Razvijamo pa lahko še naprej:
$$U(\vct{r}') = \frac{e}{4\pi r} + \frac{p_ir_i}{4\pi r^3} + \frac{Q_{ij}r_ir_j}{4\pi r^5} + \frac{T_{ijk}r_ir_jr_k}{4\pi r^7} + \frac{K_{ijkl}r_ir_jr_kr_l}{4\pi r^9} + ...$$
Vendar je v kartezičnih koordinatah s tem težko delati, zato bomo zaenkrat ostali pri kvadrupolih.
Poleg tega smo dobili asimptotsko vrsto, zahtevali smo namreč, da je $r$ velik. \\
Opazimo tudi, da imajo multipoli višjih redov zelo majhen doseg ($r^{-5},\,r^{-7}$).
\paragraph{Nesingularni multipoli.} Če namesto $r \gg r'$ predpostavimo $r \ll r'$, lahko naredimo navadno Taylorjevo vrsto:
$$U(\vct{r}) = \tilde{a} + \tilde{p}_ir_i + \tilde{Q}_{ij}r_ir_j + \tilde{T}_{ijk}r_ir_jr_k$$
Zahtevamo $\nabla^2U = 0$:
$$\nabla^2U = \tilde{Q}_{ij}\nabla_l\nabla_lr_ir_j = \tilde{Q}_{ij} \nabla_l\left[\delta_{il}r_j + \delta_{jl}r_i\right]$$
$$ = 2\tilde{Q}_{ij}\delta_{ij} = 2\tilde{Q}_{ii}$$
Da je naša zahteva izpolnjena, mora biti tr$\tilde{Q} = 0$.
Na podoben način lahko obravnavamo tudi $\tilde{T}_{ijk}$ in dobimo pogoj
$$2\tilde{T}_{ijk}\left[\delta_{ij}r_k + \delta_{ik}r_j + \delta_{jk}r_i\right] = 0$$
$$= 2r_k\left[T_{iik} + T_{iki} + T_{kii}\right] = 0$$
V bistvu smo torej dobili tri pogoje za oktupol. Izkaže se, da tem pogojem zadoščajo vsi tenzorji, ki so linearna kombinacija 7 različnih neodvisnih tenzorjev. \\[2mm]
V splošnem: Za tenzor reda $l$ potrebujemo $\displaystyle{\Binom{2+l}{l}}$ baznih tenzorjev, prav tako reda $l$. Ker nam zahteva $\nabla^2U = 0$ da dodatne linearno neodvisne pogoje, za vsak pogoj odštejemo en tenzor. \\[2mm]
Tako za $l=0$ (monopol) potrebujemo 1 bazni tenzor reda 0 (konstanta),
za $l=1$ (dipol) potrebujemo 3 bazne vektorje, za $l=2$ (kvadrupol) potrebujemo pet baznih matrik, za $l=3$ (oktupol) potrebujemo 7 baznih tenzorjev in tako naprej.
\paragraph{Potencial v sferičnih koordinatah.} Poglavitna razlika je v diferencialu $\dif V' = r^2\sin\vartheta\dif r \dif \varphi \dif \vartheta$.
$$\frac{1}{4\pi r\sqrt{1 - 2\frac{r'}{r}\cos\vartheta + \frac{r'^2}{r^2}}} = \frac{1}{4\pi r}\Sum{l} \left(\frac{r'}{r}\right)^lP_l(\cos\vartheta)$$
$P_l$ so Legendrovi polinomi:
\begin{itemize}
    \item $P_0(x) = 1$
    \item $P_1(x) = x$
    \item $P_2(x) = \frac{1}{2}\left(3x^2 - 1\right)$
    \item itd.
\end{itemize}
Tedaj je potencial enak
$$U(\vct{r}) = \frac{1}{4\pi r}\Sum{l}\frac{1}{r_l}\iiint\rho(\vct{r}')r'^l P_l(\cos\vartheta) dV'$$
$$ = \frac{1}{4\pi r^{l+1}}\Sum{l} \frac{1}{2l+1}\sum_{m=-l}^{l} \iiint r'^{l}\mathcal{Y}^*_{lm}(\vartheta', \varphi')\mathcal{Y}_{lm}(\vartheta, \varphi)\rho(\vartheta', \varphi', r') r'^2\dif r \dif(\cos\vartheta)\dif\varphi,$$
kjer so $\mathcal{Y}_{lm}$ sferični harmoniki oblike $\mathcal{Y}_{lm} = N_{lm}P_l(\cos\vartheta)e^{im\varphi}$. Bolj specifično je $\displaystyle{N_{lm} = \sqrt{\frac{2l + 1}{4\pi}\frac{(l-m)!}{(l+m)!}}}$. Uporabni so zato, ker so orotnormirani, kar nam integral močno poenostavi.
Dobili smo nekaj podobnega Fourierjevi transformaciji, vendar v treh dimenzijah. \\[2mm]
Fourierjeva transformacija:
$$f(x) = \frac{1}{4\pi} \Sum{k}\cos(kx) \int \cos(kx')f(x')\dif x'$$
Lahko si predstavljamo, da je $\cos(kx)$ je oblika sferičnega harmonika $\mathcal{Y}_{lm}\left(\vartheta, \varphi\right)$, $\cos(kx')$ pa je oblika sferičnega harmonika $\mathcal{Y}_{lm}\left(\vartheta', \varphi'\right)$. \\[3mm]
Številu $l$ pravimo tudi število vozelnih črt, številu $m$ pa število poldnevniških redov. Lahko si predstavljamo, da režemo kroglo, kjer $m$ predstavlja število rezov vzdolž poldnevnika, $l$ pa reze vzdolž vzporednikov. \\
Seveda pa s tem rešimo tudi orbitale $H$ atoma - $s$ orbitale so monopoli, $p$ orbitale so dipoli, $d$ orbitale so kvadrupoli in $f$ orbitale so oktupoli.
\paragraph{Vpliv $\nabla^2$ na sferične harmonike.} Rešujemo enačbo $\nabla^2U = 0$.
\begin{itemize}
    \item $\displaystyle{U = \frac{1}{r^{l+1}}\mathcal{Y}_{lm}(\vartheta, \varphi)}$ reši enačbo $\nabla^2U = 0$, tedaj ima $U$ singularnost v izhodišču.
    \item $\displaystyle{U = r^{l}\mathcal{Y}_{lm}(\vartheta, \varphi)}$ reši enačbo $\nabla^2U = 0$, tedaj $U$ nima singularnosti.
\end{itemize}
\paragraph{$\nabla^2$ v sferičnih koordinatah.} $$\nabla^2f = \frac{1}{r^2}\pd{}{r}\left(r^2\pd{f}{r}\right) + \frac{1}{r^2}\left(\frac{1}{\sin\vartheta}\pd{}{\vartheta}\left(\sin\vartheta\pd{f}{\vartheta}\right) + \frac{1}{\sin^2\vartheta}\pd{f}{\varphi}\right)$$
$$= \frac{1}{r^2}\pd{}{r}\left(r^2\pd{f}{r}\right) + \frac{1}{r^2}\nabla^2_\perp f$$
Velja: $\displaystyle{\nabla^2_\perp \mathcal{Y}_{lm} = -l(l+1)\mathcal{Y}_{lm}}$. To pomeni, da so sferični harmoniki lastne funkcije Laplacovega operatorja in jih lahko zato uporabljamo za vsako enačbo, v kateri nastopa $\nabla^2$.
\paragraph{Primer: Valovna enačba.}
$$\nabla^2f = \frac{1}{c^2}\pd{^2f}{t^2}$$
Vstavimo $f(\vct{r}, t) = f(\vct{r})e^{iwt}$ (enačba nihanja z lastno frekvenco).
$$\nabla^2f = -\frac{w^2}{c^2}f$$
Temu pravimo Helmholtzova enačba, ki jo rešimo z nastavkom $\displaystyle{f = \mathcal{R}_l(r)\mathcal{Y}_{lm}(\vartheta, \varphi)}$
$$\frac{1}{r^2}\pd{}{r}\left(r^2\pd{R_l}{r}\right) \mathcal{Y}_{lm} - \frac{1}{r^2}\mathcal{R}_l\cdot l(l+1)\mathcal{Y}_{lm} = -k\mathcal{R}_l\mathcal{Y}_{lm}$$
Zdaj lahko krajšamo $\mathcal{Y}_{lm}$ in dobimo enačbo za $\mathcal{R}_l$. To pa rešijo Besselove funkcije. Te imajo najlepšo obliko (kotne funkcije) v sferičnih koordinatah, npr. $\displaystyle{j_0 = \frac{\sin x}{x}}$.
\paragraph{Vektorski sferični harmoniki.} Obstajajo. So oblike $\vct{\mathcal{Y}}_{lm} = \hat{e}_r\mathcal{Y}_{lm}$. Lahko pa vzamemo tudi potencialno ali rotorsko polje sferičnih harmonikov.
\paragraph{Schrödingerjeva enačba.} $$-\frac{\hbar^2}{2m}\nabla^2\psi + \frac{e^2}{4\pi\varepsilon_0 r}\psi = E\psi$$
Ponovno uvedemo $\psi = \mathcal{R}_{nl}(r)\mathcal{Y}_{lm}(\vartheta, \varphi)$ in ostane le enačba za $\mathcal{R}(r)$. Ta sicer nima posebej lepe rešitve, toda kotnega dela smo se s pomočjo sferičnih harmonikov zelo hitro znebili.
\paragraph{Opomba.} Sferične harmonike lahko na tak način uporabljamo le, če je potencial neodvisen od $\vartheta$ in $\varphi$. V bistvu se z njimi znebimo odvodov po kotu.
\end{document}