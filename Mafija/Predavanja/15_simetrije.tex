\documentclass[a4paper]{article}
\usepackage{amsmath, amssymb, amsfonts}
\usepackage[margin=1in]{geometry}
\usepackage{graphicx}
\usepackage{tikz}
\usepackage{esint}
\setlength{\parindent}{0em}
\setlength{\parskip}{1ex}
\newcommand{\vct}[1]{\overrightarrow{#1}}
\newcommand{\dif}{\,\mathrm{d}}
\newcommand{\pd}[2]{\frac{\partial {#1}}{\partial {#2}}}
\newcommand{\dd}[2]{\frac{\mathrm{d} {#1}}{\mathrm{d} {#2}}}
\newcommand{\C}{\mathbb{C}}
\newcommand{\R}{\mathbb{R}}
\newcommand{\Q}{\mathbb{Q}}
\newcommand{\Z}{\mathbb{Z}}
\newcommand{\N}{\mathbb{N}}
\newcommand{\fn}[3]{{#1}\colon {#2} \rightarrow {#3}}
\newcommand{\avg}[1]{\langle {#1} \rangle}
\newcommand{\Sum}[2][0]{\sum_{{#2} = {#1}}^{\infty}}
\newcommand{\Lim}[1]{\lim_{{#1} \rightarrow \infty}}
\newcommand{\Binom}[2]{\begin{pmatrix} {#1} \cr {#2} \end{pmatrix}}
\newcommand{\duline}[1]{\underline{\underline{#1}}}

\begin{document}
\section*{Simetrije v fiziki}
\paragraph{Dve sklopljeni nihali.} Opravka imamo z diferencialno enačbo
\[J\ddot\varphi = \begin{bmatrix}
    -K & K' \\
    K' & -K
\end{bmatrix}\varphi = \begin{bmatrix}
    -K\varphi_1 + K'\varphi_2 \\ K'\varphi_1 - K\varphi_2
\end{bmatrix}\]
Ali drugače:
\[J\left(\ddot\varphi_1 + \ddot\varphi_2\right) = -(K-K')(\varphi_1 + \varphi_2)\]
\[J\left(\ddot\varphi_1 - \ddot\varphi_2\right) = -(K-K')(\varphi_1 - \varphi_2)\]
Označimo simetrično matriko \[S = \begin{bmatrix}
    0 & -1 \\ -1 & 0
\end{bmatrix}\]
Velja \(K = S^{-1}KS\) ali drugače \(SK = KS\). Sledi, da \(K\) in \(S\) komutirata, torej sta simultano diagonalizabilni.
\paragraph{Zrcaljenje funkcij realnih števil.} Naj bosta operatorja
\[\widehat{n} f(x) = f(-x)\]
\[\widehat{e} f(x) = f(x)\]
Lihe funkcije so tiste, za katere velja \(\widehat{n} f(x) = -f(x)\) in \(\widehat{e} f(x) = f(x)\). Sode funkcije so tiste,
za katere velja \(\widehat{n}f(x) = f(x)\) in \(\widehat{e}f(x) = f(x)\). \\
Operatorja \(\widehat{n}\) in \(\widehat{e}\) tvorite grupo z operacijo kompozituma (velja \(\widehat{e}\widehat{e} = 1, \widehat{e}\widehat{n} = \widehat{n}\widehat{e} = \widehat{n}\) ter \(\widehat{n}\widehat{n} = \widehat{e}\)). Lahko zapišemo sodo in liho upodobitev:
\[s(x) = \frac{f(x) + f(-x)}{2} = \frac{\widehat{e}f(x) + \widehat{n}f(x)}{2}\]
\[l(x) = \frac{f(x) - f(-x)}{2} = \frac{\widehat{e}f(x) - \widehat{n}f(x)}{2}\]
To, ali \(\widehat{n}f(-x)\) odštejemo ali odštejemo, so nekakšne lastne vrednosti operatorja. Ravno tako za \(\widehat{e}\), čeprav ima ta le eno lastno vrednost: \(1\). Delimo z \(2\), saj je to moč grupe.
V splošnem take projekcije opravimo po formuli \[\vct{v}_\chi = \frac{\sum \chi_i\widehat{g_i}\vct{v}}{|g|}\]
\end{document}