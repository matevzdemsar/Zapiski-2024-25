\documentclass[a4paper]{article}
\usepackage{amsmath, amssymb, amsfonts}
\usepackage[margin=1in]{geometry}
\usepackage{graphicx}
\usepackage{tikz}
\usepackage{esint}
\setlength{\parindent}{0em}
\setlength{\parskip}{1ex}
\newcommand{\vct}[1]{\overrightarrow{#1}}
\newcommand{\dif}{\mathrm{d}}
\newcommand{\pd}[2]{\frac{\partial {#1}}{\partial {#2}}}
\newcommand{\dd}[2]{\frac{\mathrm{d} {#1}}{\mathrm{d} {#2}}}
\newcommand{\C}{\mathbb{C}}
\newcommand{\R}{\mathbb{R}}
\newcommand{\Q}{\mathbb{Q}}
\newcommand{\Z}{\mathbb{Z}}
\newcommand{\N}{\mathbb{N}}
\newcommand{\fn}[3]{{#1}\colon {#2} \rightarrow {#3}}
\newcommand{\avg}[1]{\langle {#1} \rangle}
\newcommand{\Sum}[2][0]{\sum_{{#2} = {#1}}^{\infty}}
\newcommand{\Lim}[1]{\lim_{{#1} \rightarrow \infty}}
\newcommand{\Binom}[2]{\begin{pmatrix} {#1} \cr {#2} \end{pmatrix}}
\newcommand{\duline}[1]{\underline{\underline{#1}}}

\begin{document}
\section{Nelinearne diferencialne enačbe 1. reda}
Imamo enačbo oblike
$$\dot{\vct{x}} = f(\vct{x}) = A(\vct{x} - \vct{x_0}) + \mathcal{O}(x^2),$$
kjer je $f(\vct{x_0}) = 0$ in $\displaystyle{A_{ij} = \pd{f_i}{x_j}(\vct{x_0})}$. Naredili smo prvi člen Taylorjevega razvoja.
Naslednji člen bi bil $\displaystyle{\frac{1}{2}(x-x_0)_j(x-x_0)_k\pd{f_i}{x_j}\pd{f_i}{x_k}}$.
Ker so nelinearne diferencialne enačbe v splošnem zelo težko rešljive, bomo reševali le v okolizi ničel, kjer je naš enočlenski razvoj dovolj dober. \\

Matriko $A$ diagonaliziramo (to se običajno da). V smeri lastnih vektorjev je rešitev diferencialne enačbe $e^{\lambda t}$, pri čemer je lahko $\lambda$ realna ali kompleksna. \\
Če ima matrika $A$ realne lastne vrednosti, lahko navedemo še dva posebna primera: Če so vse $\lambda > 0$, govorimo o odbojni točki oziroma izvoru. Gre za nestabilno stacionarno točko. Če so vse $\lambda < 0$ govorimo o privlačni točki ali ponoru. To je stabilna stacionarna točka.
Najpogosteje so nekatere lastne vrednosti večje od 0, nekatere pa manjše. Še posebej pogosti so ti primeri zato, ker je tedaj sled matrike $A$ lahko enaka 0, kar je pogosta lastnost Hamiltonskih sistemov.
Nazadnje imamo primere, ko so nekatere lastne vrednosti enake 0. V teh primerih imamo smeri, v katere je obnašanje rešitve neeksponentno in se splača pri računanju uporabiti še drugi člen Taylorjevega razvoja. \\[3mm]
Če so lastne vrednosti kompleksne, imamo opravka s konjugiranimi pari $a \pm bi$.
$$A(\vct{u} \pm \vct{v}i) = (a \pm b_i)(\vct{u} \pm \vct{i})$$
$$\vct{x}(t) = A^+ e^{(a + bi) t}(\vct{u} + \vct{v}i) + A^- e^{(a - bi) t}(\vct{u} - \vct{v}i)$$
$$= e^{at}\left(A^+ (\cos bt + i\sin bt)(\vct{u} + \vct{v}i) + A^- (\cos bt - i\sin bt) (\vct{u} - \vct{v}i)\right)$$
Vrednost $a$ (oziroma njen predznak) nam pove, ali gre za privlačno ali odbojno točko. Drugi faktor je periodičen. Se pravi gre običajno za nekakšno spiralo proti ali stran od točke $\vct{x_0}$
\begin{align*}
    \vct{x}(t) & = e^{at} \vct{u} \left[(A^+ + A^-) \cos bt + i(A^+ - A^-)\sin bt\right] \\
    & + e^{at} \vct{v} \left[-(A^+ + A^-)\sin bt + i(A^+ - A^-)\cos bt\right]
\end{align*}
Ker mora biti končni rezultat realen, mora biti $A^+ = \overline{A^-}$, tedaj je
$$\vct{x}(t) = 2e^{at}\left(\mathfrak{Re}(A) \vct{u} \cos bt - \mathfrak{Im}(A) \vct{u} \sin bt - \mathfrak{Im}(A) \vct{v} \cos bt + \mathfrak{Re}(A) \vct{v} \sin bt\right)$$
Če izpostavimo člena z $A_r$ in $A_i$, dobimo rotacijsko matriko, ki sistem $\vct{u}, \vct{v}$ zasuka za kot $bt$. \\[3mm]
Poseben primer: $\lambda = \pm bi$ brez realnega dela. Tedaj je lahko $\mathrm{Tr}A = 0$, poleg tega je funkcija okoli točke ciklična.
\paragraph{Hamiltonski sistem.}
$$\pd{}{t}\begin{bmatrix}
    q \\ p
\end{bmatrix} = \begin{bmatrix}
    \pd{H}{p} \\ - \pd{H}{q}
\end{bmatrix} \approx \begin{bmatrix}
    \pd{^2 H}{p \partial q} & \pd{^2 H}{p^2} \\
    \pd{^2 H}{p^2} & \pd{^2 H}{q \partial p}
\end{bmatrix}\begin{bmatrix}
    q-q_0 \\ p-p_0
\end{bmatrix}$$
Vidimo, da za Hamiltonski sistem velja $\mathrm{Tr}A = 0$, torej imemo lahko samo krožne točke in sedla, kar pa nadaljnje pomeni,
da so trajektorije gibanje v prostoru take, da ohranjajo energijo.
\paragraph{Primer: Lotka-Voltera} Obravnavamo velikost populacije plenilcev ($L$) in plena ($Z$).
$$\dot{Z} = \alpha Z - \beta ZL = (\alpha - \beta L) Z$$
$$\dot{L} = -\gamma L + \delta ZL = (-\beta + \delta Z) L$$
$$\dd{Z}{L} = \frac{\alpha Z - \beta LZ}{-\gamma L + \delta LZ}$$
$$\int (-\gamma + \delta z) \frac{\dif Z}{Z} = \int (\alpha - \beta L)\frac{\dif L}{L}$$
$$-\gamma \ln\frac{Z}{Z_0} + \delta (Z - Z_0) = \alpha \ln\frac{L}{L_0} - \beta (L-L_0)$$
Če narišemo graf $L(Z)$, ima ta obliko nekakšne sklenjene krivulje okoli neke točke. To točko poiščemo z odvodom, da dobimo stacionarno točko, mora namreč veljati
$$\dot Z = \dot L = 0$$
Dobimo dve stacionarni točki:
\begin{align*}
    L_0 & = \frac{\alpha}{\beta} & L_0 & = 0 \\
    Z_0 & = \frac{\gamma}{\delta} & Z_0 & = 0 \\
\end{align*}
Prva točka je krožna točka, druga pa sedlo. Izračunamo lahko tudi obhodne čase za majhne odmike iz stacionarne točke $(\alpha/\beta, \gamma/\delta)$, in sicer imamo tedaj opravka z odvisnostjo $$L, Z \propto e^{i\sqrt{\alpha\gamma}t}$$
Časovne odvisnosti $L(t)$ in $Z(t)$ pa ne moremo enostavno izračunati.
\paragraph{Model laserja.} Imamo fotone in atome - njih števila označimo s $F$ in $A$.
$$\dot F = - \alpha F + \gamma A F = F(\gamma A - \alpha)$$
$$\dot A = - \beta A - \delta AF + R$$
$R$ označuje črpanje atomov v prostor (lahko je tudi negativno, če jih črpamo ven).
\begin{enumerate}
    \item Ravnovesje pri $F=0, A = \beta/R$
    $$J = \begin{bmatrix}
        \pd{\dot F}{F} & \pd{\dot F}{A} \\
        \pd{\dot A}{F} & \pd{\dot A}{A} \\
    \end{bmatrix} = \begin{bmatrix}
        -\alpha + \gamma A & \gamma F \\
        - \delta A & -\beta - \delta F
    \end{bmatrix}$$
    Vstavimo $F$ in $A$ ter dobimo
    $$\begin{bmatrix}
        -\alpha + \gamma\frac{R}{\beta} & 0 \\
        -\frac{\delta R}{\beta} & -\beta
    \end{bmatrix}$$
    Točka bo odbojna, če je $R > \frac{\alpha \beta}{\gamma}$, sicer pa privlačna.
    \item Ravnovesje pri $A_0 = \frac{\alpha}{\gamma}$ in $F_0 = \frac{R\gamma - \alpha\beta}{\alpha\delta}$ \\
    Uvedemo brezdimenzijske količine:
    $$\dot f = \alpha(-f + af)$$
    $$\dot a = \beta(a - af + r)$$
    Pri $f = F\frac{\delta}{\beta}$ imamo ravnovesje. Tedaj je matrika za 2. stacionarno točko enaka
    $$J = \begin{bmatrix}
        0 & \alpha(r-1) \\
        -\beta & -\beta r
    \end{bmatrix}$$
    Vidimo, da bomo imeli prag pri $r=1$. Izračunamo karakteristični polinom
    $$\lambda^2 + \beta r \lambda + \alpha\beta (r-1) = 0$$
    Označimo $\alpha\beta(r-1) = \omega_0^2$. $\omega_0$ je frekvenca nihanja nedušenega sistema, $\frac{\beta r}{2}$ pa je koeficient dušenja.
    Razmerje med njima opisuje koeficient dobrote (kvalitete): $$Q = \sqrt{\frac{a(r-1)}{\beta r^2}}$$
\end{enumerate}
\paragraph{(Kemijske) reakcije.} Iz kemijskih reakcij lahko zapišemo enačbe kemijske kinetike, npr.pri reakciji $A + B \to C$.
$$\dot A = -q AB$$
$$\dot B = -q AB$$
Vemo tudi, da je $A - B$ konstanta, označimo $K$
$$\dot A = -qA(A - K)$$
Ko integriramo, dobimo $$A(t) = \frac{A_0}{\frac{A_0}{K}+\left(1 - \frac{A_0}{K}\right)e^{-Kqt}}$$
Poseben primer pa dobimo, če je $K = 0$:
$$\dot A = -qA^2$$
$$A(t) = \frac{A_0}{1 + A_0qt}$$
Reakcijam, pri katerih je $K=0$, rečemo reakcije 2. reda (ostalim pa reakcije 1. reda).
\paragraph{Model epidemije.} Imamo tri neznanke: število bolnih ($B$), število dovzetnih ($D$) in število imunih ($I$).
$$\dot B = \alpha BD - \beta B$$
$$\dot D = -\alpha BD + \gamma I$$
$$\dot I = - \gamma I + \beta B$$
Ker je $B + D + I$ konstanta (nekateri, ki zbolijo, sicer umrejo, ampak s tem tehnično gledano postanejo imuni) in če je $\gamma = 0$ (torej če imuni ostanejo imuni - ali mrtvi), lahko stvar poenostavimo tako kot pri zajcih in lisicah:
$$\dd{B}{D} = \frac{\alpha D - \beta}{\alpha \beta} \frac{B}{D}$$
Ko to integriramo, dobimo
$$B - B_0 + D - D_0 = \frac{\beta}{\alpha} \ln \frac{D}{D_0}$$
\end{document}