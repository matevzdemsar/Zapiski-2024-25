\documentclass[a4paper]{article}
\usepackage{amsmath, amssymb, amsfonts}
\usepackage[margin=1in]{geometry}
\usepackage{graphicx}
\usepackage{tikz}
\usepackage{esint}
\setlength{\parindent}{0em}
\setlength{\parskip}{1ex}
\newcommand{\vct}[1]{\overrightarrow{#1}}
\newcommand{\dif}{\mathrm{d}}
\newcommand{\pd}[2]{\frac{\partial {#1}}{\partial {#2}}}
\newcommand{\dd}[2]{\frac{\mathrm{d} {#1}}{\mathrm{d} {#2}}}
\newcommand{\C}{\mathbb{C}}
\newcommand{\R}{\mathbb{R}}
\newcommand{\Q}{\mathbb{Q}}
\newcommand{\Z}{\mathbb{Z}}
\newcommand{\N}{\mathbb{N}}
\newcommand{\fn}[3]{{#1}\colon {#2} \rightarrow {#3}}
\newcommand{\avg}[1]{\langle {#1} \rangle}
\newcommand{\Sum}[2][0]{\sum_{{#2} = {#1}}^{\infty}}
\newcommand{\Lim}[1]{\lim_{{#1} \rightarrow \infty}}
\newcommand{\Binom}[2]{\begin{pmatrix} {#1} \cr {#2} \end{pmatrix}}
\newcommand{\duline}[1]{\underline{\underline{#1}}}

\begin{document}
\paragraph{Linearne diferencialne enačbe 1. reda.}
So oblike $\dot{\vct{x}} = A\vct{x} + \vct{f}(t)$. Členu $A\vct{x}$ rečemo homogeni del, členu $\vct{f}(t)$ pa partikularni del.
Če rešimo homogeni del, lahko na podlagi te rešitve konstruiramo tudi partikularno rešitev. Pri Matematiki III smo postopku rekli
variacija konstante, pri Mafiji to obravnavamo kot konvolucijo $\displaystyle\int_{0}^{\infty}\vct{G}(\vct{f}(t'), t-t')\dif t'$,
kjer je $\vct{G}(\vct{x_0}, t) = \vct{x}$ rešitev homogenega dela.
Imejmo zdaj sistem $\dot{\vct{x}} = A\vct{x}$. \\ Najprej si mislimo, da je $A$ diagonalizabilna.
$$A = PDP^{-1}$$
$$P^{-1}\dot{\vct{x}} = P^{-1}PDP^{-1}\vct{x}$$
Uvedemo spremenljivko $\vct{y} = P^{-1}\vct{x}$.
$$\dot{\vct{y}} = D\vct{y}$$
$$\vct{y} = e^{Dt}\vct{y_0}$$
Če $A$ ni diagonalizabilna, jo damo v Jordanovo bazo:
$$A = PJP^{-1}$$
Jordanova matrika je sestavljena iz celic oblike
$$J_k = \begin{bmatrix}
    \lambda_k & 1 &&& \\
    & \lambda_k & 1 && \\
    && \ddots & \ddots & \\
    &&& \lambda_k & 1
\end{bmatrix}$$
Da jo damo v tako obliko, moramo poiskati korenske vektorje te matrike. Ko jih najdemo, je rešitev oblike
$$\vct{x} = e^{At}\vct{x_0},~~e^{At} = Pe^{\lambda t} \begin{bmatrix}
    1 & t & t^2/2 & \dots & & \\
    & 1 & t & t^2/2 & \dots &  \\
    & & \ddots & \ddots & \ddots & \ddots \\
    &&& \ddots & \ddots & \ddots  \\
    &&&& \ddots & \ddots  \\
    &&&&& 1
\end{bmatrix} P^{-1}$$
\paragraph{Harmonično nihanje.} Harmonično nihanje opisuje sistem enačb 
$$\dd{}{t}\begin{bmatrix}
    x \\ v
\end{bmatrix} = \begin{bmatrix}
    0 & 1 \\
    -\frac{k}{m} & 0
\end{bmatrix}\begin{bmatrix}
    x \\ v
\end{bmatrix}$$
(Uvedli smo spremenljivko $v = \dot{x}$, sicer gre za enačbo $\ddot{x} = -\frac{k}{m}x$)
Uvedemo spremenljivki $y = \omega^{-1} v$ in $\tau = \omega t$.
$$\dd{}{\tau}\begin{bmatrix}
    x \\ y
\end{bmatrix} = \begin{bmatrix}
    0 & 1 \\
    -1 & 0
\end{bmatrix}\begin{bmatrix}
    x \\ y
\end{bmatrix}$$
Zdaj uporabimo prej opisani postopek za $\displaystyle{A = \begin{bmatrix}
    0 & 1 \\
    -1 & 0
\end{bmatrix}}$ in dobimo sinusno nihanje.
\paragraph{Kritično dušenje.} To imamo opravka z enačbo oblike $\ddot{x} + 2\omega\dot{x} + \omega^2x = 0$ V matrični obliki ga zapišemo kot
$$\dd{}{t}\begin{bmatrix}
    x \\ v
\end{bmatrix} = \begin{bmatrix}
    0 & 1 \\
    -\omega^2 & -2\omega
\end{bmatrix}\begin{bmatrix}
    x \\ v
\end{bmatrix}$$
$$\dd{}{\tau}\begin{bmatrix}
    x \\ y
\end{bmatrix} = \begin{bmatrix}
    0 & 1 \\
    -1 & -2
\end{bmatrix}\begin{bmatrix}
    x \\ y
\end{bmatrix}$$
Ta matrika ni diagonalizabilna (koeficient dušenja mora biti $\beta = 2\omega$). Uporabimo Jordanovo matriko in dobimo
$$\begin{bmatrix}
    x \\ y
\end{bmatrix} = \begin{bmatrix}
    x_0 \\ y_0
\end{bmatrix} e^{-\tau} \begin{bmatrix}
    -1 & -1 \\
    1 & 0
\end{bmatrix} \begin{bmatrix}
    (x_0 + y_0) (1 + \tau) - y_0 \\
    y_0 - \tau(x_0 + y_0)
\end{bmatrix}$$
Tu je matrika $\displaystyle{\begin{bmatrix}
    -1 & -1 \\
    1 & 0
\end{bmatrix}}$ prehodna matrika $P$. Za ostale matrike smo že poskrbeli.
\paragraph{Problem kompartmentov.} Večina matrik je diagonalizabilnih. Problem kompartmentov je primer, ko se nediagonalizabilni matriki ne moremo izogniti.
Mislimo si, da imamo tri kompartmente z vodo, vezane enega za drugim. V prvega dodamo nek topljenec in gledamo, kako se spreminja masa topljenca v naslednjih kompartmentih.
\begin{eqnarray*}
    \dot{m}_1 = -\frac{\phi}{V}m_1 \\
    \dot{m}_2 = -\frac{\phi}{V}m_2 + \frac{\phi}{V}m_1 \\
    \dot{m}_3 = -\frac{\phi}{V}m_3 + \frac{\phi}{V}m_3 \\
\end{eqnarray*}
V matrični obliki to zapišemo kot
$$\dd{}{t}\begin{bmatrix}
    m_1 \\ m_2 \\ m_3
\end{bmatrix} = \frac{\phi}{V} \begin{bmatrix}
    -1 & 0 & 0 \\
    1 & -1 & 0 \\
    0 & 1 & -1
\end{bmatrix} \begin{bmatrix}
    m_1 \\ m_2 \\ m_3
\end{bmatrix}$$
Tako je $m_1 = e^{-\lambda t}m_0$, $m_2 = Ce^{-\lambda t} + Ate^{-\lambda t}$ in tako naprej. Tak sistem lahko rešimo s časovno zahtevnostjo $\mathcal{O}(n)$.
\paragraph{Sistem enačb s pasovno matriko - prevajanje toplote.} Imejmo matriko, ki ima neničelne vrednosti le na glavni diagonali in diagonalah ob njej.
Primer take matrike je sistem enačb za prevajanje toplote (med kompartmenti).
\begin{eqnarray*}
    m_1c\dot{T}_1 = h (T_2 - T_1) \\
    m_2c\dot{T}_2 = h (T_1 - T_2) + h (T_3 - T_2) \\
    m_3c\dot{T}_3 = h (T_2 - T_3) + h (T_4 - T_3) \\
\end{eqnarray*}
In tako naprej za naslednje kompartmente. V matrični obliki to enačbo zapišemo kot
$$\begin{bmatrix}
    m_1 c &&& \\
    & m_2 c && \\
    && m_3 c & \\
    &&& \ddots
\end{bmatrix}\dd{}{t}\begin{bmatrix}
    T_1 \\ T_2 \\ T_3 \\ \vdots
\end{bmatrix} = h\begin{bmatrix}
    -1 & 1 & \dots & \\
    1 & -2 & 1 & \\
    \vdots & 1 & 2 & \ddots \\
    & & \ddots & \ddots \\
\end{bmatrix}\begin{bmatrix}
    T_1 \\ T_2 \\ T_3 \\ \vdots
\end{bmatrix}$$
Imamo enačbo oblike $M\dot{\vct{T}} = K\vct{T}$
Imamo posplošen primer lastnih vrednosti:
$$\vct{T} = \vct{T_{n}}e^{\lambda t}$$
Lastne vrednosti dobimo iz enačbe $\det (K - \lambda M) = 0$.
\paragraph{Lastni načini.} V limiti $n \to \infty$, kjer $n$ predstavlja število enačb, naša enačba $\dot{T}_n = T_{n-1} - 2T_{n} + T_{n+1}$
predstavlja drugi odvod po kraju. Dobimo torej
$$\pd{}{t}T = \Delta x^2 \nabla^2 T - qT$$
ali difuzijsko enačbo. Njena rešitev je kombinacija funkcij oblike $T = e^{\lambda t} T(x)$, ki jih imenujemo lastni načini. Pri tem mora veljati:
$$\lambda T = \Delta x^2 \nabla^2 T$$
Rešujemo z nastavkom $T = A\cos(kx)$
$$A \lambda T = \Delta x^2 A (-k^2) T$$
$$\lambda = -k^2 - 1$$
\paragraph{Kompleksne rešitve 2D sistemov.} Recimo, da sistem opisuje matrika $\displaystyle{A = \begin{bmatrix}
    0 & -1 \\
    1 & 0
\end{bmatrix}}$. Tedaj je $$e^{At} = \begin{bmatrix}
    \cos t & -\sin t \\
    \sin t & \cos t
\end{bmatrix} = I\cos t + A \sin t$$
Tako lastnost imajo tri bazne matrike:
$$i\sigma_x = \begin{bmatrix}
    0 & i \\
    i & 0
\end{bmatrix}~~i\sigma_y = \begin{bmatrix}
    0 & 1 \\
    -1 & 0
\end{bmatrix}~~i\sigma_z = \begin{bmatrix}
    i & 0 \\
    0 & -i
\end{bmatrix}$$
Te matrike imenujemo tudi Pavlijeve matrike. Imajo sledeče lastnosti:
\begin{eqnarray*}
    \sigma_x\sigma_y = i\sigma_z
    \sigma_y\sigma_z = i\sigma_x
    \sigma_z\sigma_x = i\sigma_y
\end{eqnarray*}
Vrh tega lahko z njimi generiramo rotacijske matrike.
\paragraph{Eulerjeva delta.} Označimo $\vct{n}$ ... os rotacije in $\sigma_i$ ... Pavlijeva matrika.
$$e^{i(n_i\sigma_i) = \cos\frac{\varphi}{2} + i\sigma_i n_i \sin\frac{\varphi}{2}}$$
\paragraph{Dušeno vrtenje.} Imamo sistem
$$\dd{}{t} \begin{bmatrix}
    x \\ y
\end{bmatrix} = A \begin{bmatrix}
    x \\ y
\end{bmatrix},~~~A = \begin{bmatrix}
    -\beta & \omega \\
    -\omega & -\beta
\end{bmatrix}$$
Vidimo, da je $A$ linearna kombinacija ene od Pavlijevih matrik in identitete. To pomeni, da je
$$e^{At} = e^{(-\beta I + \omega(i\sigma_y))t} = e^{-\beta t} e^{w(i\sigma_y)t} = e^{-\beta t} \begin{bmatrix}
    \cos \omega t & \sin \omega t \\
    - \sin \omega t & \cos \omega t \\
\end{bmatrix}$$
To smemo storiti, ker $I$ komutira z $i\sigma_i$ - sicer eksponent vsote ne bi bil enak produktu eksponentov.
\paragraph{Gibanje naboja v elektro-magnetnem polju.} Mislimo si, da imamo opravka še z dušenjem, koeficient katerega označimo z $\gamma$.
$$m\dot{\vct{v}} = e\vct{E} - \gamma\vct{v} + e\vct{v}\times\vct{B}$$
$$m\begin{bmatrix}
    \dot{v}_x \\ \dot{v}_y
\end{bmatrix} = e \begin{bmatrix}
    E_x \\ E_y
\end{bmatrix} + \gamma \begin{bmatrix}
    1 & 0 \\
    0 & 1
\end{bmatrix}\begin{bmatrix}
    v_x \\ v_y
\end{bmatrix} + eB \begin{bmatrix}
    0 & 1 \\
    -1 & 0
\end{bmatrix} \begin{bmatrix}
    v_x \\ v_y
\end{bmatrix}$$
Opravka imamo z matrično enačbo
$$\begin{bmatrix}
    0 & 1 \\
    -1 & 0
\end{bmatrix} \begin{bmatrix}
    x \\ y
\end{bmatrix} = \lambda \begin{bmatrix}
    x \\ y
\end{bmatrix}$$
Matrika je točno enaka $i\sigma_y$, ki pa je samo rotirana matrika $i\sigma_z$, ki ima lastni vrednosti $\pm i$. \\
Prehod v lastni sistem je ekvivalenten uvedbi nove spremenljivke $w = v_x + iv_y$. Imamo torej le eno kompleksno enačbo
$$m \dot w = e (E_x + iE_y) - \gamma w - eBiw$$
Homogeni del: $\displaystyle{w = w_0 e^{-\frac{\gamma + eBi}{m}t}}$ \\
Partikularni del: Označimo $f = e(E_x + iE_y)$
$$0 = f - (\gamma - eBi)w$$
$$w = \frac{f}{\gamma + eBi}$$
Za $v_x$ in $v_y$ samo vzamemo realni in imaginarni del te rešitve.
\paragraph{Faucaltovo nihalo} (nihalo v vrtečem se koordinatnem sistemu)
Opravka imamo z diferencialno enačbo $$m\ddot{\vct{x}} = 2m\vct{\Omega}\times\dot{\vct{x}} - m\frac{g}{l}\vct{x} - m\vct{g}$$
V matrični obliki jo zapišemo kot
$$\begin{bmatrix}
    m\ddot{u} \\ m\ddot{v}
\end{bmatrix} = 2m\begin{bmatrix}
    0 & -\Omega \\
    \Omega & 0
\end{bmatrix}\begin{bmatrix}
    \dot u \\ \dot v
\end{bmatrix} - \frac{mg}{l}\begin{bmatrix}
    u \\ v
\end{bmatrix}$$
Spet uvedemo spremenljivko $w$ in dobimo $$\ddot{w} = 2\Omega i \dot w - \omega_0^2 w$$
Rešimo z nastavkom $w = w_0e^{i\omega t}$, ki nam da karakteristično enačbo
$$-\omega^2 = 2\Omega (-\omega) - \omega_0^2$$
$$\omega = \Omega \pm \sqrt{\Omega^2 + \omega_0^2}$$
$$w = w_1 e^{i(\Omega + \sqrt{\Omega^2 + \omega_0^2})t} + w_2 e^{i(\Omega \pm \sqrt{\Omega^2 + \omega_0^2})t}$$
Če začnemo v koordinatnem izhodišču: $w_1 = -w_2$
$$w = 2w_1e^{i\Omega t}\sin\left(\sqrt{\Omega^2 + \omega_0^2}\right)t$$
Tu $w_1$ redstavlja začetno hitrost (in je kompleksen, torej predstavlja tako $x$ kot $y$ smer). $e^{i \Omega t}$ predstavlja vrtenje ravnine nihanja, $\sin\left(\sqrt{\Omega^2 + \omega_0^2}\right)t$
pa nihanje samo.
\paragraph{Kvaternioni.} Postopek, kakršnega smo uporabili pri obravnavi Faucaltovega nihala, je možen v dveh dimenzijah, kjer si lahko pomagamo s kompleksnimi števili.
V treh dimenzijah potrebujemo kvaternione. Nekako gre za opis vektorskega prostora, katerega baze so Pavlijeve matrike in identiteta.
$$Q = a_0 + a_1 \cdot i + a_2 \cdot j + a_3 \cdot k$$
$$i^2 = j^2 = k^2 = -1$$
\begin{eqnarray*}
    ij = k = -ji \\
    jk = i = -kj \\
    ki = j = -ik \\
    \\
    i \leftrightarrow i\sigma_x
    j \leftrightarrow i\sigma_y
    k \leftrightarrow i\sigma_z
\end{eqnarray*}
Rotacijo $P$ lahko predstavimo kot $\displaystyle{P = \cos\frac{\varphi}{2} + \vct{n}\cdot(i, j, k) \sin\frac{\varphi}{2}}$
Inverz te rotacije $P^*$ dobimo tako, da namesto $i$ vstavimo $-i$ in tako naprej.
Količino $Q = a_1 i + a_2 j + a_3 k$ pa si lahko predstavljamo kot vektor. To pomeni, da lahko na nek način seštevamo skalarje in vektorje:
$$A = s + \vct{v} = s + v_x i + v_y j + v_z k$$
$$B = t + \vct{w} = t + w_x i + w_y j + w_z k$$
$$AB = st - \vct{v}\cdot\vct{w} + \vct{t}\times\vct{w} + s\vct{w} + t\vct{v}$$
To je uporabno, zaenkrat pa omenimo bolj kot zanimivost.
\end{document}