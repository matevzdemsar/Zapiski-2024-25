\documentclass[a4paper]{article}
\usepackage{amsmath, amssymb, amsfonts}
\usepackage[margin=1in]{geometry}
\usepackage{graphicx}
\usepackage{tikz}
\usepackage{esint}
\setlength{\parindent}{0em}
\setlength{\parskip}{1ex}
\newcommand{\vct}[1]{\overrightarrow{#1}}
\newcommand{\dif}{\mathrm{d}}
\newcommand{\pd}[2]{\frac{\partial {#1}}{\partial {#2}}}
\newcommand{\dd}[2]{\frac{\mathrm{d} {#1}}{\mathrm{d} {#2}}}
\newcommand{\C}{\mathbb{C}}
\newcommand{\R}{\mathbb{R}}
\newcommand{\Q}{\mathbb{Q}}
\newcommand{\Z}{\mathbb{Z}}
\newcommand{\N}{\mathbb{N}}
\newcommand{\fn}[3]{{#1}\colon {#2} \rightarrow {#3}}
\newcommand{\avg}[1]{\langle {#1} \rangle}
\newcommand{\Sum}[2][0]{\sum_{{#2} = {#1}}^{\infty}}
\newcommand{\Lim}[1]{\lim_{{#1} \rightarrow \infty}}
\newcommand{\Binom}[2]{\begin{pmatrix} {#1} \cr {#2} \end{pmatrix}}
\newcommand{\dunderline}[1]{\underline{\underline{#1}}}
\begin{document}
\paragraph{Tenzorski račun.} \text{} \\
$$\vct{j} = \sigma \vct{E} = \sigma E^j\hat{e}_j$$
$$j_i = \hat{e}_i\sigma\hat{e}_j E^j = \sigma_{ij}E^j$$
Izrazili smo tenzor drugega reda: $\underline{\underline{\sigma}} = \sigma_{ij}\hat{e}^i\hat{e}^j = \sigma_{ij} \hat{e}_i \otimes \hat{e}_j$
$$\underline{\underline{\sigma}}\vct{E} = \sigma_{ij}\vct{E}_j\hat{e}_i\left[\hat{e}_j\cdot\hat{e}_i\right]$$
Se pravi gre za nekakšno projekcijo. Če je $\dunderline{\sigma}$ simetričen, obstaja baza, v kateri je diagonalen:
$$\dunderline{\sigma} = \sigma_{xx}\hat{e}_x\otimes\hat{e}_x + \sigma_{yy}\hat{e}_y\otimes\hat{e}_y + \sigma_{zz}\hat{e}_z\otimes\hat{e}_z$$
Pri tem ponovno poudarimo, da je $\vct{n} \otimes \vct{n}$ projekcija na $\vct{n}$, kajti:
$$(\vct{n} \otimes \vct{n})\cdot\vct{v} = \vct{n}\cdot(\vct{n}\cdot\vct{v})$$
V nekaterih snoveh lahko uporabljamo izotopni tenzor:
$$\sigma_{ij} = \sigma\delta_{ij}$$
To pomeni, da so lastnosti snovi neodvisne od osi (tj. v smeri $x$ enake kot v smeri $y$ in $z$). V snoveh, kot so voda, to dobro velja, v snoveh, kot je les, pa ne nujno.
Les je namreč sestavljen iz vlaken, ki po navadi tečejo v isto smer.
\paragraph{Simetrije in dekompozicija.} Obravnavajmo na primer gostoto moči: $q = \vct{j} \cdot \vct{E} = \vct{E} \dunderline{\sigma} \vct{E}$
Temu pravimo kvadratna forma. \\
Za električno polje dobimo podoben primer: $\vct{D} = \varepsilon\varepsilon_0\vct{E} \Rightarrow \vct{D}\cdot\vct{E} = \varepsilon_0 (\vct{E} \dunderline{\varepsilon} \vct{E})$.
Takšna simetrija po navadi izhaja iz ohranitvenih zakonov (in variacijskega principa):
$$T_{ij} = \frac{1}{2}\left(T_{ij} + T_{ji}\right) - \frac{1}{2}\left(T_{ij} - T_{ji}\right) =: S_{ij} + A_{ij}$$
Če to zapišemo v tenzorski obliki:
$$T = \frac{1}{2}\left(T + T^T\right) + \frac{1}{2}\left(T - T^T\right)$$
Čim velja $A_{ij} = 0$, je $T = T^T$
\paragraph{Vztrajnostni moment.} Pustili si bomo predpostaviti, da gre za togo telo. Tedaj velja $\vct{\Gamma} = \dunderline{J}\vct{\omega}$
$$\dif\vct{\Gamma} = \vct{r} \times \dif\vct{G} = \rho \vct{r} \times (\vct{omega} \times \vct{r}) \dif V$$
$$\vct{\Gamma}_i = \int \dif\Gamma_i = \iiint \rho \varepsilon_{ijk}r_j\varepsilon_{klm} \omega_l r_m \dif V$$
$$ = \iiint \rho (\delta_{il}\delta_{jm} - \delta_{im}\delta_{lj})r_jr_m\dif V\omega_l = \iiint\rho(\delta_{ij}r^2-r_jr_j)\dif V \omega_j$$
$$J_{ij} = \iiint \begin{bmatrix}
    y^2 + z^2 & -xy & -xz \\
    -xy & z^2 + x^2 & -yz \\
    -xz & -yz & x^2 + y^2
\end{bmatrix} \rho \, \dif x \, \dif y \, \dif z$$
Steinerjev izrek izrazimo kot $\displaystyle{J'_{ij} = J_{ij} + m\left(\delta_{ij} r'^2 - r'_i r'_j\right)}$. Nazadnje za rotacijsko energijo spet dobimo kvadratno formo:
$$W_r = \frac{1}{2}\vct{\omega} \dunderline{J} \vct{\omega}$$
\paragraph{Primer: Vrtenje togega telesa.} Upoštevamo ohranitev $\vct{\Gamma}$:
$$\dot{\vct{\Gamma}} = 0 = \dot{J}\vct{\omega} + J\dot{\vct{\omega}}$$
Naj bo lastni sistem $\hat{e}'_i$:
$$\hat{e}'_i = T_{ij}\hat{e}_j$$
$$\dot{\hat{e}}'_i = \Omega_{ij}\hat{e}'_j = \vct{\omega}\times\vct{e}'_j$$
$$\dunderline{J} = J_{11}\hat{e}'_1\otimes\hat{e}'_1 + J_{22}\hat{e}'_2\otimes\hat{e}'_2 + J_{33}\hat{e}'_3\otimes\hat{e}'_3$$
$$\dunderline{\dot{J}} = J_{11}\dot{\hat{e}}'_1\otimes\hat{e}'_1 + J_{22}\dot{\hat{e}}'_2\otimes\hat{e}'_2 + J_{33}\dot{\hat{e}}'_3\otimes\hat{e}'_3 + J_{11}\hat{e}'_1\otimes\dot{\hat{e}}'_1 + J_{22}\hat{e}'_2\otimes\dot{\hat{e}}'_2 + J_{33}\hat{e}'_3\otimes\dot{\hat{e}}'_3$$
Ko to izrazimo s koordinatami $\hat{e}'_x, \hat{e}'_y, \hat{e}'_z$, dobimo:
$$-J\dot{\vct{\omega}} = \dot{J}\vct{\omega} =$$
$$= J_x (\omega \times \hat{e}'_x)(\vct{\omega} \cdot \hat{e}'_x) + J_y (\omega \times \hat{e}'_y)(\vct{\omega} \cdot \hat{e}'_y) + J_z (\omega \times \hat{e}'_z)(\vct{\omega} \cdot \hat{e}'_z)$$
Dobimo sistem enačb:
$$J_x\dot{\vct{\omega}}_x = \left(J_z - J_y\right)\omega_y\omega_z$$
$$J_y\dot{\vct{\omega}}_y = \left(J_x - J_z\right)\omega_z\omega_x$$
$$J_x\dot{\vct{\omega}}_z = \left(J_y - J_x\right)\omega_x\omega_y$$
Ta sistem je matematično rešljiv (lahko dobimo $\omega$), vendar je naša rešitev rahlo neuporabna. Če si pustimo nekoliko poenostaviti problem, lahko na primer zahtevamo, da je vrtenje eno-osno, kar pomeni:
$$J_z = J_\parallel$$
$$J_x = J_y = J_\perp$$
Ta poenostavitev nam takoj da $\dot{\omega}_z = 0$, torej $\omega_z = \text{konst.}$, ostali dve enačbi pa imata rešitev
$$\omega_x = \omega_0 \cos\omega_pt$$
$$\omega_y = \omega_0 \sin\omega_pt$$
Torej precesijo $\omega$ okoli $\hat{e}_z$ s frekvenco $\omega_p$.
\paragraph{Inverzi in robni pogoji.} Vzemimo za primer toplotno prevodnost skozi ploščo debeline $d$:
$$\vct{j}_Q = - \dunderline{\lambda} \nabla T$$
Recimo, da podamo $T$ na obeh ploščah.
$$\vct{j} = -\lambda\frac{\Delta T}{d}\hat{e}_x$$
$$j_x = \hat{e}_x\cdot\vct{j} = -\frac{\Delta T}{d}\left(\hat{e}_x\lambda\hat{e}_y\right) = -\lambda_{xx}\frac{\Delta T}{d}$$
$$j_y = -\frac{\Delta T}{d}\left(\hat{e}_y\lambda\hat{e}_x\right) = -\lambda_{yy}\frac{\Delta T}{d}$$
Lahko si izberemo tudi drugačen robni pogoj: Recimo, da tok teče samo pravokotno na ploskev, torej le vzdolž osi $x$:
$$j\hat{e}_x = -\lambda\nabla T = -\lambda\left(\frac{\Delta T_x}{d_x} \hat{e}_x + \frac{\Delta T_y}{d_y}\hat{e}_y\right)$$
$$\hat{e}_xj\hat{e}_x = j = -\left(\frac{\Delta T_x}{d_x} \lambda_{xx} + \frac{\Delta T_y}{d_y} \lambda_{xy}\right)$$
Podobno dobimo za $j_y$:
$$\hat{e}_yj\hat{e}_x = 0 = -\left(\frac{\Delta T_x}{d_x}\lambda_{xy} + \frac{\Delta T_y}{d_y}\lambda_{yy}\right)$$
Enačbi odštejemo, da dobimo:
$$\frac{\Delta T_y}{d_y} =- \frac{\Delta T_x}{d_x}\frac{\lambda_{xx}}{\lambda_{yy}}$$
$$j = -\frac{\Delta T_{x}}{d_x}\left(\lambda_{xx} - \frac{\lambda_{xy}^2}{\lambda_{yy}}\right) = \frac{\Delta T}{d_x} \left(\frac{\lambda_{xx}\lambda_{yy} - \lambda_xy^2}{\lambda_{yy}}\right)$$
Dobili smo ravno nasprotno vrednost determinante inverza:
$$\frac{\Delta T}{d_x} = -\frac{\lambda_{yy}}{\det \lambda}j$$
$$\nabla T = -\dunderline{\lambda^{-1}}\vct{j}$$
Če se postavimo v lastni sistem (lastna vektorja $\vct{n}, \vct{m}$), je naša naloga še lažja:
$$\begin{bmatrix}
    \lambda_\parallel^{-1} & 0 \\
    0 & \lambda_\perp^{-1}
\end{bmatrix} = \lambda_\parallel^{-1} \vct{n}\otimes\vct{n} + \lambda_{\perp}^{-1} \vct{m}\otimes\vct{m}$$
V lastni koordinatni sistem bomo prišli z rotacijo za nek kot $\varphi$:
$$\hat{e}_x \dunderline{\lambda}\hat{e}_x = \lambda_\parallel^{-1}\cos^2\varphi + \lambda_\perp^{-1}\sin^2\varphi$$
Iz tega dobimo originalno $\lambda$. Vidimo tudi, da je $\displaystyle{\frac{\lambda_{yy}}{\det\dunderline{\lambda}} = \frac{\lambda_\perp\cos^2\varphi + \lambda_\parallel\sin^2\varphi}{\lambda_\perp\lambda_\parallel}}$.
\paragraph{Laplaceov operator v anizotropni snovi.} Velja $\nabla\cdot\vct{j}=0$
$$\partial_i\lambda_{ij}\partial_jT=0$$
Ta enačba vsebuje mešane odvode. Definirali smo
$$T = \frac{C}{\sqrt{x'^2 + y'^2 + z'^2}} = \frac{C}{\sqrt{\frac{x^2}{\lambda_x} + \frac{y^2}{\lambda_y} + \frac{z^2}{\lambda_z}}}$$
Tedaj je $x_i' = T_{ij}^{-1}x_j$, torej je tudi $\displaystyle{\pd{}{x_j} = T_{ji}^{-1}x\pd{}{x_i}}$
$$\pd{}{x_i}\lambda_{ij}\pd{}{x_j} = \left(T_{ik}^{-1} \pd{}{x'_k}\lambda_{ij}T_{jl}^{-1}\pd{}{x'_l}\right) \equiv \pd{}{x'_k}\pd{}{x'_k}$$
Torej je $\displaystyle{\delta_{kl} = T^{-1}_{ik}\lambda_{ij}T^{-1}_{jl}}$ \\
Če je $T$ "koren" $\lambda$ (kar se za matrike da definirati), je v $x'$ bazi $\nabla\lambda\nabla = \nabla^2$
\paragraph{Tekoči kristali.} Snov opišemo s tako imenovanim direktorjem (vektorjem, ki opisuje (povprečno?) smer molekul v kristlu). Iščemo način, da ga določimo. \\
Najprej uvedemo tenzorski parameter:
$$Q_{ij} = \frac{S}{2} \left(n_in_j - \delta_{ij}\right)$$
Vidimo, da je $\det Q \neq 0$. $S$ predstavlja skalarni ureditveni parameter, ki je izotropen. \\[3mm]
Snov bo zavzela tako stanje, da bo prosta energija čim manjša. Prosta energija pa je sestavljena iz dveh delov, in sicer ureditvenega ter elastičnega dela: \\[3mm]
Ureditveni del opisuje enačba: $$F = \frac{1}{2}A\mathrm{tr}(Q^2) + \frac{1}{3}B\mathrm{tr}(Q^3) + \frac{1}{4}C\left(\mathrm{tr}(Q^2)\right)^2$$
Elastični del:
$$\frac{1}{2}L_1\pd{{Q}_{ij}}{x_k}\pd{Q_{ij}}{x_k} + \frac{1}{2}L_2\pd{Q_{ij}}{x_j}\pd{Q_{ik}}{x_k} + \frac{1}{2}L_3Q_{ij}\pd{Q_{kl}}{x_i}\pd{Q_{kl}}{x_j}$$
Da to minimiziramo, uporabimo smerni odvod, kar nam da $\vct{n}$.
\end{document}