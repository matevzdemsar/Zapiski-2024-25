\documentclass[a4paper]{article}
\usepackage{amsmath, amssymb, amsfonts}
\usepackage[margin=1in]{geometry}
\usepackage{graphicx}
\usepackage{tikz}
\usepackage{esint}
\setlength{\parindent}{0em}
\setlength{\parskip}{1ex}
\newcommand{\vct}[1]{\overrightarrow{#1}}
\newcommand{\pd}[2]{\frac{\partial {#1}}{\partial {#2}}}
\newcommand{\dd}[2]{\frac{\mathrm{d} {#1}}{\mathrm{d} {#2}}}
\newcommand{\C}{\mathbb{C}}
\newcommand{\R}{\mathbb{R}}
\newcommand{\Q}{\mathbb{Q}}
\newcommand{\Z}{\mathbb{Z}}
\newcommand{\N}{\mathbb{N}}
\newcommand{\fn}[3]{{#1}\colon {#2} \rightarrow {#3}}
\newcommand{\avg}[1]{\langle {#1} \rangle}
\newcommand{\Sum}[2][0]{\sum_{{#2} = {#1}}^{\infty}}
\newcommand{\Lim}[1]{\lim_{{#1} \rightarrow \infty}}
\newcommand{\Binom}[2]{\begin{pmatrix} {#1} \cr {#2} \end{pmatrix}}


\begin{document}
Kartezične koordinate:
$$\vct{r} = x\hat{e}_x + y\hat{e}_y + z\hat{e}_z = x^{i}\hat{e}_i$$
To ni čisto točno vektor, saj $\hat{e}_x$, $\hat{e}_y$ in $\hat{e}_z$ niso vektorji. Lokalne bazne vektorje dobimo z odvodom:
$$\vct{e_i} = \pd{\vct{r}}{x^i}$$
$$\vct{v} = v^i\hat{e}_i = v^i\partial_i$$
(Načeloma $\partial_i \vct{r}$) \\[4mm]
V cilindričnih koordinatah: $$\vct{r} = (r\cos\varphi, r\sin\varphi)$$
$$\vct{e_r} = (\cos\varphi, \sin\varphi)$$
$$\vct{e_\varphi} = (-r\sin\varphi, r\cos\varphi)$$
Infinitezimalni premik:
$$d\vct{r} = \pd{\vct{r}}{x^i}dx^i = \vct{e}_idx^i$$
Taka oznaka je pravilna v vseh koordinatnih sistemih.
$$|d\vct{r}|^2 = d\vct{r} \cdot d\vct{r} = dx^i\vct{e}_i\cdot\vct{e}_jdx^j = g_{ij}dx^idx^j$$
Pri tem je $g$ metrični tenzor. Dobimo ga s skalarnim množenjem baznih vektorjev. V cilindričnih koordinatah je na primer enak
$$g = \begin{bmatrix}
    1 & 0 \\ 0 & r^2
\end{bmatrix}$$
\paragraph{Gradient.} Obravnavamo skalarno polje $f$
$$df = \pd{f}{x^i}dx^i = (\nabla f)_idx^i$$
Mimogrede, tu se ponovno pokaže misel od prejšnjič, da so kontravariantni vektorji primerljivi s plastnicami.
\paragraph{Operator $\nabla$}. \\
Kartezčne koordinate: $\displaystyle{\nabla = \left(\pd{}{x}, \pd{}{y}, \pd{}{z}\right)}$ \\
Polarne koordinate: $\displaystyle{\nabla = \left(\pd{}{r}, \frac{1}{r}\pd{}{\varphi}\right)}$ \\
In tako naprej. Obstajajo tabele $\nabla$ v različnih koordinatnih sistemih, za poljuben sistem pa jih lahko tudi izpeljemo, če $\nabla$ v kartezičnih koordinatah zapišemo kot $\nabla = \hat{e}^i\partial_i$
in izrazimo $\hat{e}_i$ za iskani koordinatni sistem (npr. $\hat{e}_r = (\cos\varphi, \sin\varphi)$).
\paragraph{Menjava koordinat.} To ni nujno linearen proces. Recimo, da preslikamo $x^i \to u^j$.
$$\pd{f}{x^i} = \pd{f}{u^j}\pd{u^j}{x^i} = \pd{f}{u^j}T^i$$
S $T$ smo označili posplošeno linearno preslikavo - gre za matriko z elementi $T^{ij} = \pd{u^j}{x^i}$
\paragraph{Integral gradienta.} Primer: Vzgon
$$F_{v} = \varoiint pd\vct{S} = \varoiint \rho g h d\vct{S} = \rho g V = \iiint \rho gdV = \iiint \nabla pdV$$
V splošnem:
$$\varoiint fd\vct{S} = \iiint \nabla f dV$$
\paragraph{Usmerjeni odvod.} Imamo parametrizirano krivuljo $\vct{r}(t)$. Opazujemo odvod neke funkcije $f$ v smeri tangente na $\vct{r}$ (označimo s $t$).
$$\dd{f(\vct{r})}{t} = \pd{f}{x^i}\pd{x^i}{t} = (\nabla f)\vct{T}$$
Označimo $\vct{t}\cdot\vct{\nabla}$ - operator odvoda v smeri $\vct{t}$
$$(\vct{t} \cdot \vct{\nabla})f = \frac{f(\vct{r} + h\vct{t}) - f(\vct{r})}{h}$$
To velja tudi, če je $f$ vektorska funkcija. \\[4mm]
Nekaj primerov: \\
Advekcijski člen za pretok temperature v toku tekočine:
$$\dd{T}{t} = \pd{T}{t} + (\vct{v}\cdot\vct{\nabla})T$$
Advekcijski člen za gibanje tekočin:
$$\rho \dd{\vct{v}}{t} = -\nabla p + \text{druge sile}$$
$$\dd{\vct{v}}{t} = \pd{\vct{v}}{t} + (\vct{v}\cdot\nabla)\vct{v}$$
Sile na dipol v nehomogenem polju:
$$F_m = (\vct{p_m}\cdot\nabla)\vct{B} = (p_j\partial_j)B_i = p_j\partial_iB_j = p_j(\partial_iB_j) = \vct{p}(\vct{\nabla}\cdot\vct{B})$$
$$-\vct{p_m}\vct{B} = W_m$$
Sledi:
$$\vct{F} = -\nabla W_m = \nabla_i(p_jB_j) = p_j\partial_iB_j$$
\paragraph{Gradient vektorja.} Za primer vzemimo $\vct{B}$
$$\partial_iB^j = \begin{bmatrix}
    \partial_xB_x & \partial_xB_y & \partial_xB_z \\
    \partial_yB_x & \partial_yB_y & \partial_yB_z \\
    \partial_zB_x & \partial_zB_y & \partial_zB_z \\
\end{bmatrix}$$
\paragraph{Divergenca.} Videli smo jo že v Gaussovem izreku, na splošno pa pri pretokih vektorjev skozi ploskve. \\
Najbolj nas bo zanimal pretok skozi zaključeno ploskev: \\
$\displaystyle{\varoiint \vct{v}d\vct{S}}$ v pošljimo v limito $V \to 0$
\begin{align*}
    \varoiint \vct{v} d\vct{S} & = v_x(x + dx, y, z)\,dy\,dz - v_x(x, y, z)\,dy\,dz \\
    & + v_y(x, y + dy, z)\,dx\,dz - v_x(x, y, z)\,dx\,dz \\
    & + v_z(x, y, z + dz)\,dx\,dy - v_x(x, y, z)\,dx\,dy \\
    & = \left(\pd{v_x}{x} + \pd{v_y}{y} + \pd{v_z}{z}\right)\,dx\,dy\,dz
\end{align*}
Izraz $\displaystyle{\left(\pd{v_x}{x} + \pd{v_y}{y} + \pd{v_z}{z}\right)}$ imenujemo divergenca (označimo $\vct{\nabla}$ ali $div$).
$$\vct{\nabla} \vct{v} = div\vct{v} = \partial_iv^i$$
Tudi divergenca ima v različnih koordinatnih sistemih različne oblike, ki so ravno tako tabelirane,
da pa se jih tudi izpeljati.
\paragraph{Rotor.} Rotor je uporaben pri integralih po zaključenih krivuljah. Na primer:
\begin{align*}
    \vct{v} \cdot d\vct{r} & = v_y(x + dx, y, z)\,dy - v_x(x, y+dy, z)\,dx \\
    & ~~- v_y(x, y, z)\,dy + v_x(x, y, z)\,dx \\
    & = \pd{v_y}{x}\,dx\,dy - \pd{v_x}{y}\,dx\,dy \\
    & = (\vct{\nabla} \times \vct{v})\,d\vct{S}_{xy}
\end{align*}
Uvedli smo rotor:
$$(\vct\nabla \times \vct v) = \varepsilon_{ijk}\partial_jv_k$$
\paragraph{Kombinirani diferencialni operatorji.}
$$\nabla \times (\nabla f) = 0$$
$$\nabla \cdot (\nabla \times \vct v) = 0$$
V elektromagnetizmu:
$$\oint \vct H d \vct e = \iint \left(\dot{\vct D} + \vct j\right)d\vct S$$
Kajti: $$\vct\nabla\times\vct H = \dot{\vct{D}} + \vct{j}$$
\paragraph{Laplaceov operator.} $\displaystyle{\nabla^2f = \vct\nabla (\nabla f) = \partial_i\partial_if}$ \\
V kartezičnih koordinatah je enak $\displaystyle{\pd{^2}{x^2} + \pd{^2}{y^2} + \pd{^2}{z^2}}$ \\
V fiziki je izjemno uporaben, npr. pri difuzijski enačbi:
$$D\nabla^2T = \pd{T}{t}$$
Poissonova enačba:
$$\nabla^2U = -\frac{\rho_e}{\varepsilon_0}$$
Laplaceova enačba:
$$\nabla^2U = 0$$
(Poissonova, če nimamo izvorov) \\[4mm]
Valovna enačba:
$$c^2\nabla^2y = \pd{^2y}{t^2}$$
V cilinddričnih koordinatah: \\
$\displaystyle{\nabla^2f = \frac{1}{r}\pd{}{r}\left(\pd{f}{r}\right) + \frac{1}{r^2}\pd{^2f}{\varphi^2}}$ \\
V sferičnih koordinatah: \\
$\displaystyle{\nabla^2f = \frac{1}{r^2}\pd{}{r}\left(r^2\pd{f}{r}\right) + \frac{1}{r^2}\left[\frac{1}{\sin\vartheta}\pd{}{\vartheta}\left(\sin\vartheta\pd{f}{\vartheta}\right)+ \frac{1}{\sin^2\varphi}\pd{^2f}{\varphi^2}\right]}$ \\
\paragraph{Vektorski Laplaceov operator.} Definiran je enako, se ga pa običajno da napisati drugače.
$$\nabla^2\vct{v} = div\,grad\,\vct{v} = \partial_j\partial_jv_i$$
Vendar:
$$\vct\nabla \times (\vct\nabla \times \vct{v}) = \vct\nabla(\vct\nabla\vct{v}) - \nabla^2\vct{v}$$
To je običajno lažje izračunati.
\paragraph{Zunanji odvod in diferencialne forme.} Skalarno funkcijo spremenimo v vektorsko z gradientom.
$$\int_{r} v_idx^i[\vct{r}(t)] = \int_{r}\vct{v}d\vct{r} = \int\vct{v}\pd{\vct{r}}{t}dt$$
Vektorje, povezane s ploskvami (tj. jih moramo integrirato po ploskvah), včasih imenujemo tudi psevdovektorji ali bivektorji. Do njih pridemo tako, da na vektorjih izvedemo rotor. \\[4mm]
Skalarje, povezane s prostornino (razne gostote), imenujemo tudi psevdoskalarji. Do njih pridemo tako, da na psevdoskalarjih uporabimo divergenco. Lahko jim rečemo tudi trivektor, gre namreč za mešani produkt $\vct\alpha\cdot(\vct\beta\times\vct\gamma)$. \\[4mm]
Zunanji odvod definiramo kot $df = \nabla_ifdx^i$
$$d(v_idx^i) = \pd{v_i}{x^j}dx^i\wedge dx^j$$
Z $\wedge$ smo označili zunanji produkt vektorjev (gre za nekakšen vektorski produkt).
Ker nam zunanji produkt vrne nekaj podobnega rotorju, velja $d(df) = 0$
\paragraph{Stokesov izrek:} Zunanji odvod nam omogoči, da v integralu izrazimo dejstvo, da integriramo po robu množice.
$$\oint_\partial M\omega = \int d\omega$$
Posledično je naša ugotovitev $d(df) = 0$ ekvivalentna trditvi $\partial(\partial M) = 0$, ali drugače: Rob množice nima roba.
\end{document}