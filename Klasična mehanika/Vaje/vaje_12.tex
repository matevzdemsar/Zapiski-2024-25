\documentclass[a4paper]{article}
\usepackage{amsmath, amssymb, amsfonts}
\usepackage[margin=1in]{geometry}
\usepackage{graphicx}
\usepackage{tikz}
\usepackage{esint}
\setlength{\parindent}{0em}
\setlength{\parskip}{1ex}
\newcommand{\vct}[1]{\overrightarrow{#1}}
\newcommand{\dif}{\mathrm{d}}
\newcommand{\pd}[2]{\frac{\partial {#1}}{\partial {#2}}}
\newcommand{\dd}[2]{\frac{\mathrm{d} {#1}}{\mathrm{d} {#2}}}
\newcommand{\C}{\mathbb{C}}
\newcommand{\R}{\mathbb{R}}
\newcommand{\Q}{\mathbb{Q}}
\newcommand{\Z}{\mathbb{Z}}
\newcommand{\N}{\mathbb{N}}
\newcommand{\fn}[3]{{#1}\colon {#2} \rightarrow {#3}}
\newcommand{\avg}[1]{\langle {#1} \rangle}
\newcommand{\Sum}[2][0]{\sum_{{#2} = {#1}}^{\infty}}
\newcommand{\Lim}[1]{\lim_{{#1} \rightarrow \infty}}
\newcommand{\Binom}[2]{\begin{pmatrix} {#1} \cr {#2} \end{pmatrix}}
\newcommand{\duline}[1]{\underline{\underline{#1}}}

\begin{document}
\paragraph{Eulerjevi koti.} \text{} \\
$\varphi$ je kot precesije:
$$R_{\varphi} = \begin{bmatrix}
    \cos\varphi & \sin\varphi & 0 \\
    -\sin\varphi & \cos\varphi & 0 \\
    0 & 0 & 1
\end{bmatrix}$$
$\theta$ je kot nutacije:
$$R_\theta = \begin{bmatrix}
    1 & 0 & 0 \\
    0 & \cos\theta & \sin\theta \\
    0 & -\sin\theta & \cos\theta
\end{bmatrix}$$
$\psi$ je kot zasuka:
$$R_\psi = \begin{bmatrix}
    \cos\psi & \sin\psi & 0 \\
    -\sin\psi & \cos\psi & 0 \\
    0 & 0 & 1
\end{bmatrix}$$
Vrtenje telesa tedaj opišemo kot
$$\vct{\omega} = \dot\varphi \widehat{k} + \dot\theta \widehat{x}'' + \dot\psi \widehat{k}' = \omega_x'\widehat{x}' + \omega_y'\widehat{y}' + \omega_z'\widehat{k}'$$
Izraziti moramo bazne vektorje ($\widehat{k}$, $\widehat{x}''$):
$$k = R_\psi R_\theta R_\varphi\begin{bmatrix}
    0 \\ 0 \\ 1
\end{bmatrix} = ... = \begin{bmatrix}
    \sin\psi\sin\theta \\
    \cos\psi\sin\theta \\
    \cos\theta
\end{bmatrix}$$
$$\widehat{x}'' = R_\psi R_\theta \begin{bmatrix}
    1 \\ 0 \\ 0
\end{bmatrix} = ... = \begin{bmatrix}
    \cos\psi \\ -\sin\psi \\ 0
\end{bmatrix}$$
$$\omega_x' = \dot\varphi\sin\psi\sin\theta + \dot\theta\cos\psi$$
$$\omega_y' = \dot\varphi\cos\psi\sin\theta - \dot\theta\sin\psi$$
$$\omega_z' = \dot\varphi\cos\theta + \dot\psi$$
\paragraph{Kinetična energija simetrične vrtavke.} $$T = \frac{1}{2}\vct{\omega}\duline{J}\vct{\omega}$$
$$\duline{J} = \begin{bmatrix}
    J_\parallel && \\
    & J_\perp & \\
    && J_\perp \\
\end{bmatrix}$$
\paragraph{Stožec na nagnjeni podlagi.} $$J_x = J_y = \int (y^2 + z^2)\,\dif m$$
Uporabimo cilindrične koordinate:
$$x = r\cos\varphi,~~y = r\sin\varphi,~~\dif V = r\,\dif r\,\dif\varphi\,\dif z$$
$$J_x = \int_{0}^{2\pi}\int_{0}^{R}\int_{0}^{h} \rho \left(r^2\sin^2\varphi + z^2\right)r\,\dif r\,\dif\varphi\,\dif z$$

\end{document}