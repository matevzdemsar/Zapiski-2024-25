\documentclass[a4paper]{article}
\usepackage{amsmath, amssymb, amsfonts}
\usepackage[margin=1in]{geometry}
\usepackage{graphicx}
\usepackage{tikz}
\usepackage{esint}
\setlength{\parindent}{0em}
\setlength{\parskip}{1ex}
\newcommand{\vct}[1]{\overrightarrow{#1}}
\newcommand{\dif}{\mathrm{d}}
\newcommand{\pd}[2]{\frac{\partial {#1}}{\partial {#2}}}
\newcommand{\dd}[2]{\frac{\mathrm{d} {#1}}{\mathrm{d} {#2}}}
\newcommand{\C}{\mathbb{C}}
\newcommand{\R}{\mathbb{R}}
\newcommand{\Q}{\mathbb{Q}}
\newcommand{\Z}{\mathbb{Z}}
\newcommand{\N}{\mathbb{N}}
\newcommand{\fn}[3]{{#1}\colon {#2} \rightarrow {#3}}
\newcommand{\avg}[1]{\langle {#1} \rangle}
\newcommand{\Sum}[2][0]{\sum_{{#2} = {#1}}^{\infty}}
\newcommand{\Lim}[1]{\lim_{{#1} \rightarrow \infty}}
\newcommand{\Binom}[2]{\begin{pmatrix} {#1} \cr {#2} \end{pmatrix}}

\begin{document}
\paragraph{Orbite v potencialu $1/r^2$} \text{} \\
$$H = \frac{m\dot{r}^2}{2} + V_{ef}$$
$$V_{ef} = \frac{p_\varphi^2}{2mr^2} - \frac{\kappa}{r^2} = \frac{p_\varphi^2}{2m}\left(1 - \frac{\kappa 2m}{p_\varphi^2}\right)\frac{1}{r^2}$$
Označimo $\displaystyle{\gamma = \frac{\kappa 2m}{p_\varphi^2}}$. Naša rešitev bo odvisna od $\gamma$: Če je $\gamma < 1$, so 
orbite neomejene in opazovano telo bo odletelo stran. Če je $\gamma > 1$, so orbite omejene, toda potencial gre v centru kroženja proti neskončno,
zato bo Petja padla v center. Če je $\gamma = 1$, pa je $V_{ef} = 0$, kar pomeni kroženje.
$$H = \frac{m\dot{r}^2}{2} + \frac{p_\varphi^2}{2m}(1 - \gamma)\frac{1}{r^2}$$
Uvedemo $u = 1/r$, $\displaystyle{\dot{r} = -\frac{1}{u^2}\dd{u}{t}} = -\frac{1}{u^2}\dd{u}{\varphi}\dd{\varphi}{t} = -\frac{p_\varphi}{mr^2u^2}\dd{u}{\varphi} = -\frac{p_\varphi}{m}\dd{u}{\varphi}$.
$$H = \frac{m}{2}\frac{p_\varphi^2u'^2}{m^2} + \frac{p_\varphi^2}{2m}\left(1 - \gamma\right)^2u^2$$
Da dobimo enačbo gibanja, še na obeh straneh odvajamo po $\varphi$:
$$0 = \frac{p_\varphi^2}{m}u'\left(u'' + (1+\gamma)u\right)$$
Enačba ima dve rešitvi: $u' = 0$ in $u'' + (1 - \gamma)u = 0$. Prva rešitev je kroženje (saj je tedaj radij konstanten), druga rešitev pa je diferencialna enačba:
$$u'' + (1 - \gamma)u = 0$$
Tu se pokaže odvisnost gibanja od $\gamma$: \\[3mm]
Če je $\gamma < 1$, dobimo za rešitev kosinusno funkcijo:
$$r = \frac{1}{C\cos\left(\sqrt{1-\gamma}(\varphi - \varphi_0)\right)}$$
Če je $\gamma = 1$:
$$u'' = 0 \Rightarrow r = \frac{1}{A\varphi + B}$$
Kar je spirala: $\displaystyle{\Lim{\varphi}r = 0}$. Če je hkrati $u' = 0$ pa gre kot prej za krožnico. \\[3mm]
Če je $\gamma > 1$, dobimo enačbo oblike
$$u'' - \alpha u = 0,~\alpha > 0$$
$$u = Be^{\sqrt{\alpha}\varphi} + Ce^{\sqrt{\alpha}\varphi}$$
$$r = \frac{1}{u}$$
Če je $B \neq 0$ in $C \neq 0$, $r(\varphi)$, je $r$ omejen. Sicer pač ne.
\end{document}