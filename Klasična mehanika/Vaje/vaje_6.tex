\documentclass[a4paper]{article}
\usepackage{amsmath, amssymb, amsfonts}
\usepackage[margin=1in]{geometry}
\usepackage{graphicx}
\usepackage{tikz}
\setlength{\parindent}{0em}
\setlength{\parskip}{1ex}
\newcommand{\vct}[1]{\overrightarrow{#1}}
\newcommand{\pd}[2]{\frac{\partial {#1}}{\partial {#2}}}
\newcommand{\dd}[2]{\frac{\mathrm{d} {#1}}{\mathrm{d} {#2}}}
\newcommand{\C}{\mathbb{C}}
\newcommand{\R}{\mathbb{R}}
\newcommand{\Q}{\mathbb{Q}}
\newcommand{\Z}{\mathbb{Z}}
\newcommand{\N}{\mathbb{N}}
\newcommand{\fn}[3]{{#1}\colon {#2} \rightarrow {#3}}
\newcommand{\avg}[1]{\langle {#1} \rangle}
\newcommand{\Sum}[2][0]{\sum_{{#2} = {#1}}^{\infty}}
\newcommand{\Lim}[1]{\lim_{{#1} \rightarrow \infty}}
\newcommand{\Binom}[2]{\begin{pmatrix} {#1} \cr {#2} \end{pmatrix}}


\begin{document}
Konstante gibanja lahko pogosto uganemo iz Lagrangeove funkcije, npr.:
$$\pd{L}{q_i} = 0$$
Tedaj $q_i$ imenujemo ciklična koordinata. Iz Euler-Lagrangeove enačbe dobimo zahtevo
$$\dd{}{t}\left(\pd{L}{\dot{q}_i}\right) = 0 \rightarrow \pd{L}{\dot{q}_i} = \text{konst.}$$
Količini $\displaystyle{\pd{L}{\dot{q}_i}}$ rečemo tudi posplošeni impulz ali $p_i$. \\[4mm]

Označimo lahko tudi energijsko funkcijo $\displaystyle{H = \sum_{i=1}^{N} \pd{L}{\dot{q}_i}}\dot{q}_i - L$ \\
Če $L$ ni odvisna od časa, je $H$ konstantna, torej:
$$\pd{L}{t} = 0 \Rightarrow \dd{H}{t} = 0$$
\paragraph{1. naloga:} Matematično nihalo na premičnem pritrdišču.
\begin{figure}[h!]
    \centering
    \begin{tikzpicture}
        \draw[dotted] (-5, 0) -- (5, 0);
        \filldraw (-0.5, 0) circle (0.1);
        \draw (-0.5, 0) -- (0.5, -3);
        \filldraw (0.5, -3) circle (0.1);
        \draw[thin, ->] (-0.5, 0) -- (0.5, 0); 
        \draw[thin, ->] (-0.5, 0) -- (-0.5, 1);
        \node (X) at (0.7, -0.2) {$x$};
        \node (Y) at (-0.6, 1.2) {$y$}; 
        \node (A) at (0.5, 0.3) {$m_1, x_1, y_1$};
        \node (B) at (0.5, -3.3) {$m_2, x_2, y_2$};
    \end{tikzpicture}
\end{figure}
\newline
Izberemo sledeče vezi:
$$y_1 = z_1 = 0$$
$$l^2 = (x_2 - x_1)^2 + y_2^2$$
\\
$$x_2 = l\sin\varphi + x_1 \rightarrow \dot{x}_2 = l\cos\varphi \,\dot{\varphi} + \dot{x}_1$$
$$y_2 = -l\cos\varphi \rightarrow \dot{y}_2 = l\sin\varphi \,\dot{\varphi}$$
$$T = \frac{1}{2}m_1\dot{x}_1^2 + \frac{1}{2}m_2(\dot{x}_2^2 + \dot{y}_2^2) = \frac{1}{2}m_1\dot{x}_1^2 + \frac{1}{2}m_2\left(l^2\dot{\varphi}^2 + 2l\dot{\varphi}\dot{x}_1\cos\varphi + \dot{x}_1^2\right)$$
$$V = -m_2gl\cos\varphi$$
\\
$$L = T-V = \frac{1}{2}m_1\dot{x}_1 + \frac{1}{2}m_2\left(l^2\dot{\varphi}^2 + 2l\dot{\varphi}\dot{x}_1\cos\varphi + \dot{x}_1^2\right) + m_2gl\cos\varphi$$
\\
$$x_1:~\dd{}{t}\left(\pd{L}{\dot{x}_1}\right) = \pd{L}{x_1}$$
$$(m_1 + m_2)\ddot{x}_1 + m_2l\left(\ddot{\varphi}\cos\varphi - \dot{\varphi}^2\sin\varphi\right) = 0$$
\newpage
$$\varphi:~\dd{}{t} \left(\pd{L}{\dot{\varphi}}\right) = \pd{L}{\varphi}$$
$$l\ddot{\varphi} + \ddot{x}_1\cos\varphi = -g\sin\varphi$$
Vstavimo izraz za $\ddot{x}_1$ v enačbo, pridobljeno iz $\varphi$, in dobimo:
$$l\ddot{\varphi} + \frac{m_2l}{m_1 + m_2}\left(\varphi^2\sin\varphi\cos\varphi - \ddot{\varphi}\cos^2\varphi\right) = -g\sin\varphi$$
Pri majhnih odmikih lahko stvar poenostavimo: $$l\ddot{\varphi}\left( 1 - \frac{m_2}{m_1 + m_2} \right) = l\ddot{\varphi} \frac{m_1}{m_1 + m_2} = -g\varphi$$
Dobimo nihanje s frekvenco $$\omega^2 = \frac{g}{l}\left(\frac{m_1 + m_2}{m_1}\right)$$
\paragraph{2. naloga:} Dve uteži, ena izmed katerih je na plošči.
\begin{figure}[h!]
    \centering
    \begin{tikzpicture}
        \draw (0, 0) -- (5, 0);
        \draw (1, 2) -- (6, 2);
        \draw (5, 0) -- (6, 2);
        \draw (1, 2) -- (0, 0);
        \draw[thin, ->] (3, 1) -- (3.75, 1) node[right] {$x$};
        \draw[thin, ->] (3, 1) -- (3.33, 1.67) node[right] {$y$};
        \draw[thin, ->] (3, 1) -- (3, 1.75) node[left] {$z$};
        \filldraw (4, 1.5) circle (0.05) node[right] {$m_1$};
        \draw (4, 1.5) -- (3, 1);
        \draw[dotted] (3, 1) -- (3, 0);
        \draw (3, 0) -- (3, -1);
        \filldraw (3, -1) circle (0.05) node[right] {$m_2$};
    \end{tikzpicture}
\end{figure}
\newline
Pojdimo v cilindrične koordinate:
$$m_1:~x_1, y_1 \rightarrow \varphi, r$$
$$z_2 = l-r$$
\\
Splošna formula za kinetično energijo v polarnih koordinatah:
$$T = \frac{1}{2}m\left(\dot{r}^2 + r^2\dot{\varphi}^2\right)$$
\\
V našem primeru moramo prišteti še kinetično energijo $m_2$:
$$T = \frac{1}{2}m_1\left(\dot{r}^2 + r^2\dot{\varphi}^2\right) + \frac{1}{2}m_2\dot{r}^2$$
$$V = gm_2(r-l)$$
$$L = \frac{1}{2}m_1(\dot{r}^2 + r^2\dot{\varphi}^2) + \frac{1}{2}m_2\dot{r}^2 - gm_2(r-l)$$
Z uporabo Euler-Lagrangeove enačbe dobimo
$$(m_1 + m_2)r^2 = m_1r\dot{\varphi}^2 - gm_2$$
$$\dd{}{t} \left(\pd{L}{\dot{\varphi}}\right) = 0 \Rightarrow p_\varphi = m_1\dot{\varphi}r^2$$
Recimo, da je naše začetno stanje $r = r_0$, $\dot{r} = \ddot{r} = 0$
$$p_\varphi^2 = r_0^3gm_1m_2$$
$$\dot\varphi = \sqrt{\frac{gm_2}{r_0m_1}}$$
Če ima začetno stanje nekolikšen odmik, npr. $r = r_0 + \delta$:
$$(m_1 + m_2)\ddot{\delta} = \frac{p_\varphi^2}{m_1(r_0 + \delta)^3} - gm_2$$
$$= \frac{p_\varphi^2}{m_1r_0^3\left(1 + \frac{\delta}{r_0}\right)^3} - gm_2$$
$$\approx \frac{p_\varphi^2}{m_1r_0^3}\left(1 - 3\frac{\delta}{r_0}\right) - gm_2$$
V začetnem stanju je veljalo $\displaystyle{\frac{p_\varphi^2}{m_1r_0^3}-gm_2 = 0}$. Ko to upoštevamo, dobimo
$$(m_1 + m_2)\ddot{\delta} = -\frac{3p_\varphi^2}{m_1r_0^4}\delta$$
Dobimo enačbo nihanja s frekvenco
$$\omega = \sqrt{\frac{3p_\varphi^2}{m_1(m_1 + m_2)r_0^4}} = \sqrt{\frac{3r_0^3gm_1m_2}{m_1(m_1 + m_2)r_0^4}} = \sqrt{\frac{3gm_2}{(m_1 + m_2)r_0}}$$

\paragraph{3. naloga:} Gibanje po krogli brez trenja. Zanima nas, kdaj se odlepi. \\
Uporabimo generalizirane koordinate $r, \vartheta$.
$$T = \frac{1}{2}m\left(r^2\dot{\vartheta}^2 + \dot{r}^2\right)$$
$$V = mgr\cos\vartheta$$
$$L = T - V = \frac{1}{2}m(\dot{r}^2 + r^2\dot{\vartheta}^2) - mgr\cos\vartheta$$
Pri tem nismo upoštevali sile podlage:
$$Q_r = \vct{F}_p\cdot\pd{\vct{r}}{r} = F_p\hat{e}_r\cdot\hat{e}_r = F_p$$
$$Q_\vartheta = F_p \hat{e_r}\cdot\hat{e_\vartheta} = 0 \text{ (vsaj za polarne koordinate)}$$
Zapišemo E-L enačbi za $r$ in $\vartheta$. Mimogrede: Na neki točki bo $F_p$ postala 0, ker se bo klada odlepila od podlage.
$$\dd{}{t} \left(m\dot{r}\right) = m\ddot{r} = mr\dot{\vartheta}^2 + F_p - mg\cos\vartheta$$
$$\dd{}{t}\left(\pd{L}{\dot{\vartheta}}\right) = \pd{L}{\vartheta} + 0$$
$$2\dot{r}\vartheta + r^2\ddot{\vartheta} = gr\sin\vartheta$$
Trenutek, ko se klada odlepi, poiščemo z ohranitvijo energije:
$$t=0:~E = mgR$$
$$mgR = \frac{1}{2}m(\dot{r}^2 + r^2\dot{\vartheta}^2) + mgr\cos\vartheta$$
Dokler se ne odcepi, je $r = R$ in $\dot{r} = \ddot{r} = 0$
$$R\dot{\vartheta}^2 = 2g(1-\cos\vartheta)$$
Izraz vstavimo v enačbo, pridobljeno iz E-L enačbe, in dobimo:
$$2mg(1-\cos\vartheta) - mg\cos\vartheta + F_p = 0$$
$$F_p = mg(3\cos\vartheta - 2)$$
Želimo $F_p = 0$, torej mora biti $$\cos\vartheta = \frac{2}{3} \Rightarrow \vartheta \approx 48^\circ$$
\end{document}