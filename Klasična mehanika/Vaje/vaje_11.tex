\documentclass[a4paper]{article}
\usepackage{amsmath, amssymb, amsfonts}
\usepackage[margin=1in]{geometry}
\usepackage{graphicx}
\usepackage{tikz}
\usepackage{esint}
\setlength{\parindent}{0em}
\setlength{\parskip}{1ex}
\newcommand{\vct}[1]{\overrightarrow{#1}}
\newcommand{\dif}{\mathrm{d}}
\newcommand{\pd}[2]{\frac{\partial {#1}}{\partial {#2}}}
\newcommand{\dd}[2]{\frac{\mathrm{d} {#1}}{\mathrm{d} {#2}}}
\newcommand{\C}{\mathbb{C}}
\newcommand{\R}{\mathbb{R}}
\newcommand{\Q}{\mathbb{Q}}
\newcommand{\Z}{\mathbb{Z}}
\newcommand{\N}{\mathbb{N}}
\newcommand{\fn}[3]{{#1}\colon {#2} \rightarrow {#3}}
\newcommand{\avg}[1]{\langle {#1} \rangle}
\newcommand{\Sum}[2][0]{\sum_{{#2} = {#1}}^{\infty}}
\newcommand{\Lim}[1]{\lim_{{#1} \rightarrow \infty}}
\newcommand{\Binom}[2]{\begin{pmatrix} {#1} \cr {#2} \end{pmatrix}}
\newcommand{\duline}[1]{\underline{\underline{#1}}}

\begin{document}
\paragraph{Dinamika togega telesa.}
$$\duline{J} = \sum_i m_i(Ir^2 - \vct{r}\otimes\vct{r})$$
$$\vct{r}\otimes\vct{r} = \begin{pmatrix}
    x^2 & xy & xz \\
    yx & y^2 & yz \\
    zx & zy & z^2
\end{pmatrix}$$
Vsoto ocenimo z integralom.
$$\duline{J} = \int \dif m \begin{pmatrix}
    x^2 & xy & xz \\
    yx & y^2 & yz \\
    zx & zy & z^2
\end{pmatrix}$$
$$T = \frac{1}{2}\vct{\omega}^T\duline{J}\vct{\omega}$$
V lastnem sistemu velja $$\duline{J} = \begin{pmatrix}
    J_x & 0 & 0 \\
    0 & J_y & 0 \\
    0 & 0 & J_z
\end{pmatrix}$$
V tem sistemu je
$$\vct{L} = J_x \omega_x'\hat{i'} + J_y \omega_y'\hat{j'} + J_z \omega_z'\hat{k'}$$
$$\\dot{vct{L}} =
J_x \dot{\omega}_x'\hat{i'} + J_x\dot{\omega}_x'(\vct{\omega} \times \hat{i'}) +
J_y \dot{\omega}_y'\hat{j'} + J_y\dot{\omega}_y'(\vct{\omega} \times \hat{j'}) +
J_z \dot{\omega}_z'\hat{k'} + J_z\dot{\omega}_z'(\vct{\omega} \times \hat{k'})$$
Tako dobimo Eulerjeve enačbe:
$$M_x' = J_x\dot\omega_x' - (J_y-J_z)\omega_y'\omega_z'$$
$$M_y' = J_y\dot\omega_x' - (J_z-J_x)\omega_z'\omega_x'$$
$$M_z' = J_z\dot\omega_x' - (J_x-J_z)\omega_x'\omega_y'$$
\paragraph{Prosta simetrična vrtavka.} $J_x = J_y = J \neq J_z$, $\vct{M}=0$
Uporabimo Eulerjeve enačbe:
$$0 = J\dot{\omega}_x' + (J - J_z)\omega_y'\omega_z'$$
$$0 = J\dot{\omega}_y' + (J_z - J)\omega_z'\omega_x'$$
$$0 = J_z\dot\omega_z'$$
Iz tretje enačbe dobimo $\omega_z' = \omega_0$. Uvedemo novo spremenljivko $\xi = \omega_x' + i\omega_y'$, $\dot\xi = \dot\omega_x' + i\dot\omega_y'$. Seštejemo prvi dve enačbi:
$$J\dot\xi + J\omega_0i\xi - J_z\omega_0\xi = 0$$
$$\frac{\dif\xi}{\xi} = -i\omega_0\left(1 - \frac{J_z}{J}\dif t\right)$$
$$\xi = Ce^{-i\omega_z\left(1 - \frac{J'}{J}\right)t}$$
Označimo $\displaystyle{\Omega_p = \omega_z \left(1 - \frac{J_z}{J}\right)}$
$$\omega_x' = |C|\cos(\Omega_p t)$$
$$\omega_y' = |C|\sin(\Omega_p t)$$
$$\omega_z' = \omega_0$$
\paragraph{Navoj v ležajih pravokotne plošče.} Pravokotnik naj ima stranici $a$ in $b$.
$$J_x = \int (y^2 + z^2)\dif m = \iint y^2 \frac{m}{ab}\,\dif x\,\dif y$$
$$=\int_{-b/2}^{b/2}\int_{-a/2}^{a/2} y^2 \frac{m}{ab}\,\dif x\,\dif y = ... = \frac{mb^2}{12}$$
$$J_y = ... = \frac{ma^2}{12}$$
$$J_z = \frac{m}{12}(b^2 + a^2)$$
$$J_{xy} \int -xy\,\dif m = 0$$
$$\text{(Liha funkcija na simetričnem intervalu)}$$
$$\vct{\omega} = \omega_0 \frac{a}{\sqrt{a^2 + b^2}}i' + \omega_0\frac{b}{\sqrt{a^2 + b^2}}j'$$
$$M_x' = M_y' = 0$$
$$M_z' = \frac{mab\omega_0^2}{12(a^2 + b^2)}(a^2 - b^2)$$
Vstavimo v Eulerjeve enačbe.
V posebnem primeru, ko je $a=b$, lahko izračunamo silo $F$, kajti $M_z = Fr$, $\displaystyle{r = \frac{a}{2\sqrt{a^2+b^2}}}$.
Tedaj je $$F = \frac{mb\,\omega_0^2}{6\sqrt{a^2 + b^2}}$$
\end{document}