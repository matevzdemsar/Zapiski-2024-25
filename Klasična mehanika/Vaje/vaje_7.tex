\documentclass[a4paper]{article}
\usepackage{amsmath, amssymb, amsfonts}
\usepackage[margin=1in]{geometry}
\usepackage{graphicx}
\usepackage{tikz}
\setlength{\parindent}{0em}
\setlength{\parskip}{1ex}
\newcommand{\vct}[1]{\overrightarrow{#1}}
\newcommand{\pd}[2]{\frac{\partial {#1}}{\partial {#2}}}
\newcommand{\dd}[2]{\frac{\mathrm{d} {#1}}{\mathrm{d} {#2}}}
\newcommand{\C}{\mathbb{C}}
\newcommand{\R}{\mathbb{R}}
\newcommand{\Q}{\mathbb{Q}}
\newcommand{\Z}{\mathbb{Z}}
\newcommand{\N}{\mathbb{N}}
\newcommand{\fn}[3]{{#1}\colon {#2} \rightarrow {#3}}
\newcommand{\avg}[1]{\langle {#1} \rangle}
\newcommand{\Sum}[2][0]{\sum_{{#2} = {#1}}^{\infty}}
\newcommand{\Lim}[1]{\lim_{{#1} \rightarrow \infty}}
\newcommand{\Binom}[2]{\begin{pmatrix} {#1} \cr {#2} \end{pmatrix}}


\begin{document}
\paragraph{1. naloga:} Valj s spiralnim vodilom. Okoli valja je navito vodilo, karakterizirano s konstantno $p = \dd{z}{\varphi}$.
Po tem vodilu spustimo maso m, razen tega se valj lahko prosto vrti okoli svoje osi. \\[4mm]
Uvedemo cilindrične koordinate:
$$x = R\cos\varphi$$
$$y = R\sin\varphi$$
$$z = z$$
Imamo vezi $r=R$ in $z=p(\varphi - \Phi)$. Zasuk valja označimo s $\Phi$.
$$T = \frac{m}{2}\left(\dot{r}^2 + r^2\dot{\varphi}^2+\dot{z}^2\right) + \frac{1}{2}J\dot{\Phi}^2$$
$$J = \frac{1}{2}MR^2$$
$$T = \frac{m}{2}\left(R^2\dot{\varphi}^2 + p^2(\dot{\varphi} - \dot{\Phi})^2\right) + \frac{MR^2}{4}\dot{\Phi}^2$$
$$V = mgp(\varphi - \Phi)$$
$$L = T - V = \frac{m}{2}\left(R^2\dot{\varphi}^2 + p^2(\dot{\varphi} - \dot\Phi^2) + \frac{Mr^2}{r}\dot\Phi^2\right) - mgp(\varphi - \Phi)$$
E-L enačba: imamo dve generalizirani koordinati, in sicer $\varphi$ in $\Phi$. Tedaj velja:
$$\dd{}{t}\pd{L}{\dot{\varphi}} = \pd{L}{\varphi}$$
$$\dd{}{t}\pd{L}{\dot{\Phi}} = \pd{L}{\Phi}$$
Najprej obravnavajmo $\varphi$, nato $\Phi$:
$$\dd{}{t} \left[\frac{m}{2}\left(2R^2\dot{\varphi} + 2R^2(\dot\varphi - \dot\Phi)\right)\right] + mgp = 0$$
$$R^2\ddot\varphi + R^2(\ddot\varphi - \ddot\Phi) + gp = 0$$
$$\dd{}{t} \left[\frac{m}{2} \left(2p^2(\dot{\varphi} - \dot{\Phi})\right) + \frac{MR^2}{2}\dot\Phi - mgp = 0\right]$$
$$-p^2\ddot\varphi + \left(\frac{MR^2}{2m} + p^2\right)\ddot\Phi - gp = 0$$
\begin{eqnarray}
    \ddot\varphi \left(R^2 + p^2\right) - \ddot\Phi p^2 + gp = 0 \\
    \ddot\Phi \left(\frac{MR^2}{2m} + p^2\right) - \ddot\varphi p^2 - gp = 0
\end{eqnarray}
Enačbo (1) pomnožimo z $\displaystyle{\frac{MR^2}{2m} + p^2}$, enačbo (2) pa s $p^2$, nato ju seštejemo (želimo se namreč znebiti spremenljivke $\Phi$). Dobimo linearno enačbo za $\ddot\varphi$ s preprosto rešitvijo
$$\ddot\varphi = \frac{gp\left(\frac{MR^2}{2m}\right)}{p^4 - \left(\frac{MR^2}{2m} + p^2\right)\left(R^2 + p^2\right)}$$
To je konstantno, označimo z $\alpha$. Tedaj je $\displaystyle{\varphi(t) = \frac{1}{2}\alpha t^2 + Ct + D}$. Na podoben način bi lahko dobili tudi $\ddot\Phi = \beta$ in rešitev za $\Phi(t)$.
\paragraph{Izbira oblike vodila za harmonsko nihanje.} Harmonsko nihanje pomeni, da je neodvisno od začetne amplitude. Njegova enačba je
$$\ddot\vartheta + w^2\vartheta = 0$$
Želimo najti vodilo, po katerem se bo masa gibala harmonično. \\
Želimo rezultat oblike $$L = \frac{1}{2} m\dot{s}^2 - \frac{1}{2}ks^2$$
Trenuto imamo $$L = \frac{1}{2}m(\dot{x}^2 + \dot{y}^2) - mgy$$
Kinetični del: $$\frac{1}{2}m(\dot{x}^2 + \dot{y}^2) = \frac{1}{2}m\dot{s}^2(\dot{x'}^2 + \dot{y'}^2)$$
(Pri tem smo uvedli nov koordinatni sistem $x',\, y'$)
$$\dot x = x' \dot s$$
$$\dot y = y' \dot s$$
Torej mora biti $x'^2 + y'^2 = 1$ oziroma $\displaystyle{(dx)^2 + (dy)^2 = (ds)^2}$. Sledi $\displaystyle{s = \int \sqrt{(dx)^2 + (dy)^2}}$. \\
To vstavimo v izraz za potencialno energijo: $$k\frac{1}{2}\left[\int \sqrt{(dx)^2 + (dy)^2}\right]^2 = mgy$$
$$\sqrt{\frac{k}{2}}\int \sqrt{1 + \left(\frac{dx}{dy}\right)^2}dy = \sqrt{mgy}$$
Z obojestranskim odvajanjem lahko izrazimo $\displaystyle{\dd{y}{x}}$:
$$\dd{y}{x} = \sqrt{\frac{2ky}{mg - 2ky}}$$
Da izrazimo $x$ in $y$, vzemimo $\displaystyle{\dd{y}{x} = \tan\varphi}$. Z nekaj trigonometrije dobimo $$\dd{\varphi}{x} = \frac{k}{mg}\frac{1}{\cos^2\varphi}$$
Z integriranjem dobimo
$$x = \frac{mg}{4k}(2\varphi + \sin2\varphi + C)$$
$$y = \frac{mg}{4k}(1 \cos2\varphi)$$
To je enačba cikloide. Izračunamo lahko tudi periodo nihanja (z integralom $\displaystyle{\int \frac{\sqrt{(dx)^2 + (dy)^2}}{\sqrt{2g(y - y_0)}}}$,
kjer $x$ in $y$ izrazimo preko $\varphi$). Dobimo $\displaystyle{T = 4\pi \sqrt{\frac{m}{4k}}}$
\end{document}