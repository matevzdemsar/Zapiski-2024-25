\documentclass[a4paper]{article}
\usepackage{amsmath, amssymb, amsfonts}
\usepackage[margin=1in]{geometry}
\usepackage{graphicx}
\usepackage{tikz}
\usepackage{esint}
\setlength{\parindent}{0em}
\setlength{\parskip}{1ex}
\newcommand{\vct}[1]{\overrightarrow{#1}}
\newcommand{\dif}{\mathrm{d}}
\newcommand{\pd}[2]{\frac{\partial {#1}}{\partial {#2}}}
\newcommand{\dd}[2]{\frac{\mathrm{d} {#1}}{\mathrm{d} {#2}}}
\newcommand{\C}{\mathbb{C}}
\newcommand{\R}{\mathbb{R}}
\newcommand{\Q}{\mathbb{Q}}
\newcommand{\Z}{\mathbb{Z}}
\newcommand{\N}{\mathbb{N}}
\newcommand{\fn}[3]{{#1}\colon {#2} \rightarrow {#3}}
\newcommand{\avg}[1]{\langle {#1} \rangle}
\newcommand{\Sum}[2][0]{\sum_{{#2} = {#1}}^{\infty}}
\newcommand{\Lim}[1]{\lim_{{#1} \rightarrow \infty}}
\newcommand{\Binom}[2]{\begin{pmatrix} {#1} \cr {#2} \end{pmatrix}}
\newcommand{\duline}[1]{\underline{\underline{#1}}}

\begin{document}
\paragraph{Milankovićevi cikli.} Zanima nas, zakaj na Zamlji pride do ledenih dob. Zemljina os vrtenja je nagnjena glede na ekvator za $\vartheta = 23.44^\circ$ in se s časom spreminja za $\pm 1^\circ$. Na isti kot pride vsakih $41\,000$ let.
Zemlja se vrti okoli Sonca po elipsi, ki se počasi vrti, in sicer s precesijskim časom $26\,000$ let. Posledično je na vsake toliko časa severna polobla hkrati nagnjena stran od Sonca in še na največji oddaljenosti od njega.
(Trenutno je poleti Zemlja bolj oddaljena od Sonca kot pozimi). To vodi do ledene dobe.
\paragraph{Sommerfeldov model majhnih nihanj.} Sledeči sistem obravnavajmo s pomočjo Lagrangeovega formalizma.
\begin{figure}[h!]
    \centering
    \begin{tikzpicture}
        \draw (-4, 0) -- (4, 0);
        \draw[dashed] (-3, 0) -- (-1.5, -3);
        \draw[dashed] (3, 0) -- (1.5, -3);
        \draw[dashed] (-1.5, -3) -- (1.5, -3);
        \filldraw (-1.5, -3) circle (0.07) node[left] {$m_1$};
        \filldraw (1.5, -3) circle (0.07) node[right] {$m_2$};
        \node (A) at (-3, -1.5) {$k_1,\,l_1$};
        \node (B) at (3, -1.5) {$k_2,\,l_2$};
        \node (C) at (0, -3.5) {$k,\,l$};
    \end{tikzpicture}
\end{figure}
$$T = \frac{1}{2}m_1\left(\dot l_1^2 + l_1^2\dot\varphi_1^2\right) + \frac{1}{2}m_2\left(\dot l_2^2 + l_2^2\dot\varphi_2^2\right)$$
$$V = m_1gl_1(1 - \cos\varphi_1) + m_2gl_2(1 - \cos\varphi_2) + \frac{1}{2}k_1(l_{10} - l_1)^2 + \frac{1}{2}k_2(l_{20} - l_2)^2 + \frac{1}{2}k(l_0 - l)^2$$
Iz tega sestavimo $L = T - V$ in uporabimo Euler-Lagrangeovo enačbo, ki pa ne bo rešljiva. (Dobimo nelinearne diferencialne enačbe, ki običajno niso analitično rešljive.)
Največ, kar lahko naredimo, je da si ogledamo obnašanje sistema za majhne odmike okoli stabilne lege.
$$L = T - V = \frac{1}{2}\sum_{ij} w_{ij}(\underline{q})\dot q_i \dot q_j$$
$$V(\underline{q}) = V(\underline{q_0}) + \sum_{i=1}^{n} \pd{V}{q_i}\Big|_{q_{i,0}}(q_i - q_{i, 0}) + \sum_{ij} \pd{^2V}{q_i \partial q_j}\Big|_{\underline{q_{0}}}(q_i - q_{i, 0})(q_j - q_{j, 0}) + ... \text{ (Taylor)}$$
Prva vsota je enaka 0, saj je $\displaystyle{\pd{V}{q_i}(q_0) = 0}$. Navsezadnje je $q_0$ stabilna lega, torej mora imeti potencial v njej minimum. Mimogrede označimo $\eta_i = q_i - q_{i, 0}$.
$$T = \frac{1}{2}\sum_{ij} T_{ij} \dot\eta_i\dot\eta_j$$
$$T_{ij} = w_{ij}(\underline{q_0}) + \sum_{k} \pd{w_i}{q_k}\eta_k + \dots$$
S tem smo dobili matriki $T$ in $V$:
$$V_{ij} = \pd{^2 V}{q_i \partial q_j}\Big|_{\underline{q_0}}$$
$$T_{ij} = w_{ij}(\underline{q_0}) + \sum_{k}\pd{w_i}{q_k}\eta_k$$
$$\tilde{L} = \frac{1}{2}\sum_{ij}\left(T_{ij}\dot\eta_i\dot\eta_j - V_{ij}\eta_i\eta_j\right)$$
To si smemo privoščiti za dovolj majhne odmike od ravnovesne lege. Zvaj uporabimo Euler-Lagrangeovo enačbo:
$$\dd{}{t}\pd{\tilde{L}}{\dot\eta_i} - \pd{\tilde{L}}{\eta_i} = 0$$
$$\sum_{j} T_{ij}\ddot{\eta}_j + \sum_j V_{ij}\eta_j = 0~~\forall i$$
Dobimo sistem enačb, ki ga bomo znali rešiti.
\paragraph{Notacija.} Pri tej analizi uporabljamo sledečo notacijo:
\begin{itemize}
    \item $\underline{a} = \begin{pmatrix}
        a_1 \\ a_2 \\ \vdots \\ a_n
    \end{pmatrix}$ je vektor (stolpec).
    \item $\underline{a}^T = (a_1, a_2, ..., a_n)$ je vrstica.
    \item $\duline{A}$ je matrika, $\duline{A}^T$ je transponirana matrika.
    \item $\displaystyle{\duline{A}\underline{a}_i = \sum_{j=1}^{n} A_{ij}a_j}$
    \item $\displaystyle{\underline{a}^T\duline{A} = \sum_{j=1}^n a_j A_{ij}}$
\end{itemize}
V taki notaciji imamo opravka z enačbo
$$\tilde{L} = \frac{1}{2}\left(\underline{\dot\eta}^T\duline{T}\underline{\dot\eta} - \underline{\eta}^T\duline{V}\underline{\eta}\right)$$
\end{document}