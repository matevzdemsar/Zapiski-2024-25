\documentclass[a4paper]{article}
\usepackage{amsmath, amssymb, amsfonts}
\usepackage[margin=1in]{geometry}
\usepackage{graphicx}
\usepackage{tikz}
\usepackage{esint}
\setlength{\parindent}{0em}
\setlength{\parskip}{1ex}
\newcommand{\vct}[1]{\overrightarrow{#1}}
\newcommand{\dif}{\mathrm{d}}
\newcommand{\pd}[2]{\frac{\partial {#1}}{\partial {#2}}}
\newcommand{\dd}[2]{\frac{\mathrm{d} {#1}}{\mathrm{d} {#2}}}
\newcommand{\C}{\mathbb{C}}
\newcommand{\R}{\mathbb{R}}
\newcommand{\Q}{\mathbb{Q}}
\newcommand{\Z}{\mathbb{Z}}
\newcommand{\N}{\mathbb{N}}
\newcommand{\fn}[3]{{#1}\colon {#2} \rightarrow {#3}}
\newcommand{\avg}[1]{\langle {#1} \rangle}
\newcommand{\Sum}[2][0]{\sum_{{#2} = {#1}}^{\infty}}
\newcommand{\Lim}[1]{\lim_{{#1} \rightarrow \infty}}
\newcommand{\Binom}[2]{\begin{pmatrix} {#1} \cr {#2} \end{pmatrix}}
\newcommand{\duline}[1]{\underline{\underline{#1}}}

\begin{document}
\paragraph{Formalna obravnava vrtavke.} Radi bi opisali gibanje vrtavke kot funkcijo časa. Imamo Eulerjeve kote $\varphi$, $\psi$ in $\vartheta$.
\begin{figure}[h!]
    \centering
    \begin{tikzpicture}
        \draw[->] (0, 0) -- (2, 0) node[right] {$x$};
        \draw[->] (0, 0) -- (0, 2) node[right] {$y$};
        \draw[->] (0, 0) -- (-1, -1.73) node[below] {$z$};
        \draw[dotted, ->] (0, 0) -- (1.73, 1) node[above] {$x'$};
        \draw[dotted, ->] (0, 0) -- (-1, 1.73) node[above] {$y'$};
        \draw[dotted, ->] (0, 0) -- (0, -2) node[right] {$z'$};
        \draw[->] (0, 1) arc (90:120:1);
        \draw[->] (-0.15, 1.6) arc (135:405:0.2);
        \draw[->] (-0.95, 1.3) arc (180:450:0.2);
        \node (A) at (0.5, 1.6) {$\varphi$};
        \node (B) at (-1.2, 1.3) {$\psi$};
        \node (A) at (-0.5, 0.5) {$\vartheta$};
    \end{tikzpicture}
\end{figure}
\newline
Uporabili bomo Lagrangeov formalizem:
$$T = \frac{1}{2}J_1{\omega'}_1^2 + \frac{1}{2}J_1{\omega'}_2^2 + \frac{1}{2}J_3{\omega'}_3^2$$
$$\vct{\omega} = \dot\varphi \vct{e_3} + \dot\vartheta \vct{e_1''} + \dot\psi\vct{e_3'}$$
$\dot\varphi \vct{e_3}$ pomeni precesijo, $\dot\vartheta \vct{e_1''}$ pomeni nagibanje, $\dot\varphi\vct{e_3'}$ pa vrtenje okoli lastne osi. \\
V različne baze ($[x', y', z']$, $[x'', y'', z'']$, $[x''', y''', z''']$) pridemo z rotacijskimi matrikami, obravnavani v poglavju o Eulerjevih kotih.
S temi rotacijskimi matrikamo $\vct{\omega}$ preslikamo v $\vct{\omega'}$, nato pa izračunamo kinetično energijo.
$$T = \frac{1}{2}J_1({\omega'}_1^2 + {\omega'}_2^2) + \frac{1}{2}J_3{\omega'}_3^2$$
$${\omega'}_1^2 + {\omega'}_2^2 \sin^2\vartheta\sin^2\psi\,\dot\varphi^2 + \sin^2\vartheta\cos^2\psi\,\dot\varphi^2 + (\cos^2\psi + \sin^2\psi)\,\dot\vartheta^2 + 2(\sin\vartheta\sin\psi\cos\psi - \in\vartheta\sin\psi\cos\psi)\dot\vartheta\dot\varphi$$
$$= \sin^2\vartheta\,\dot\varphi^2 + \dot\vartheta^2$$
$${\omega'}_3^2 = \left(\dot\psi + \cos\vartheta\,\dot\varphi\right)^2$$
Potencialna energija: Če označimo z $l$ višino težišča vrtavke, ko ta stoji pokonci, je potencialna energija pri nekem kotu $\vartheta$ enaka
$$V = mgl\cos\vartheta$$
$$L = T - V = \frac{1}{2} J_1\left(\sin^2\vartheta\,\dot\varphi^2+\dot\vartheta^2\right) + \frac{1}{2}J_3\left(\dot\psi + \cos\vartheta\,\dot\varphi\right)^2 - mgl\cos\vartheta = L(\vartheta, \dot\vartheta, \dot\varphi, \dot\psi)$$
Iz ohranitve energije:
$$E = T + V = \text{konst.}$$
Iz Lagrangeovega formalizma vidimo, da sta konstantni tudi količini
$$p_\psi = \pd{L}{\dot\psi} = J_3\left(\dot\psi + \cos\vartheta\,\dot\varphi\right) = J_1 a = \text{konst.}$$
$$p_\varphi = \pd{L}{\dot\varphi} = J_1\sin^2\vartheta\,\dot\varphi + J_3\cos\vartheta\left(\dot\psi + \cos\vartheta\,\dot\varphi\right) = J_1b = \text{konst.}$$
Iz $p_\psi$ lahko izrazimo:
$$J_3\dot\psi = J_1 a - J_3\cos\vartheta\,\dot\varphi$$
To nam vstavimo v $p_\varphi$:
$$(J_1\sin^2\vartheta + J_3\cos^2\vartheta)\,\dot\varphi + \cos\vartheta(J_1 a - J_3\cos\vartheta\,\dot\varphi) = J_1 b$$
Na obeh straneh delimo z $J_1$. Člen $J_3\cos^2\vartheta$ se odšteje. Izrazimo $\dot\varphi$
$$\dot\varphi = \frac{b-a\cos\vartheta}{\sin^2\vartheta}$$
To vstavimo v enačbo za energijo:
$$E = \frac{1}{2}J_1\left(\frac{(b - a\cos\vartheta)^2}{\sin^2\vartheta} + \dot\vartheta^2\right) + \frac{1}{2}J_3{\omega'}_3^2 + mgl\cos\vartheta = E_0$$
Dobili smo diferencialno enačbo z eno neznanko. Zdaj uvedemo novo količino.
$$\tilde{E} = E_0 - \frac{1}{2}J_3{\omega'}^2 = \tilde{E}_0 = \tilde{E} + \tilde{V}(\vartheta)$$
$$\pd{\vartheta}{t} = \sqrt{\frac{2}{J}\left(\tilde{E_0}-\tilde{V}(\vartheta)\right)}$$
Integriramo po $\vartheta$:
$$t = \sqrt{\frac{J_1}{2}} \int_{\vartheta(0)}^{\vartheta(t)}\frac{\dif\vartheta}{\sqrt{\tilde{E}_0 - \tilde{V}(\vartheta)}}$$
Tu uvedemo spremenljivko $u = \cos\vartheta$. Vemo, da bo $-1 \leq u \leq 1$:
$$t = \int \sqrt{\frac{J_1}{2}} \frac{\dif u}{\sin^2\vartheta \tilde{E}_0 - \sin^2\vartheta\tilde{V}} =: \int \frac{\dif u}{\sqrt{f(u)}}$$
Definirali smo $\displaystyle{f(u) = \frac{2}{J_1}\left(\sin^2\vartheta\tilde{E}_0 - \sin^2\vartheta\tilde{U}\right)}$. V korenu bomo dobili nekakšen polinom.
Upoštevamo $u = \cos\vartheta$, $\sin^2\vartheta = (1 - u^2)$.
$$f(u) = \frac{2}{J_1}\left((1-u^2)\tilde{E}_0 - (1-u^2)\frac{J_1}{2}\frac{(b-au^2)^2}{(1-u^2)} + mglu\right)$$
Med $u_2=-1$ in $u_1 = 1$ je $f(u) \geq 0$, torej je integral definiran.
\end{document}