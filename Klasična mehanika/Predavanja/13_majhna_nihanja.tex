\documentclass[a4paper]{article}
\usepackage{amsmath, amssymb, amsfonts}
\usepackage[margin=1in]{geometry}
\usepackage{graphicx}
\usepackage{tikz}
\usepackage{esint}
\setlength{\parindent}{0em}
\setlength{\parskip}{1ex}
\newcommand{\vct}[1]{\overrightarrow{#1}}
\newcommand{\dif}{\,\mathrm{d}}
\newcommand{\pd}[2]{\frac{\partial {#1}}{\partial {#2}}}
\newcommand{\dd}[2]{\frac{\mathrm{d} {#1}}{\mathrm{d} {#2}}}
\newcommand{\C}{\mathbb{C}}
\newcommand{\R}{\mathbb{R}}
\newcommand{\Q}{\mathbb{Q}}
\newcommand{\Z}{\mathbb{Z}}
\newcommand{\N}{\mathbb{N}}
\newcommand{\fn}[3]{{#1}\colon {#2} \rightarrow {#3}}
\newcommand{\avg}[1]{\langle {#1} \rangle}
\newcommand{\Sum}[2][0]{\sum_{{#2} = {#1}}^{\infty}}
\newcommand{\Lim}[1]{\lim_{{#1} \rightarrow \infty}}
\newcommand{\Binom}[2]{\begin{pmatrix} {#1} \cr {#2} \end{pmatrix}}
\newcommand{\duline}[1]{\underline{\underline{#1}}}

\begin{document}
\paragraph{Termini za ustne izpite.} 12. ter 13. 6 in 1. ter 2. 7.
\paragraph{Majhna nihanja.} Imamo vektor odmika $\underline{\eta}$.
$$\tilde{L} = \frac{1}{2}\dot{\underline{\eta}}^T\duline{T}\dot{\underline{\eta}} - \frac{1}{2}\underline{\eta}^T\duline{V}\underline{\eta}$$
$$\sum_j T_{ij}\ddot\eta_j + \sum_j V_{ij}\eta_j = 0,~\forall i$$
\paragraph{Lastna nihanja.} $\eta_i = \alpha a_i e^{i\omega t},~\omega\in\R$
To vstavimo v enačbo, da dobimo
$$\sum_j V_{ij}a_j - \omega^2 \sum_j T_{ij}a_j = 0$$
$$\duline{V}\underline{a} = \omega^2\duline{T}\underline{a} = \lambda\duline{T}\underline{a},~\omega^2 \geq 0$$
\paragraph{Poseben primer.} $\duline{T} = T\duline{I}$.
$$(\duline{V} - \lambda T \duline{I})\underline{a} = (\duline{V} - \tilde{\lambda} I)\underline{a} = 0$$
Pokažemo lahko, da so lastni vektorji $\underline{a_k}$ ortogonalni. Iz normiranih vektorjev $\underline{a_k}$ sestavimo matriko $\duline{A}$.
$$\duline{A} = \begin{bmatrix}
    a_{11} & a_{21} & a_{31} & \dots & a_{n1} \\
    a_{12} & \ddots &&& \\
    a_{13} && \ddots && \\
    \vdots &&& \ddots & \\
    a_{1n} &&&& a_{nn} \\
\end{bmatrix} = (\underline{a_1}, \underline{a_2}, ... \underline{a_n})$$
Velja:
$$\duline{A}^T\duline{A} = \duline{A}\duline{A}^T = \duline{T}$$
$$\duline{A}^T \duline{V} \duline{A} = \duline{\tilde{\Lambda}}$$
tu je $\duline{\tilde{\Lambda}}$ diagonalna matrika oblike
$$\duline{\tilde{\Lambda}} = \begin{bmatrix}
    \lambda_1 &&& \\
    & \lambda_2 && \\
    && \ddots & \\
    &&& \lambda_n \\
\end{bmatrix}$$
Poleg tega velja tudi
$$\duline{A}^T \duline{T} \duline{A} = \duline{I}$$
$$\duline{A}^T \duline{V}\duline{A} = \duline{A}^T\duline{T}\duline{A}\duline{\Lambda} = \duline{\Lambda}$$
\paragraph{Normalne koordinate.} Od prej:
$$\tilde{L} = \frac{1}{2}\dot{\underline{\eta}} \duline{T} \dot{\underline{\eta}} - \frac{1}{2}\underline{\eta} \duline{V} \underline{\eta}$$
Ker je $\eta$ linearna kombinacija lastnih vektorjev, zapišemo $\eta = \duline{A}\underline{\alpha}$. Sledi:
$$\tilde{L} = \frac{1}{2}\dot{\underline{\alpha}}^T\dot{\underline{\alpha}} - \frac{1}{2}\underline{\alpha}^T\duline{\lambda}\underline{\alpha}$$
\paragraph{Hamiltonov formalizem.} $L = T - V$;
$$L = L(\underline{q}, \dot{\underline{q}}, t)$$
$$p_i = \pd{L}{\dot{q}_i}$$
$$\dd{}{t}\pd{L}{\dot{q}} - \pd{L}{q_i} = 0$$
Definiramo $H = \displaystyle{\sum_{i}p_i\dot{q}_i} - L = T + V = H(\underline{q}, \underline{p}, t)$
$$\dif H = \sum_i \left(\dot{q}_i\dif p_i + p_i \dif \dot{q}_i\right) - \sum_i \left(\pd{L}{q_i}\dif q_i + \pd{L}{\dot{q}_i}\dif\dot{q}_i\right) - \pd{L}{t}\dif t$$
Dobimo Hamiltonove enačbe:
$$\dot{q}_i = \pd{H}{p_i}$$
$$\dot{p}_i = -\pd{H}{q_i}$$
$$\pd{H}{t} = -\pd{L}{t}$$
Kajti $\displaystyle{\dif H = \pd{H}{q_i}\dif q_i + \pd{H}{p_i} \dif p_i + \pd{H}{t}\dif t}$
\paragraph{Nabit delec v elektromagnetnem polju.} $$\vct{F} = e(\vct{E} + \vct{v} \times \vct{B})$$
$$\dd{}{t}\pd{L}{\dot{q}_i} - \pd{L}{q_i} = 0,~~F_i = \pd{V}{q_i} + \pd{}{t}\pd{V}{\dot{q}_i},~~V(\duline{q}, \duline{\dot{q}}, t)$$
\\
$$\vct{B} = \nabla\times\vct{A}$$
$$\vct{E} = -\nabla\phi - \pd{\vct{A}}{t}$$
$$\nabla\times\vct{E} = - \nabla\times\pd{\vct{A}}{t} = -\dd{}{t}\nabla\times\vct{A} = - \pd{\vct{B}}{t}$$
$$\vct{F} = e\left[-\nabla\phi - \pd{\vct{A}}{t} + \vct{v}\times(\nabla \times \vct{A})\right]$$
$$H = \frac{(\vct{p} - e\vct{A})^2}{2m} + V + e\phi$$
\end{document}