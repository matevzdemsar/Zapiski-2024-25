\documentclass[a4paper]{article}
\usepackage{amsmath, amssymb, amsfonts}
\usepackage[margin=1in]{geometry}
\usepackage{graphicx}
\usepackage{tikz}
\usepackage{esint}
\setlength{\parindent}{0em}
\setlength{\parskip}{1ex}
\newcommand{\vct}[1]{\overrightarrow{#1}}
\newcommand{\dif}{\,\mathrm{d}}
\newcommand{\pd}[2]{\frac{\partial {#1}}{\partial {#2}}}
\newcommand{\dd}[2]{\frac{\mathrm{d} {#1}}{\mathrm{d} {#2}}}
\newcommand{\C}{\mathbb{C}}
\newcommand{\R}{\mathbb{R}}
\newcommand{\Q}{\mathbb{Q}}
\newcommand{\Z}{\mathbb{Z}}
\newcommand{\N}{\mathbb{N}}
\newcommand{\fn}[3]{{#1}\colon {#2} \rightarrow {#3}}
\newcommand{\avg}[1]{\langle {#1} \rangle}
\newcommand{\Sum}[2][0]{\sum_{{#2} = {#1}}^{\infty}}
\newcommand{\Lim}[1]{\lim_{{#1} \rightarrow \infty}}
\newcommand{\Binom}[2]{\begin{pmatrix} {#1} \cr {#2} \end{pmatrix}}
\newcommand{\duline}[1]{\underline{\underline{#1}}}

\begin{document}
\paragraph{Poissonovi oklepaji.} Ponovno imamo $f(\underline{q}, \underline{p}, t)$, kjer so posplošene koordinate $\underline{q}$ in impulzi $\underline{p}$ odvisni od časa. Spet je $\displaystyle{p_i = \pd{L}{\dot{q}_i}},~\dot{q}_i = \pd{H}{p_i}, ~\dot{p}_i = -\pd{H}{q_i}$.
$$\dd{f}{t} = \sum_{i} \left(\pd{f}{q_i}\dot{q}_i + \pd{f}{p_i}\dot{p}_i\right) + \pd{f}{t}$$
$$= \sum_i \left(\pd{f}{q_i}\pd{H}{p_i} - \pd{f}{p_i}\pd{H}{q_i}\right) + \pd{f}{t} = \left[f, H\right] + \pd{f}{t}$$
Definirali smo Poissonov oklepaj $\displaystyle{\left[f, g\right] = \sum_i \left(\pd{f}{q_i}\pd{g}{p_i} - \pd{f}{p_i}\pd{g}{q_i}\right)}$. Včasih se uporablja tudi oznako $\{f, g\}$.
\paragraph{Lastnosti.} Poissonovi oklepaji imajo naslednje lastnosti:
\begin{enumerate}
    \item $[f, \lambda g + \eta h] = \lambda[f, g] + \eta[f, h]$
    \item $[f, g] = -[g, f]$
    \item $[f, gh] = [f, g]h + g[f, h]$
    \item $[f, f] = 0$
    \item $[x, p_x] = 1$
    \item $[x, p_y] = 0$
\end{enumerate}
\paragraph{Lagrangeov formalizem za zvezno sredstvo.} Na primer za struno:
$$\dif T = \frac{1}{2}\dot{u}^2\rho S_0 \dif x = \frac{1}{2}\rho S_0 \left(\pd{u}{t}\right)^2\dif x$$
$$\dif V = F(\dif l - \dif x) = F(\sqrt{\dif x^2 + \dif u^2} - \dif x) \approx \frac{1}{2}F \left(\pd{u}{x}\right)^2\dif x$$
$$\dif L = \frac{1}{2}\rho S_0 \left(\pd{u}{t}\right)^2\dif x \pm \frac{1}{2} F\left(\pd{u}{x}\right)^2\dif x \pm \rho S_0 u\dif x$$
$$\mathcal{L} = \dd{L}{x} = ...$$
$$L = \int_{x_1}^{x_2}\mathcal{L}\dif x = \int_{x_1}^{x_2} \mathcal{L}\left(u(x, t), u_x, u_t, x, t\right)$$
$$S = \int_{0}^{t}L\dif t = \iint\mathcal{L}\dif x \dif t \, (\mathrm{d} y, \dif z)$$
Euler-Lagrangeova enačba za tak sistem je
$$\dd{}{t}\pd{\mathcal{L}}{u_t} + \dd{}{x}\pd{\mathcal{L}}{u_x} - \pd{\mathcal{L}}{x} = 0$$
Iz tega lahko izpeljamo na primer Maxwellove zakone, valovne enačbe itd.
\end{document}