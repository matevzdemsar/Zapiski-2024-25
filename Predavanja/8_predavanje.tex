\documentclass[a4paper]{article}
\usepackage{amsmath, amssymb, amsfonts}
\usepackage[margin=1in]{geometry}
\usepackage{graphicx}
\usepackage{tikz}
\usepackage{esint}
\setlength{\parindent}{0em}
\setlength{\parskip}{1ex}
\newcommand{\vct}[1]{\overrightarrow{#1}}
\newcommand{\dif}{\mathrm{d}}
\newcommand{\pd}[2]{\frac{\partial {#1}}{\partial {#2}}}
\newcommand{\dd}[2]{\frac{\mathrm{d} {#1}}{\mathrm{d} {#2}}}
\newcommand{\C}{\mathbb{C}}
\newcommand{\R}{\mathbb{R}}
\newcommand{\Q}{\mathbb{Q}}
\newcommand{\Z}{\mathbb{Z}}
\newcommand{\N}{\mathbb{N}}
\newcommand{\fn}[3]{{#1}\colon {#2} \rightarrow {#3}}
\newcommand{\avg}[1]{\langle {#1} \rangle}
\newcommand{\Sum}[2][0]{\sum_{{#2} = {#1}}^{\infty}}
\newcommand{\Lim}[1]{\lim_{{#1} \rightarrow \infty}}
\newcommand{\Binom}[2]{\begin{pmatrix} {#1} \cr {#2} \end{pmatrix}}

\begin{document}
\section{Razpad jeder}
\paragraph{Razpad $\gamma$} Ko je jedro v vzbujenem stanju, pridejo v osnovno stanje tako, da odda foton (kot elektron v atomu).
$$X^* \to X + \gamma$$
Energija fotona je $\Delta E = \hbar \omega$. Zanima nas razpadni čas:
$$\Gamma = \frac{1}{\tau} = \frac{\omega^3|\avg{\vct{p}_{e_{1, 2}}}|}{3\pi\varepsilon_0c^3\hbar}$$
Bolj ali manj točno tako kot pri elektronih. S $\vct{p}_e$ označimo dipolni moment $e\vct{r}$
$$\avg{\vct{p}_{e_{1,2}}} = \int_V \psi^*_2(\vct{r})\hat{\vct{p}}_{e}\psi_1(\vct{r}) \dif V = \int_V \psi^*_2(\vct{r})e\vct{r}\psi_1(\vct{r}) \dif V$$
Da bo to različno od 0, mora biti integrirana funkcija soda. Sledi, da morata imeti $\psi_2$ in $\psi_1$ različno parnost. Drugi pogoj je povezan z vrtilno količino jedra.
\begin{align*}
    \text{Pogoj 1:  } & P(\psi_2) = -P(\psi_1) \\
    \text{Pogoj 2:  } & J' - J = 0, \pm 1 \\
\end{align*}
Če ta dva pogoja nista izpolnjena, je $\avg{\vct{p}_e} = 0$ in do prehoda ne pride (kajti $1/\tau = 0 \Rightarrow \tau = \infty$)
\paragraph{Razpad $\alpha$} Gre za razpad oblike
$$^A_ZX \to ^{A-4}_{Z-2}Y + ^4_2He$$
Primer:
$$^{226}_{88}Ra \to ^{222}_{86}Rn + ^{4}_{2}\alpha$$
$$\Delta E = (M_{Ra} - M_{Rn} - M_{\alpha})c^2 = 4.87\,\text{MeV}$$
Energijska razlika je dovolj majhna, da se nam z relativnostjo še ni treba ukvarjati. Upoštevajmo ohranitev gibalne količine ter kinetične energije:
$$\Delta E = \frac{1}{2}M_{Ra}v^2_{Rn} + \frac{1}{2}M_{\alpha}v^2_{\alpha}$$
$$0 = M_{Rn} v_{Rn} + M_{\alpha}v_{\alpha}$$
Dobimo $\displaystyle{v_{Rn} = \frac{M_\alpha}{M_{Rn}}v_{\alpha}}$,
kar pomeni $T_\alpha \approx \Delta E$, kajti hitrost $\alpha$ delca je mnogo večja od hitrosti jedra. \\[3mm]
Za \(r > R\) deluje na \(\alpha\) delec potencial \[V_C = C\frac{2(Z-2)}{r}\]
Vpeljali smo konstanto \(\displaystyle{C = \frac{e_0^2}{4\pi\varepsilon_0}} = 1,44\,\mathrm{MeV}\)
Gre za potencialno bariero, ki jo mora delec premagati, da pride iz jedra. Da mu to uspe, mora imeti določeno kinetično energijo \(T_\alpha\) (označimo tudi \(W_k\)). Ta je, kot smo že omenili, približno enaka \(\Delta E\),
točneje pa jo izračunamo kot potencial v točki \(R_0\), ki predstavlja radij jedra - sila med delci v jedru ima namreč zelo kratek doseg.
\[W_k = V_C(R_0) = \frac{2e_0^2(Z-2)}{4\pi\varepsilon_0R_0}\]
Obratno lahko iz kinetične energije (ki je, kot bomo kmalu videli, merljiva na podlagi razpadnega časa) izračunamo raij jedra:
\[R_0 = \frac{2e_0^2(Z-2)}{4\pi\varepsilon_0W_k} = C\frac{2(Z-2)}{W_k}\]
\paragraph{Razpadni čas.} Razpadni čas je odvisen bolj ali manj samo od potrebne kinetične energije \(\alpha\) delca:
\[\ln\tau = \frac{a}{\sqrt{W_k}} + b\]
To formulo izpeljemo na osnovi prepustnosti. Prepustnost skozi neko tanko tanko plast potenciala je enaka:
\[T = \frac{1}{1 + \frac{1}{16}\left(\frac{k}{K} + \frac{K}{k}\right)^2\left(e^{-K\Delta r} - e^{K\Delta r}\right)^2}\]
\[k = \frac{1}{\hbar}\sqrt{2mW_k}\]
\[K = \frac{1}{\hbar}\sqrt{2m(V_C - W_k)}\]:
\[T_i = \frac{1}{1 + \frac{1}{16}\left(\frac{k}{K} + \frac{K}{k}\right)^2\left(e^{-K\Delta r} - e^{K\Delta r}\right)^2}\]
Obravnavamo limito \(K\Delta r \gg 1\), torej lahko nekaj stvari zanemarimo:
\[\frac{1}{1 + \frac{1}{16}\left(\frac{k}{K} + \frac{K}{k}\right)^2\left(e^{-K\Delta r} - e^{K\Delta r}\right)^2} \approx \frac{1}{\frac{1}{16}\left(\frac{k}{K} + \frac{K}{k}\right)^2\left(e^{K\Delta r}\right)^2 = ... = B\,e^{-2K\Delta r}}\]
Označili smo \(\displaystyle{B = \frac{16}{\left(\frac{k}{K} + \frac{K}{k}\right)^2}}\)
Skupna prepustnost je produkt prepustnosti posameznih plasti:
\[T = \prod_{i}T_i = \prod_{i}e^{-2K_i\Delta r_i}\]
\[\ln T = \sum_{i}\ln T_i = B' - \sum_{i} 2K_i\Delta r\]
V limiti, ko so plasti neskončno tanke, vsoto zapišemo kot integral:
\[\ln T = B' - \int_{R}^{R_0}2K(r)\,\dif r = B' - 2\int_{R}^{R_0}\frac{1}{\hbar}\sqrt{2m(V_C(r) - W_k)}\,\dif r\]
Iz relacije \(W_k = V_C(R_0)\) dobimo \(V_C/W_k = R_0/r\), kar nam omogoči uvedbo nove spremenljivke \(u = W_k/V_C = r/R_0\)
Preden vstavimo v integral, razčistimo še sledeče:
\[V_C(r) - W_k = W_k\left(\frac{V_C(r)}{W_k} - 1\right) = W_k\left(\frac{1}{u} - 1\right)\]
Izrazimo še \(\dif u = \dif r / R_0\) in vse skupaj vstavimo v integral.
\[\ln T = B' - 2\int_{R/R_0}^{1} \frac{1}{\hbar}\sqrt{2mW_k}\sqrt{\frac{1}{u} - 1}\,R_0\,\dif u\]
\[\ln T = B' - \frac{2R_0}{\hbar}\sqrt{2mW_k}\int_{R/R_0}^{1}\sqrt{\frac{1}{u}-1}\,\dif u = B' - \frac{2R_0}{\hbar}\sqrt{2mW_k}\left[\int_{0}^{1}\sqrt{\frac{1}{u}-1}\,\dif u - \int_{0}^{R/R_0}\sqrt{\frac{1}{u}-1}\,\dif u\right]\]
Integral od \(0\) do \(1\) rešujemo s substitucijo \(u = \sin^2\varphi\). Pri drugem integralu si tega ne moremo privoščiti, ker ne poznamo novih mej, tu pa sta novi meji \(0\) in \(\pi/2\).
\[\int_{0}^{1} \sqrt{\frac{1}{u} - 1} \,\dif u = \int_{0}^{\pi/2}2\sin\varphi\cos\varphi\sqrt{\frac{1}{\sin^2\varphi} - \frac{\sin^2\varphi}{\sin^2\varphi}\,\dif\varphi} = \int_{0}^{\pi/2}\cos^2\varphi\,\dif\varphi = \frac{\pi}{2}\]
Pri integralu od \(0\) do \(R/R_0\) se bomo morali zadovoljiti s približkom, da je \(u\) dovolj majhen, da velja \(\displaystyle{\sqrt{\frac{1}{u} - 1} \approx \frac{1}{\sqrt{u}}}\):
\[\int_{0}^{R/R_0}\sqrt{\frac{1}{u} - 1}\,\dif u \approx \int_{0}^{R/R_0}\frac{\dif u}{\sqrt{u}} = 2\sqrt{\frac{R}{R_0}}\]
Zdaj \(R_0\) izrazimo s kinetično energijo:
\[\ln T = B' - \frac{2}{\hbar}\sqrt{2mW_k}C\frac{2(Z-2)}{W_k}\left(\frac{\pi}{2} - 2\sqrt{\frac{2RW_k}{2C(Z-2)}}\right)\]
Zdaj dobljeni \(\ln T\) prepišemo kot vsoto funkcije kinetične energije in neke konstante:
\[\ln T = B'' - \frac{2\pi}{\hbar}(Z-2)C\frac{2m}{W_k}\]
Zdaj naredimo sledeči razmislek: razpadni čas je odvisen od frekvenco 'poskusov' razpada (\(\nu\)). To število lahko karakteriziramo s časom, ki ga delec porabi do roba jedra:
\[\frac{1}{\tau} = T\nu = \frac{v}{R} = \frac{\sqrt{\frac{2W_k}{m}}}{2R}\]
Izrazimo \(\tau\), nato pa še njegov logaritem. Upoštevamo izračunani \(\ln T\) od prej.
\[\tau = T^{-1} \frac{2R\sqrt{m_\alpha}}{\sqrt{W_k}}\]
\[\ln\tau = \ln T^{-1} + \ln\frac{2R\sqrt{m_\alpha}}{\sqrt{W_k}} = -B'' + \frac{e_0^2(Z-2)}{2\varepsilon_0\hbar} \sqrt{\frac{2m}{W_k}} + \ln\frac{2R\sqrt{m_\alpha}}{\sqrt{W_k}}\]
Če drugi člen zanemarimo kot konstantnega, dobimo \(\displaystyle{\ln\tau \approx \frac{a}{\sqrt{W_k}} + b}\). Običajno lahko izmerimo \[a \sim 340\,\mathrm{MeV}\] \\[3mm]
V praksi to pomeni, da razpade opazimo šele, ko so potrebne kinetične energije dovolj velike, da je razpadni čas razmeroma majhen.
Teoretično so razpadi možni že pri \(A = 140\), vendar so razpadni časi tako veliki, da jih nima smisla opazovati.
Pri \(A = 206\) pa dobimo kinetične energije reda velikosti \(4\,\mathrm{MeV}\), kar nam da dovolj majhen razpadni čas, da je razpad vredno opazovati.
\paragraph{Spontani razcep.} Nastaneta dve približno enako veliki jedri. Zaenkrat za ta razccep povemo samo, da obstaja.
\paragraph{Razpad \(\beta\).} Delimo ga na pozitivni in negativni \(\beta\) razpad - odvisno od tega, kaj se pri njem zgodi z nabojem jedra.
\begin{table}[h!]
    \centering
    \begin{tabular}{c c}
        \(\beta^-:\) & \(n \to p e^- \overline{\nu_e}\) \\
        \(\beta^+:\) & \(p\to n e^+ \nu_e\) \\
    \end{tabular}
\end{table}
Pri tem se sprosti nekaj energije, označimo \(E_0\). Nastalo jedro je pogosto v vzbujenem stanju in odda še žarek \(\gamma\), kar je uporabno za obsevanje raka ali kaj podobnega.
Če ima reakcija dolg razpadni čas, pa jo lahko uporabimo za datiranje arheoloških najdb. \\
Vidimo, da je hitrost razpadanje jeder odvisna od njihovega števila:
$$\dd{N}{t} = - \alpha N(t)$$
$$\int_{N_0}^{N}\frac{\dif N}{N} = -\int_{0}^{t}\alpha \dif t$$
$$N = N_0\,e^{-\alpha t}$$
Običajno označimo $\alpha = 1/\tau$, kjer je $\tau$ razpadni čas. Definiramo pa lahko tudi pojem razpolovnega časa, ki je definiran kot
$$t_{1/2} = \tau \ln 2$$
\end{document}