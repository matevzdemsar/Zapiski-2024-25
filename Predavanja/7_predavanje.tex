\documentclass[a4paper]{article}
\usepackage{amsmath, amssymb, amsfonts}
\usepackage[margin=1in]{geometry}
\usepackage{graphicx}
\usepackage{tikz}
\usepackage{esint}
\setlength{\parindent}{0em}
\setlength{\parskip}{1ex}
\newcommand{\vct}[1]{\overrightarrow{#1}}
\newcommand{\dif}{\mathrm{d}}
\newcommand{\pd}[2]{\frac{\partial {#1}}{\partial {#2}}}
\newcommand{\dd}[2]{\frac{\mathrm{d} {#1}}{\mathrm{d} {#2}}}
\newcommand{\C}{\mathbb{C}}
\newcommand{\R}{\mathbb{R}}
\newcommand{\Q}{\mathbb{Q}}
\newcommand{\Z}{\mathbb{Z}}
\newcommand{\N}{\mathbb{N}}
\newcommand{\fn}[3]{{#1}\colon {#2} \rightarrow {#3}}
\newcommand{\avg}[1]{\langle {#1} \rangle}
\newcommand{\Sum}[2][0]{\sum_{{#2} = {#1}}^{\infty}}
\newcommand{\Lim}[1]{\lim_{{#1} \rightarrow \infty}}
\newcommand{\Binom}[2]{\begin{pmatrix} {#1} \cr {#2} \end{pmatrix}}

\begin{document}
Pri prejšnjem predavanju smo za energijo nukleonov izpeljali
$$E_n = \hbar \omega (\frac{3}{2} + n_x + n_y + n_z)$$
\begin{align*}
    n & = 0 & E_n & = \frac{3}{2}\hbar\omega & \text{Število stanj: } & 2 \\
    n & = 1 & E_n & = \frac{5}{2}\hbar\omega & \text{Število stanj: } & 6 \\
    n & = 2 & E_n & = \frac{7}{2}\hbar\omega & \text{Število stanj: } & 12 \\
    n & = 3 & E_n & = \frac{9}{2}\hbar\omega & \text{Število stanj: } & 20 \\
\end{align*}
Število stanj je odvisno od števila možnih kombinacij $n_x, n_y, n_z$, da bo $n_x + n_y + n_z = n$. Število kombinacij pa nato pomnožimo z 2, saj imamo dva možna spina.
Če nadaljujemo to zaporedje, pri $n=4$ dobimo $30$ možnih stanj, za $n = 5$ pa $42$ možnih stanj.
Zdaj poglejmo delne vsote tega zaporedja:
\begin{align*}
    n & = 0 & \Sigma_n & = 2 \\
    n & \leq 1 & \Sigma_n & = 8 \\
    n & \leq 2 & \Sigma_n & = 20 \\
    n & \leq 3 & \Sigma_n & = 40 \\
    n & \leq 4 & \Sigma_n & = 70 \\
    n & \leq 5 & \Sigma_n & = 112 \\
\end{align*}
Opazimo: Prve tri delne vsote se ujemajo z magičnimi števili. Ta so $2, 8, 20, 28, 50, 82, 126$. Zakaj se od prvih treh členov naprej ne ujemajo več? Očitno je del težave naš približek,
da je potencial enak potencialu harmonskega oscilatorja.
\paragraph{Alternativno reševanje:} Namesto $R(\vct{r}) = X(x)Y(y)Z(z)$ vzemimo $R(\vct{r}) = \mathcal{R}(r)\mathcal{Y}_{lm}(\vartheta, \varphi)$.
Tedaj dobimo lastna stanja:
$$n = 2n_r + l + 1$$
$$E_n = \hbar\omega\left(n + \frac{1}{2}\right)$$
Število stanj pri določenem $l$ je $2l + 1$, na podlagi $l$ določimo tudi oznake stanj:
\begin{table}[h!]
    \centering
    \begin{tabular}{c|c}
        $l$ & Oznaka \\
        \hline
        0 & s \\
        1 & p \\
        2 & d \\
        3 & f \\
        4 & g \\
        5 & h \\
        6 & i \\
    \end{tabular}
\end{table}
Poleg tega moramo zapisati tudi, kolikokrat se je določeno stanje že pojavilo (npr. 1s, 2s, 3s itd.) Vidimo, da bo stanje z višjim $l$ boljše vezano kot stanje z nižjim $l$.
Tudi to pa nam ne da pričakovanih magičnih števil (dobili smo iste rešitve kot prej, le v drugem koordinatnem sistemu).
\paragraph{Popravek Hamiltonove funkcije.} V Hamiltonovi funkciji upoštevajmo sklopitev spin-tir.
$$H \to H - C\dd{V}{r} \vct{l} \cdot \vct{s}$$
$$E \to E + \avg{j, j_3, l, s|-C\dd{V}{r}\hat{\vct{l}}\cdot\hat{\vct{s}}|j, j_3, l, s}$$
Tu je $\displaystyle{-C\dd{V}{r}}$ neodvisna od $j, j_3, l$ ter $s$ - torej konstanta - zato računajmo le $\avg{j, j_3, l, s|\hat{\vct{l}}\cdot\hat{\vct{s}}|j, j_3, l, s}$.
$$\avg{j, j_3, l, s|\hat{\vct{l}}\cdot\hat{\vct{s}}|j, j_3, l, s} = \frac{1}{2}\avg{j, j_3, l, s|\hat{\vct{j}}^2 - \hat{\vct{l}}^2 - \hat{\vct{s}}^2|j, j_3, l, s}$$
$$= \frac{1}{2}\left(j(j+1) - l(l+1) - s(s+1)\right)$$
Ker je $s$ omejen na eno od vrednosti $\displaystyle{-\frac{1}{2}, \frac{1}{2}}$ in velja $\vct{j} = \vct{l} + \vct{s}$, velja:
$$E \to E - C\dd{V}{r}\cdot\begin{cases}
    \frac{1}{2}\,l \\
    - \frac{1}{2}\,(l+1)
\end{cases}$$
Dobimo degeneracijo stanj:
$\displaystyle{\Delta E = E_{s = 1/2} - E_{s = - 1/2} \propto - (l + \frac{1}{2})}$
\\
Kot pri orbitalah atomov lahko dojemo to kot nekakšen lupine, ki se polnijo z nukleoni. Posebej se polnijo lupine s protoni in nevtroni.
Se pa protoni in nevtroni paroma sklopijo v vrtilno količino $j = 0$.
\paragraph{Parnost.} Operator parnosti označimo s $\hat{P}$.
$$\hat{P} \psi = \psi ~~\text{Soda parnost}$$
$$\hat{P} \psi = -\psi ~\text{Liha parnost}$$
Na primer za sferični harmonik $\mathcal{Y}_{lm}$:
$$\hat{P}(\mathcal{Y}_{lm}) = (-1)^l$$
\paragraph{Vzbujena stanje jeder.} Delimo jih na več vrst. \\[3mm]
1. Enodelčna vezana stanja: Ko jedru dodamo nevtron, tipično so energijske razlike reda 1 MeV. \\[3mm]
2. Rotacijska vzbujena stanja: Če je jedro elipsoidno (tj. daleč od magičnega števila - ta so okrogla)
ima rotacijsko energijo $$W = \frac{\hbar^2j(j+1)}{2I},$$ kjer je $I$ vztrajnostni moment. 
\end{document}