\documentclass[a4paper]{article}
\usepackage{amsmath, amssymb, amsfonts}
\usepackage[margin=1in]{geometry}
\usepackage{graphicx}
\usepackage{tikz}
\setlength{\parindent}{0em}
\setlength{\parskip}{1ex}
\newcommand{\vct}[1]{\overrightarrow{#1}}
\newcommand{\pd}[2]{\frac{\partial {#1}}{\partial {#2}}}
\newcommand{\dd}[2]{\frac{\mathrm{d} {#1}}{\mathrm{d} {#2}}}
\newcommand{\C}{\mathbb{C}}
\newcommand{\R}{\mathbb{R}}
\newcommand{\Q}{\mathbb{Q}}
\newcommand{\Z}{\mathbb{Z}}
\newcommand{\N}{\mathbb{N}}
\newcommand{\fn}[3]{{#1}\colon {#2} \rightarrow {#3}}
\newcommand{\avg}[1]{\langle {#1} \rangle}
\newcommand{\Sum}[2][0]{\sum_{{#2} = {#1}}^{\infty}}
\newcommand{\Lim}[1]{\lim_{{#1} \rightarrow \infty}}
\newcommand{\Binom}[2]{\begin{pmatrix} {#1} \cr {#2} \end{pmatrix}}


\begin{document}
Z Rutherfordovim poskusom (sipanje delcev $\alpha$ na zlati plošči) je pokazal, da je atom sestavljen iz majhnega jedra, okoli katerega krožijo elektroni. Z nadaljnjimi poskusi so pokazali tudi, da masno in vrstno število nista enaki.
To je pomenilo, da je jedro sestavljeno iz pozitivnih in nevtralnih delcev, ki jih skupaj drži t. i. močna jedrska sila. Potencial te sile je enak
$$V(r) \propto \frac{e^{-r/a}}{r},~a=\frac{\hbar}{mc}$$
Nosilci te sile so pioni ($m_\pi = 0.1\,GeV$). \\
Delce dobimo iz radioaktivnih izotopov, vesolja ali pospeševalnikov. Zaznamo jih lahko z meglično celico, v kateri se delci ustavljajo in puščajo sled. Alternativno jih lahko zaznamo s fotografsko ploščo, ki spremeni barvo, ko se vanjo zaleti nabit delec (ali svetloba). \\
\paragraph{Pozitron.} Skozi svinčeno ploščo pošljemo nabit delec. Na fotografski ploščici dobimo sled. Na podlagi spreminjanja ukrivljenosti zaradi izgube naboja (kajti vemo, da je $p=eBR$) lahko določimo njegov naboj (pozitiven), na podlagi izgub gibalne količine pa
njegovo maso (ki je enaka masi elektrona). S tem smo našli "pozitiven elektron" ali pozitron. Prvi je ta eksperiment opravil C. powell, ki je pozneje odkril tudi muon in pion. \\
Na podlagi eksperimentov z meglično celico so začeli graditi prve pospeševalnike, prvo večje odkritje katerih je bil Kaon - delec, ki razpade na dva piona. \\
\paragraph{Čudni delci:} Čudni delci nastanejo v parih + in -, nobeden od njiju ni nevtralen delec. Primer so kaoni. Delci so lahko tudi večkrat čudni - če je delec npr. 3-krat čuden, mora priti do treh razpadov, da dobimo ne čudne delce.
Te osnovne delce lahko razdelimo na vrsti (barioni in mezoni), in sestavimo nekakšen periodni sistem teh delcev. Ker je bilo teh delcev veliko, se je pojavila ideja,
da se jih da razdeliti na še bolj osnovne delce: Do leta 1974 so mislili, da so le trije, leta 1974 pa sta raziskovala odkrila delec $J/\psi$ (eden od njiju ga je poimenoval $J$, drugi pa $\psi$. Nista se uspela dogovoriti, kdo ga je odkril prej.)
Ta je bil sestavljen iz kvarkov $c$ in $c^-$. Leta 1977 je bil odkrit kvark $b$, leta 1995 pa kvark $t$.
\paragraph{Standardni model.} Dve vrsti delcev: kvarki in leptoni. Tri vrste interakcij. In Higgsov bozon.
Sile med delci poteka preko nosilcev sile. Elektromagnetno silo prenašajo fotoni, šibko prenašajo šibki bozoni, močno pa gluoni.
\paragraph{Šibka sila.} Pri prehodu, ki ga povzroča šibka sila, se spremeni vrsta kvarka. Najbolj znan je prehod $u \leftrightarrow d$, ki ga poznamo kot $\beta$ razpad. Množno je več prehodov, niso pa vsi enako verjetni.
\paragraph{Poskusi v fiziki osnovnih delcev.} Novi delci lahko nastanejo pri trkih med osnovnimi delci. Te poskuse delimo na poskuse s fiksno tarčo in poskuse v trkalniku (kjer dva delca pospešimo enega proti drugemu). Drugi način je težje izvesti, a lahko z njim pridemo do večjih energij.
Delce pospešujemo z elektromagnetnim valovanjem.
\paragraph{Sevanje Čerenkova.} Soroden s fronto pri nadzvočnem letu: Iz kota Machovega stožca lahko sklepamo, kakšno hitrost ima delec. Pri svetlobi je podobno: $$c/v = \cos\varphi$$
\end{document}