\documentclass[a4paper]{article}
\usepackage{amsmath, amssymb, amsfonts}
\usepackage[margin=1in]{geometry}
\usepackage{graphicx}
\usepackage{tikz}
\usepackage{esint}
\setlength{\parindent}{0em}
\setlength{\parskip}{1ex}
\newcommand{\vct}[1]{\overrightarrow{#1}}
\newcommand{\dif}{\mathrm{d}}
\newcommand{\pd}[2]{\frac{\partial {#1}}{\partial {#2}}}
\newcommand{\dd}[2]{\frac{\mathrm{d} {#1}}{\mathrm{d} {#2}}}
\newcommand{\C}{\mathbb{C}}
\newcommand{\R}{\mathbb{R}}
\newcommand{\Q}{\mathbb{Q}}
\newcommand{\Z}{\mathbb{Z}}
\newcommand{\N}{\mathbb{N}}
\newcommand{\fn}[3]{{#1}\colon {#2} \rightarrow {#3}}
\newcommand{\avg}[1]{\langle {#1} \rangle}
\newcommand{\Sum}[2][0]{\sum_{{#2} = {#1}}^{\infty}}
\newcommand{\Lim}[1]{\lim_{{#1} \rightarrow \infty}}
\newcommand{\Binom}[2]{\begin{pmatrix} {#1} \cr {#2} \end{pmatrix}}
\newcommand{\duline}[1]{\underline{\underline{#1}}}

\begin{document}
\paragraph{Numerično reševanje (navadnih) diferencialnih enačb.} Rešujemo začetni problem prvega reda:
$$y'(x) = f(x, y(x)),~~y(a) = y_a$$
$$y' = f(x, y),~~y(a) = y_a$$
Numerično seveda ne bomo dobili analitične rešitve, lahko pa poiščemo njen približek v čim več točkah. 
Želimo torej množico točk $y_i \approx y(x_i)$. Imamo začetni pogoj $y_0 = y(x_0)$. Ostale točke izračunamo na dva načina:
z eksplicitno ali implicitno metodo. \\
Eksplicitna metoda: Nov približek izračunamo neposredno iz prejšnjih. \\
Implicitna metoda: Nov približek dobimo z reševanjem enačbe. Ta metoda je zahtevnejša, a pogosto zanesljivejša. \\
Obe metodi delimo na enočlensko in veččlensko - gre za to, ali pri računanju točke vzamemo eno ali več že znanih točk.
\paragraph{Lokalna in globalna napaka.} Lokalna napaka je razlika med $y_{n+1} in z(x_{n+1})$, kjer je $z$ rešitev diferencialne enačbe ($z' = f(x, z)$).
Globalna napaka je vsota vseh lokalnih napak. \\
Lokalna napaka je reda $p \in \N$, če je
$y_{n+1} - z(x_{n+1}) \propto h^{p+1} + \mathcal{O}(h^{p+2})$, kjer je $h$ razlika med točkami $x_i$.
\paragraph{Eulerjeva metoda.} Gre v bistvu za prvi člen Taylorjevega razvoja.
$$y_{n+1} = y_n + hf(x_n, y_n)$$
Izpeljava reda eksplicient Eulerjeve metode:
$$y_{n+1} - y(x_{n+1}) =y_n + h(fx_n, y_n) - y(x_n + h)$$
Uporabimo Taylorjev razvoj.
$$y(x_n) + hf(x_n, y(x_n)) - \left(y(x_n) + hy'(x_n) + \frac{h^2}{2}y''(x_n) + ...\right)$$
Upoštevamo $y'(x_n) = f(x_n, y(x_n))$ in pokrajšamo, kar je moč pokrajšati.
$$= -\frac{h^2}{2}y''(x_n) + \mathcal{O}(h^3)$$
Ostanejo nam členi reda velikosti $h^2$, torej je metoda reda 1.
$$y''(x) = f_x(x, y) + f_y(x, y) y' = f_x(x, y) + f_y(x, y) f(x, y)$$
\paragraph{Metoda Runge-Kutta.} Tu je npr. metoda Range-Kutta četrtega reda.
\begin{align*}
    k_1 & = hf(x_n, y_n) \\
    k_2 & = hf(x_n + h/2, y_n + k_1/2) \\
    k_3 & = hf(x_n + h/2, y_n + k_2/2) \\
    k_4 & = hf(x_n + h, y_n + k_3) \\
    y_{n+1} & = y_n + \frac{1}{6}(k_1 + 2k_2 + 2k_3 + k_4) \\
\end{align*}
Koeficienti so nastavljeni tako, da se pri analizi napake vsi členi do $h^5$ pokrajšajo.
Da se poiskati tudi metode, ki nam dajo višji red, vendar na neki točki postane nepraktično.
\paragraph{Eksplicitne veččlenske metode.} Dobimo jih s pomočjo interpolacijeskega polinoma za funkcijo $f$. Ker potrebujemo prvih $k$
točk, moramo najprej uporabiti enočlensko metodo, da jih dobimo. Primer dvočlenske metode je
$$y_{n+1} = y_{n} + \frac{3}{2}\left(3f(x_n, y_n) - f(x_{n-1}, y_{n-1})\right)$$
Običajno je najbolje, da je enočlenska metoda, ki jo uporabimo za račun prvih $k$ členov, istega reda kot izbrana veččlenska metoda
(nižji red pomeni manj zanesljivi začetni podatki, višji red pa je nepotrebno časovno in spominsko zahteven.)
\paragraph{Enokoračne implicitne metode.} Splošna oblika implicitne metode je $$y_{n+1} = \phi(h, x_n, y_n, y_{n+1}, f)$$
Za izračun novega približka moramo torej rešiti enačbo, ki bo običajno nelinearna. Za reševanje uporabimo kar iteracijo oblike
$$y_{n+1} = \tilde{\phi}(y_{n+1})$$
Začetni približek za $y_{n+1}$ pa dobimo po neki eksplicitni metodi. \\
Implicitna Eulerjeva metoda:
$$y_{n+1} = y_{n} + hf(x_{n+1}, y_{n+1})$$
Primer:
$$f(x, y) = xy$$
$$y_{n+1} = \frac{1}{1-hx_{n+1}}y_n$$
Običajno nimamo tako preprostih funkcij.
\paragraph{Trapezna metoda.} $$y_{n+1} * y_n + \frac{h}{2}(f(x_n, y_n) + f(x_{n+1}, y_{n+1}))$$
Metoda je drugega reda (lokalna napaka reda velikosti $h^3$).
Navadna iteracija za $y_{n+1} = \tilde{\phi}(y_{n+1})$ konvergira, če je $$\left|\pd{\phi}{y}(h, x_n, y_n, y, f)\right| \leq 1,$$
kar je pri dovolj majhnih $h$ po navadi izpolnjeno. \\
Če imamo opravka z več funkcijami, imamo vektorsko enačbo, ki pa jo rešujemo na podoben način.
\end{document}